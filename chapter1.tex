\chapter{Εισαγωγή}
\label{chapter1}
Οι γιατροί κάνουν διαγνώσεις με βάση το ιατρικό ιστορικό, βιοδείκτες και τις εξετάσεις μεμονωμένων ασθενών, τα οποία ερμηνεύουν σύμφωνα με την προσωπική τους κλινική εμπειρία.
Στη συνέχεια, αντιστοιχούν τον κάθε ασθενή στην παραδοσιακή ταξινόμηση των ιατρικών νοσημάτων/παθήσεων σύμφωνα με μια υποκειμενική ερμηνεία της ιατρικής βιβλιογραφίας.
Αυτή η διαδικασία είναι όλο και περισσότερο επιρρεπής σε σφάλματα.
Επιπλέον, οι τεχνολογίες απεικονίσεων αυξάνουν συνεχώς την ικανότητά τους να παράγουν μεγάλες ποσότητες δεδομένων, τα οποία οι ειδικοί αδυνατούν να καταλάβουν και να χρησιμοποιήσουν αποτελεσματικά, καθιστώντας το έργο τους δυσκολότερο.
Επομένως, απαιτείται η αυτοματοποίηση των ιατρικών διαδικασιών για την αύξηση της ποιότητας της υγείας των ασθενών και τη μείωση του κόστους των συστημάτων υγειονομικής περίθαλψης.

Η ανάγκη για αυτοματοποίηση των ιατρικών διαδικασιών κυμαίνεται από τη διάγνωση έως τη θεραπεία και ιδίως σε περιπτώσεις/περιοχές όπου υπάρχει έλλειψη υγειονομικής περίθαλψης.
Προηγούμενες προσπάθειες αυτοματοποίησης περιλαμβάνουν συστήματα εμπειρογνωμόνων βασισμένα σε κανόνες, τα οποία έχουν σχεδιαστεί να μιμούνται τη διαδικασία που ακολουθούν οι ιατρικοί εμπειρογνώμονες όταν επιλύουν ιατρικά προβλήματα.
Αυτά τα συστήματα έχουν αποδειχθεί αναποτελεσματικά επειδή απαιτούν χειροκίνητη δημιουργία χαρακτηριστικών και γνώση ειδικών πάνω στο πρόβλημα για να επιτευχθεί επαρκής ακρίβεια και επίσης είναι δύσκολο να εμφανίσουν γραμμική βελτίωση με την παρουσία νέων δεδομένων.

Η βαθιά μάθηση έχει αναδειχθεί ως μια πιο ακριβής και αποτελεσματική τεχνολογία σε ένα ευρύ φάσμα ιατρικών προβλημάτων όπως η διάγνωση, η πρόβλεψη και η παρέμβαση.
Είναι μια μέθοδος μάθησης αναπαραστάσεων που αποτελείται από επίπεδα που μετασχηματίζουν τα δεδομένα με μη-γραμμικό τρόπο, αποκαλύπτοντας έτσι ιεραρχικές σχέσεις και δομές.
Παρόλο που η βαθιά μάθηση έχει εφαρμοστεί με επιτυχία σε πολλούς τομείς, αυτό έρχεται με κόστος της ερμηνευσιμότητας των αναπαραστάσεων, καθώς αυτές αποτελούνται από εκατομμύρια παραμέτρους.

Σκοπός της διδακτορικής διατριβής είναι η δημιουργία μιας νέας αρχιτεκτονικής νευρωνικών δικτύων, τα \textbf{Δίκτυα Αραιής Ενεργοποίησης} (Sparsely Activated Networks, SANs), τα οποία αποσυνθέτουν τις εισόδους τους σε ένα σύνολο από αραιά επαναλαμβανόμενα πρότυπα διαφορετικού πλάτους και σε συνδυασμό με ένα προτεινόμενο μέτρο $\varphi$ μαθαίνουν αναπαραστάσεις με ελάχιστα μήκη περιγραφής, αυξάνοντας έτσι την ερμηνευσιμότητα του μοντέλου.

\section{Οργάνωση τόμου}
Στο Κεφάλαιο~\ref{chapter2}, παρουσιάζουμε τις θεμελιώδεις έννοιες των νευρωνικών δικτύων και της βαθιάς μάθησης καθώς επίσης και τις γενικές ιδιότητες των πιο συχνά χρησιμοποιούμενων αρχιτεκτονικών.

Στα Κεφάλαια~\ref{chapter3} και~\ref{chapter4} παρουσιάζουμε μια βιβλιογραφική επισκόπηση των αρχιτεκτονικών βαθιάς μάθησης σε δομημένα δεδομένα, σήματα και απεικονίσεις που έχουν χρησιμοποιηθεί στην ιατρική και ιδιαίτερα στην καρδιολογία.
Συζητάμε τα πλεονεκτήματα και τους περιορισμούς των εφαρμογών της βαθιάς μάθησης στην ιατρική, ενώ προτείνουμε ορισμένες κατευθύνσεις ως τις πιο βιώσιμες για κλινική χρήση.
Πιο συγκεκριμένα στις ενότητες~\ref{sec3:structured} και~\ref{sec3:signals} παρουσιάζουμε τις εφαρμογές βαθιάς μάθησης με χρήση δομημένων δεδομένων και μορφές σημάτων, ενώ στην ενότητα~\ref{sec4:discussion} παρουσιάζουμε πλεονεκτήματα και περιορισμούς των εφαρμογών βαθιάς μάθησης στην καρδιολογία και προτείνουμε κάποιες κατευθύνσεις για την υλοποίηση μοντέλων βαθιάς μάθησης που μπορούν να εφαρμοστούν κλινικά.

Στο Κεφάλαιο~\ref{chapter5} προτείνουμε τα Signal2Image (S2Is) ως εκπαιδεύσιμα επίπεδα προθέματος νευρωνικών δικτύων, τα οποία μετατρέπουν σήματα, όπως το ηλεκτροεγκεφαλογράφημα (Electroencephalogram, EEG), σε αναπαραστάσεις εικόνων, καθιστώντας τα κατάλληλα για την εκπαίδευση βαθιών νευρωνικών δικτύων βασισμένα σε εικόνες, τα οποία ορίζονται ως `μοντέλα βάσης'.
Συγκρίνουμε την ακρίβεια και τις επιδόσεις τεσσάρων S2Is (`σήμα ως εικόνα', φασματογράφημα, CNN ενός και δύο επιπέδων) σε συνδυασμό με ένα σύνολο `μοντέλων βάσης' μαζί με μοντέλα διαφορετικού βάθους και 1D παραλλαγές των τελευταίων.
Παρέχουμε επίσης εμπειρικές αποδείξεις ότι το CNN S2I ενός επιπέδου αποδίδει καλύτερα σε 11 από τα 15 μοντέλα που δοκιμάστηκαν σε σύγκριση με τα μη-εκπαιδεύσιμα S2Is για την ταξινόμηση σημάτων EEG και οπτικοποιούμε τις εξόδους κάποιων από τα S2Is.

Στο Κεφάλαιο~\ref{chapter6}, ενότητα~\ref{sec6:flethos} ορίζουμε το μέτρο $\varphi$, έπειτα στην ενότητα~\ref{sec6:sans} ορίζουμε τις πέντε συναρτήσεις ενεργοποίησης που θα συγκριθούν καθώς επίσης και την αρχιτεκτονική και διαδικασία εκπαίδευσης των SAN, στην ενότητα~\ref{sec6:experiments} δοκιμάζουμε τα SANs στις βάσεις δεδομένων Physionet, UCI-epilepsy και MNIST και οπτικοποιούμε τις ενδιάμεσες αναπαραστάσεις και αποτελέσματα.
Στην ενότητα~\ref{sec6:discussion} συζητάμε τα ευρήματα των πειραμάτων και τους περιορισμούς των SAN\@.

Τέλος στο Κεφάλαιο~\ref{chapter7} παρουσιάζουμε τα τελικά συμπεράσματα και προτείνουμε πιθανές μελλοντικές κατευθύνσεις.
