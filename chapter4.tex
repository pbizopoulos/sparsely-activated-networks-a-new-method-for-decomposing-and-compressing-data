\chapter{Βαθιά μάθηση με απεικονίσεις}
\label{chapter4}
\graphicspath{{./images/deep-learning-in-cardiology/}}

\section{Εισαγωγή}
Οι μέθοδοι απεικόνισης που έχουν βρει χρήση στην καρδιολογία περιλαμβάνουν τη τομογραφία μαγνητικού συντονισμού (Magnetic Resonance Imaging, MRI), την Fundus, την ηλεκτρονική τομογραφία (Computerized Tomography, CT), το ηχοκαρδιογράφημα, την τομογραφία οπτικής συνοχής (Optical Coherence Tomography, OCT), το ενδοαγγειακό υπερηχογράφημα (Intravascular Ultrasound, IVUS) και άλλες.
Η βαθιά μάθηση υπήρξε ως επί το πλείστον επιτυχής σε αυτόν τον τομέα, κυρίως λόγω αρχιτεκτονικών που χρησιμοποιούν μεγάλο αριθμό συνελικτικών επιπέδων.
Μια σύνοψη των εφαρμογών βαθιάς μάθησης που χρησιμοποιούν απεικονίσεις παρουσιάζονται στους Πίνακες~\ref{table:imaging1},~\ref{table:imaging2},~\ref{table:imaging3},~\ref{table:imaging4},~\ref{table:imaging5},~\ref{table:imaging6},~\ref{table:imaging7} και~\ref{table:imaging8}.

\begin{sidewaystable}
	\caption{Δημόσιες καρδιολογικές βάσεις δεδομένων, MRI}
	\label{table:cardiologypublicdatabases2}
	\centering
	\begin{tabular}{l c r l}
		\toprule
		\thead{Βάση Δεδομένων} & \thead{Ακρωνύμιο} & \thead{Ασθενείς}    & \thead{Πρόβλημα}                                  \\
		\midrule
		MICCAI 2009 Sunnybrook~\cite{radau2009evaluation}                                                          & SUN09             & 45                  & κατάτμηση LV                                      \\
		MICCAI 2011 Left Ventricle Segmentation STACOM~\cite{fonseca2011cardiac}                                   & STA11             & 200                 & κατάτμηση LV                                      \\
		MICCAI 2012 Right Ventricle Segmentation Challenge~\cite{petitjean2015right}                               & RV12              & 48                  & κατάτμηση RV                                      \\
		MICCAI 2013 SATA~\cite{asman2013miccai}                                                                    & SAT13             & ---\footnote{Ο αριθμός των ασθενών δεν αναφέρεται.} & κατάτμηση LV                                      \\
		MICCAI 2016 HVSMR~\cite{pace2015interactive}                                                               & HVS16             & 20                  & κατάτμηση καρδιάς                                 \\
		MICCAI 2017 ACDC~\cite{bernard2018deep}                                                                    & AC17              & 150                 & κατάτμηση LV/RV                                   \\
		York University Database~\cite{andreopoulos2008efficient}                                                  & YUDB              & 33                  & κατάτμηση LV                                      \\
		Data Science Bowl Cardiac Challenge Data~\cite{dsbcdc2016}                                                 & DS16              & 1140                & εκτίμηση όγκου LV μετά την συστολή και διαστολή \\
		\bottomrule
	\end{tabular}
\end{sidewaystable}

\begin{sidewaystable}
	\caption{Δημόσιες καρδιολογικές βάσεις δεδομένων, Fundus και άλλες}
	\label{table:cardiologypublicdatabases3}
	\centering
	\begin{tabular}{l c r l}
		\toprule
		\thead{Βάση Δεδομένων} & \thead{Ακρωνύμιο} & \thead{Ασθενείς}    & \thead{Πρόβλημα}                                                \\
		\midrule
		\multicolumn{4}{l}{\thead{Βάσεις δεδομένων αμφιβληστροειδούς (όλα Fundus εκτός του~\cite{zhang2016robust})}}                                                                                                           \\
		\midrule
		Digital Retinal Images for Vessel Extraction~\cite{staal2004ridge}                                         & DRIVE             & 40                  & κατάτμηση αγγείων                                               \\
		Structured Analysis of the Retina~\cite{hoover2000locating}                                                & STARE             & 20                  & κατάτμηση αγγείων                                               \\
		Child Heart and Health Study in England Database~\cite{owen2009measuring}                                  & CHDB              & 14                  & κατάτμηση αγγείων                                               \\
		High Resolution Fundus~\cite{odstrcilik2013retinal}                                                        & HRF               & 45                  & κατάτμηση αγγείων                 \\
		Kaggle Retinopathy Detection Challenge 2015~\cite{graham2015kaggle}                                        & KR15              & \_\_\textsuperscript{b} & ταξινόμηση διαβητικής αμφιβληστροειδοπάθειας                    \\
		TeleOptha~\cite{decenciere2013teleophta}                                                                   & e-optha           & 381                 & MA και ανίχνευση αιμορραγίας                                    \\
		Messidor~\cite{decenciere2014feedback}                                                                     & Messidor          & 1200                & διάγνωση διαβητικής αμφιβληστροειδοπάθειας                      \\
		Messidor2~\cite{decenciere2014feedback}                                                                    & Messidor2         & 874                 & διάγνωση διαβητικής αμφιβληστροειδοπάθειας                      \\
		Diaretdb1~\cite{kauppi2013constructing}                                                                    & DIA               & 89                  & MA και ανίχνευση αιμορραγίας                                    \\
		Retinopathy Online Challenge~\cite{niemeijer2010retinopathy}                                               & ROC               & 100                 & ανίχνευση MA                                                    \\
		IOSTAR~\cite{zhang2016robust}                                                                              & IOSTAR            & 30                  & κατάτμηση αγγείων με χρήση SLO                                  \\
		RC-SLO~\cite{zhang2016robust}                                                                              & RC-SLO            & 40                  & κατάτμηση αγγείων με χρήση SLO                                  \\
		\midrule
		\multicolumn{4}{l}{\thead{Άλλες βάσεις δεδομένων απεικόνισης}}                                                                                                                                                        \\
		\midrule
		MICCAI 2011 Lumen+External Elastic Laminae~\cite{balocco2014standardized}                                  & IV11              & 32                  & κατάτμηση περιγράμματος lumen, external IVUS \\
		UK Biobank~\cite{sudlow2015uk}                                                                             & UKBDB             & ---\footnote{Ο αριθμός των ασθενών δεν αναφέρεται.} & πολλαπλές βάσεις δεδομένων απεικόνισης                          \\
		Coronary Artery Stenoses Detection and Quantification~\cite{kiricsli2013standardized}                      & CASDQ             & 48                  & αγγειογραφία CT για στένωση κορωνιαίων αρτηριών        \\
		\midrule
		\multicolumn{4}{l}{\thead{Πολυτροπικές βάσεις δεδομένων}}                                                                                                                                                                      \\
		\midrule
		VORTAL~\cite{charlton2016assessment}                                                                       & VORTAL            & 45                  & εκτίμηση ρυθμού αναπνοής με χρήση ECG και PCG                   \\
		Left Atrium Segmentation Challenge STACOM 2013~\cite{tobon2015benchmark}                                   & STA13             & 30                  & κατάτμηση αριστερού καρδιακού κόλπου με MRI, CT                 \\
		MICCAI MMWHS 2017~\cite{zhuang2016multi}                                                                   & MM17              & 60                  & 120 εικόνες για κατάτμηση της καρδιάς με MRI, CT                \\
		\bottomrule
	\end{tabular}
\end{sidewaystable}

\section{Τομογραφία μαγνητικού συντονισμού}
Η τομογραφία μαγνητικού συντονισμού (MRI) βασίζεται στην αλληλεπίδραση μεταξύ ενός συστήματος ατομικών πυρήνων και ενός εξωτερικού μαγνητικού πεδίου παρέχοντας μια εικόνα του εσωτερικού ενός φυσικού αντικειμένου~\cite{sebastiani1991mathematical}.
Οι κύριες χρήσεις της MRI περιλαμβάνουν την κατάτμηση της αριστερής κοιλίας (Left Ventricle, LV), της δεξιάς κοιλίας (Right Ventricle, RV) και ολόκληρης της καρδιάς.

\subsection{Κατάτμηση LV}
Τα CNNs χρησιμοποιήθηκαν για κατάτμηση LV με MRI\@.
Οι Tan et al.~\cite{tan2016cardiac} χρησιμοποίησαν ένα CNN για να εντοπίσουν το ενδοκάρδιο του LV και ένα επιπλέον CNN για να προσδιορίσουν την ακτίνα του ενδοκαρδίου, χρησιμοποιώντας τις STA11 και SUN09 για εκπαίδευση και αξιολόγηση αντίστοιχα.
Χωρίς φιλτράρισμα των εικόνων που απεικονίζουν τα apical ή την χρήση παραμορφώσιμων μοντέλων επιτυγχάνουν συγκρίσιμες επιδόσεις με προηγούμενες μεθόδους.
Στο~\cite{romaguera2017left} οι συγγραφείς εκπαίδευσαν ένα CNN πέντε \textit{επιπέδων} χρησιμοποιώντας MRI από την SUN09.
Εκπαίδευσαν το μοντέλο τους χρησιμοποιώντας SGD και RMSprop με το πρώτο να φτάνει Dice 92\%.

\begin{sidewaystable}
	\caption{Εφαρμογές βαθιάς μάθησης με χρήση MRI, για κατάτμηση LV}
	\label{table:imaging1}
	\centering
	\begin{tabular}{l c l l}
		\toprule
		\thead{Αναφορά}                            & \thead{Μέθοδος} & \thead{Εφαρμογή/Σημειώσεις\footnote{Σε παρένθεση οι βάσεις δεδομένων που χρησιμοποιήθηκαν.}}               & \thead{Dice\footnote{το ($\S$) υποδηλώνει `για κάθε βάση', το ($*$) υποδηλώνει μέσο τετραγωνικό σφάλμα για EF το ($+$) υποδηλώνει `για ενδοκαρδιακά και επικαρδιακά', το ($-$) υποδηλώνει ακρίβεια, και το ($\#$) υποδηλώνει `για CT και MRI'}} \\
		\midrule
		Tan 2016~\cite{tan2016cardiac}              & CNN             & CNN για εύρεση τοποθεσίας και CNN για ευθυγράμμιση ενδοκαρδιακών συνόρων (SUN09, STA11)                & 88\%                                                                                                                                                                                                                                            \\
		Romaguera 2017~\cite{romaguera2017left}     & CNN             & CNN πέντε \textit{επιπέδων} με SGD και RMSprop (SUN09)                                                     & 92\%                                                                                                                                                                                                                                            \\
		Poudel 2016~\cite{poudel2016recurrent}      & u-net, RNN      & συνδυασμός u-net και RNN (SUN09, μη-δημόσια)                                                               & \textit{πολλαπλά}                                                                                                                                                                                                                               \\
		Rupprecht 2016~\cite{rupprecht2016deep}     & CNN             & συνδυασμός ενός CNN τεσσάρων \textit{επιπέδων} με Sobolev (STA11, non-medical)                             & 85\%                                                                                                                                                                                                                                            \\
		Ngo 2014~\cite{anh2014fully}                & DBN             & συνδυασμός DBN με level set (SUN09)                                                                        & 88\%                                                                                                                                                                                                                                            \\
		Avendi 2016~\cite{avendi2016combined}       & CNN, AE         & CNN για εντοπισμό καρδιακού θαλάμου, AEs για συμπερασμό σχήματος (SUN09) & 96.69\%                                                                                                                                                                                                                                         \\
		Yang 2016~\cite{yang2016deep}               & CNN             & δίκτυο εξαγωγής χαρακτηριστικών και ένα μη-τοπικό δίκτυο σύντηξης επισημάνσεων (SAT13)                     & 81.6\%                                                                                                                                                                                                                                          \\
		Luo 2016~\cite{luo2016cardiac}              & CNN             & άτλας αντιστοίχισης του LV και ένα CNN τριών \textit{επιπέδων} (DS16)                                      & 4.98\%$^*$                                                                                                                                                                                                                                      \\
		Yang 2017~\cite{yang2017deep}               & CNN, u-net      & εντοπισμός με CNN παλινδρόμησης και κατάτμηση με u-net (YUDB, SUN09)                                       & 91\%, 93\%$^\S$                                                                                                                                                                                                                                 \\
		Tan 2017~\cite{tan2017convolutional}        & CNN             & CNN παλινδρόμησης (STA11, DS16)                                                                            & \textit{πολλαπλά}                                                                                                                                                                                                                               \\
		Curiale 2017~\cite{curiale2017automatic}    & u-net           & residual u-net (SUN09)                                                                                     & 90\%                                                                                                                                                                                                                                            \\
		Liao 2017~\cite{liao2017estimation}         & CNN             & τοπικό δυαδικό μοτίβο για εντοπισμό και FCN για κατάτμηση (DS16)                                           & 4.69\%$^*$                                                                                                                                                                                                                                      \\
		Emad 2015~\cite{emad2015automatic}          & CNN             & εντοπισμός LV με χρήση CNN και πυραμίδες κλιμάκων (YUDB)                                                   & 98.66\%$^-$                                                                                                                                                                                                                                     \\
		\bottomrule
	\end{tabular}
\end{sidewaystable}

Χρησιμοποιήθηκαν επίσης συνδυασμοί CNN με RNNs.
Στο~\cite{poudel2016recurrent} οι συγγραφείς δημιούργησαν ένα recurrent u-net που μαθαίνει αναπαραστάσεις από μια στοίβα από 2D και έχει την ικανότητα να αξιοποιεί τις χωρικές εξαρτήσεις μεταξύ των τμημάτων μέσω εσωτερικών μονάδων μνήμης.
Συνδυάζει ανίχνευση ανατομίας και κατάτμηση σε μια ενιαία αρχιτεκτονική από-άκρο-σε-άκρο, επιτυγχάνοντας συγκρίσιμα αποτελέσματα με άλλες μεθόδους, ξεπερνώντας τις βάσεις αναφοράς για το DBN, τα recurrent DBN και FCN όσον αφορά το Dice.

Άλλες δημοσιεύσεις συνδυάζουν μεθόδους βαθιάς μάθησης με το level-set για την κατάτμηση της LV\@.
Οι Rupprecht et al.~\cite{rupprecht2016deep} εκπαίδευσαν ένα CNN τεσσάρων \textit{επιπέδων}, που προβλέπει ένα διάνυσμα που υποδεικνύει το σημείο στο εξελισσόμενο περίγραμμα προς το πλησιέστερο σημείο στο όριο του αντικειμένου ενδιαφέροντος.
Αυτές οι προβλέψεις σχημάτισαν ένα διανυσματικό πεδίο το οποίο στη συνέχεια χρησιμοποιήθηκε για την εξέλιξη του περιγράμματος, χρησιμοποιώντας το πλαίσιο ενεργού περιγράμματος Sobolev.
Οι Anh et al.~\cite{anh2014fully} δημιούργησαν μια μέθοδο μη-άκαμπτης κατάτμησης βασισμένη στη level-set μέθοδο ρυθμισμένη ανάλογα με την απόσταση, που αρχικοποιήθηκε από τα αποτελέσματα μιας δομημένης εξαγωγής από ένα DBN\@.
Οι Avendi et al.~\cite{avendi2016combined} χρησιμοποίησαν ένα CNN για να ανιχνεύσουν τον LV θάλαμο και στη συνέχεια χρησιμοποίησαν στοιβαγμένα AE για να συμπεράνουν το σχήμα της LV\@.
Στη συνέχεια το αποτέλεσμα ενσωματώθηκε σε παραμορφώσιμα μοντέλα για να βελτιωθεί η ακρίβεια και η ευρωστία της κατάτμησης.

Μέθοδοι που βασίζονται σε άτλαντες έχουν επίσης χρησιμοποιηθεί για την επίλυση αυτού του προβλήματος.
Οι Yang et al.~\cite{yang2016deep} δημιούργησαν ένα δίκτυο βαθιάς σύντηξης συνδυάζοντας ένα δίκτυο εξαγωγής χαρακτηριστικών και ένα μη-τοπικό δίκτυο σύντηξης επισημάνσεων βασισμένο σε patch.
Τα χαρακτηριστικά που δημιουργούνται κατά τη διάρκεια της μάθησης χρησιμοποιούνται περαιτέρω για τον ορισμό ενός μέτρου ομοιότητας για την επιλογή άτλα MRI\@.
Σύγκριναν τη μέθοδος τους με την ψηφοφορία με πλειοψηφία, τη σύντηξη ετικετών με βάση τα patch, την αντιστοίχιση patch πολλαπλών ατλάντων και το SVM με επαυξημένα χαρακτηριστικά επιτυγχάνοντας καλύτερα αποτελέσματα όσον αφορά την ακρίβεια.
Οι Luo et al.~\cite{luo2016cardiac} υιοθέτησαν μια μέθοδο χαρτογράφησης άτλα LV για να επιτευχθεί ακριβής εντοπισμός με χρήση δεδομένων MRI από το DS16.
Στη συνέχεια, ένα CNN τριών \textit{επιπέδων} εκπαιδεύτηκε για την πρόβλεψη του όγκου LV, πετυχαίνοντας συγκρίσιμα αποτελέσματα με τους νικητές του διαγωνισμού ACDC 2017 με βάση αναφοράς την μέση τετραγωνική ρίζα της τελικής διαστολής και των τελικών συστολικών όγκων.

Μέθοδοι παλινδρόμησης έχουν χρησιμοποιηθεί για τον εντοπισμό της LV πριν την κατάτμηση της.
Οι Yang et al.~\cite{yang2017deep} πρώτα εντόπισαν το LV σε ολόκληρη την εικόνα χρησιμοποιώντας CNN παλινδρόμησης και στη συνέχεια το ταξινόμησαν μέσα στην περιοχή ενδιαφέροντος (Region of Interest, ROI), χρησιμοποιώντας μια αρχιτεκτονική που βασίζεται στο u-net.
Το μοντέλο τους επιτυγχάνει υψηλή ακρίβεια με καλές υπολογιστικές επιδόσεις κατά τη διάρκεια των συμπερασμών.

Επίσης χρησιμοποιήθηκαν διάφορες άλλες μέθοδοι για την κατάτμηση του LV\@.
Οι Tan et al.~\cite{tan2017convolutional} παραμετροποίησαν όλες τις εικόνες του προβλήματος της τμηματοποίησης του LV με βάση τις ακτινικές αποστάσεις μεταξύ του κεντρικού σημείου LV και των ενδοκαρδιακών και επικαρδιακών περιγραμμάτων στις πολικές συντεταγμένες.
Στη συνέχεια, εκπαίδευσαν ένα CNN παλινδρόμησης στην STA11 για να συμπεραίνουν αυτές τις παραμέτρους και δοκίμασαν τη γενικευσιμότητα της μεθόδου στην DS16 παρουσιάζοντας καλά αποτελέσματα.
Στο~\cite{curiale2017automatic} οι συγγραφείς χρησιμοποίησαν την απόσταση Jaccard ως τη αντικειμενική συνάρτηση βελτιστοποίησης, ενσωματώνοντας μια residual στρατηγική μάθησης και εισάγοντας ένα επίπεδο κανονικοποίησης παρτίδας (batch normalization) για την εκπαίδευση ενός u-net.
Αποδείχθηκε ότι αυτή η διαμόρφωση είχε καλύτερα αποτελέσματα από άλλα απλά u-nets σε σχέση με το δείκτη Dice.
Στο άρθρο τους οι Liao et al.~\cite{liao2017estimation} ανίχνευσαν το ROI που περιείχε LV και έπειτα χρησιμοποίησαν FCN για να ταξινομήσουν τα LV μέσα στο ROI\@.
Τα αποτελέσματα της 2D τμηματοποίησης ενσωματώθηκαν μεταξύ διαφορετικών εικόνων για την εκτίμηση του όγκου.
Το μοντέλο εκπαιδεύτηκε εναλλάξ στην κατάτμηση του LV και στην εκτίμηση του όγκου, τοποθετώντας το τέταρτο στον διαγωνισμό του DS16.
Οι Emad et al.~\cite{emad2015automatic} εντόπισαν το LV χρησιμοποιώντας ένα CNN και μια ανάλυση πυραμίδας κλιμάκων, για να λάβουν υπόψη διαφορετικά μεγέθη της καρδιάς με την YUDB\@.
Πέτυχαν καλά αποτελέσματα, αλλά με σημαντικό κόστος υπολογισμού (10 δευτερόλεπτα ανά εικόνα κατά τη διάρκεια συμπερασμών).

\subsection{Κατάτμηση LV/RV}
Μια βάση δεδομένων που χρησιμοποιήθηκε για την κατάτμηση LV/RV ήταν η MICCAI 2017 ACDC Challenge (AC17), που περιέχει εικόνες MRI από 150 ασθενείς χωρισμένες σε πέντε ομάδες (φυσιολογικοί, προηγούμενο MI, διαστολή της καρδιομυοπάθειας, υπερτροφική καρδιομυοπάθεια, μη-φυσιολογική RV).
Οι Zotti et al.~\cite{zotti2017gridnet} χρησιμοποίησαν ένα μοντέλο που περιλαμβάνει μια καρδιακή μονάδα παλινδρόμησης κέντρου μάζας, που επιτρέπει την καταγραφή του σχήματος εκ των προτέρων και μια συνάρτηση απώλειας προσαρμοσμένη στην καρδιακή ανατομία.
Τα χαρακτηριστικά δημιουργούνται με μια αρχιτεκτονική μετατροπής πολλαπλών αναλυτικοτήτων conv-deconv `πλέγματος', η οποία αποτελεί επέκταση του u-net.
Αυτό το μοντέλο σε σύγκριση με το απλό conv-deconv και το u-net, εμφανίζει καλύτερα αποτελέσματα κατά μέσο όρο 5\% όσον αφορά τον δείκτη Dice.
Οι Patravali et al.~\cite{patravali20172d} εκπαίδευσαν ένα μοντέλο βασισμένο στο u-net, χρησιμοποιώντας το Dice σε συνδυασμό με το cross-entropy ως μέτρηση για την τμηματοποίηση των LV/RV και του μυοκαρδίου.
Το μοντέλο σχεδιάστηκε για να δέχεται μια στοίβα εικόνων ως κανάλια εισόδου, ενώ η έξοδος προβλέπει τη μεσαία εικόνα.
Με βάση τα πειράματα που διεξήγαγαν, βγήκε το συμπέρασμα ότι τρεις εικόνες ήταν βέλτιστες ως είσοδος για το μοντέλο αντί για μια ή πέντε.
Οι Isensee et al.~\cite{isensee2017automatic} χρησιμοποίησαν ένα σύνολο 2D και 3D u-net για κατάτμηση των LV/RV και του μυοκαρδίου του LV σε κάθε επανάληψη του καρδιακού κύκλου.
Πληροφορίες εξήχθησαν από την κατατμημένη χρονοσειρά με τη μορφή χαρακτηριστικών, που αντικατοπτρίζουν τις διαγνωστικές κλινικές διαδικασίες του σκοπού της ταξινόμησης.
Με βάση αυτά τα χαρακτηριστικά, εκπαίδευσαν ένα ensemble perceptrons πολλαπλών~\textit{επιπέδων} και έναν ταξινομητή RF για την πρόβλεψη της παθολογικής κατηγορίας.
Το μοντέλο τους κατέλαβε την πρώτη θέση στον διαγωνισμό του ACDC\@.

\begin{sidewaystable}
	\caption{Εφαρμογές βαθιάς μάθησης με χρήση MRI, για κατάτμηση LV/RV}
	\label{table:imaging2}
	\centering
	\begin{tabular}{l c l l}
		\toprule
		\thead{Αναφορά}                            & \thead{Μέθοδος} & \thead{Εφαρμογή/Σημειώσεις\footnote{Σε παρένθεση οι βάσεις δεδομένων που χρησιμοποιήθηκαν.}}               & \thead{Dice\footnote{το ($\S$) υποδηλώνει `για κάθε βάση', το ($*$) υποδηλώνει μέσο τετραγωνικό σφάλμα για EF το ($+$) υποδηλώνει `για ενδοκαρδιακά και επικαρδιακά', το ($-$) υποδηλώνει ακρίβεια, και το ($\#$) υποδηλώνει `για CT και MRI'}} \\
		\midrule
		Zotti 2017~\cite{zotti2017gridnet}          & u-net           & παραλλαγή του u-net με πολυ-κλιμακωτό conv-deconv αρχιτεκτονική πλέγματος (AC17)                           & 90\%                                                                                                                                                                                                                                            \\
		Patravali 2017~\cite{patravali20172d}       & u-net           & 2D/3D u-net (AC17)                                                                                         & \textit{πολλαπλά}                                                                                                                                                                                                                               \\
		Isensee 2017~\cite{isensee2017automatic}    & u-net           & ensemble u-net, με συστηματοποίηση πολυ-επιπέδων perceptron και ταξινομητή RF (AC17)                 & \textit{πολλαπλά}                                                                                                                                                                                                                               \\
		Tran 2016~\cite{tran2016fully}              & CNN             & FCN τεσσάρων \textit{επιπέδων} (SUN09, STA11)                                                              & 92\%, 96\%$^+$                                                                                                                                                                                                                                  \\
		Bai 2017~\cite{bai2017semi}                 & CNN             & VGGnet16 και DeepLab με χρήση CRF και πιο λεπτομερή αποτελέσματα (UKBDB)                                  & 90.3\%                                                                                                                                                                                                                                          \\
		Lieman 2017~\cite{lieman2017fastventricle}  & u-net           & επέκταση του ENet~\cite{paszke2016enet} με skip-connections (μη-δημόσια)                                    & \textit{πολλαπλά}                                                                                                                                                                                                                               \\
		Winther 2017~\cite{winther2017nu}           & u-net           & $\nu$-net παραλλαγή του u-net (DS16, SUN09, RV12, μη-δημόσια)                                              & \textit{πολλαπλά}                                                                                                                                                                                                                               \\
		Du 2018~\cite{du2018deep}                   & DBN             & χαρακτηριστικά DAISY και DBN παλινδρόμησης με χρήση 2900 εικόνων (μη-δημόσια)                              & 91.6\%, 94.1\%$^+$                                                                                                                                                                                                                              \\
		Giannakidis 2016~\cite{giannakidis2016fast} & CNN             & κατάτμηση RV με χρήση 3D πολυ-κλιμακωτών CNN με δύο διαδρομές (μη-δημόσια)                                 & 82.81\%                                                                                                                                                                                                                                         \\
		\bottomrule
	\end{tabular}
\end{sidewaystable}

Διάφορες άλλες βάσεις δεδομένων έχουν επίσης χρησιμοποιηθεί για την επίλυση της κατάτμησης LV/RV με CNNs.
Στο~\cite{bai2017semi} οι συγγραφείς δημιούργησαν μια μέθοδο ημι-επιβλεπώμενης μάθησης, στην οποία το δίκτυο κατάτμησης για το LV/RV και το μυοκάρδιο εκπαιδεύτηκε από τα επισημασμένα και μη-επισημασμένα δεδομένα.
Η αρχιτεκτονική του δικτύου βασίστηκε στο VGGnet16, παρόμοια με την αρχιτεκτονική του DeepLab~\cite{chen2018deeplab}, ενώ η τελική κατάτμηση βελτιώθηκε με τη χρήση ενός υπό όρους τυχαίου πεδίου (Conditional Random Field, CRF).
Οι συγγραφείς καταδεικνύουν ότι η εισαγωγή μη-επισημασμένων δεδομένων, βελτιώνει την απόδοση τμηματοποίησης όταν τα δεδομένα εκπαίδευσης είναι λίγα.
Στο~\cite{giannakidis2016fast} οι συγγραφείς υιοθετούν ένα 3D πολλαπλών κλιμάκων CNN για τον εντοπισμό των pixel που ανήκουν στον RV\@.
Το δίκτυο έχει δύο συνελικτικές διαδρομές με τις εισόδους του κεντροθετημένες στην ίδια θέση εικόνας, ενώ το δεύτερο τμήμα εξάγεται από μια εκδοχή της εικόνας που έχει υποβληθεί σε δειγματοληψία.
Τα αποτελέσματα που προέκυψαν ήταν καλύτερα από προηγούμενες μεθόδους, παρόλο που τα τελευταία βασίστηκαν σε χειροποίητα χαρακτηριστικά και εκπαιδεύτηκαν σε λιγότερο μεταβλητές βάσεις δεδομένων.

Τα FCNs έχουν επίσης χρησιμοποιηθεί για κατάτμηση LV/RV\@.
Στο άρθρο τους, οι Tran et al.~\cite{tran2016fully} εκπαίδευσαν ένα μοντέλο FCN τεσσάρων \textit{επιπέδων} για την κατάτμηση των LV/RV με τις SUN09, STA11.
Σύγκριναν προηγούμενες μεθόδους μαζί με δύο αρχικοποιήσεις του μοντέλου τους: μία fine-tuned έκδοση του μοντέλου τους χρησιμοποιώντας το STA11 και μια αρχικοποιημένη με Xavier, με το τελευταίο να έχει τις καλύτερες επιδόσεις σε σχεδόν όλα τα προβλήματα.

Τα FCN με skip-connections και u-net έχουν επίσης χρησιμοποιηθεί για την επίλυση αυτού του προβλήματος.
Οι Lieman et al.~\cite{lieman2017fastventricle} δημιούργησαν μια αρχιτεκτονική FCN με skip-connections με το όνομα FastVentricle βασισμένη στο ENet~\cite{paszke2016enet}, η οποία είναι ταχύτερη και λειτουργεί με λιγότερη μνήμη από τις προηγούμενες αρχιτεκτονικές κοιλιακής κατάτμησης επιτυγχάνοντας υψηλή κλινική ακρίβεια.
Στο~\cite{winther2017nu} οι συγγραφείς εισάγουν το $\nu$-net το οποίο είναι μια παραλλαγή u-net για την κατάτμηση του ενδοκαρδίου και του επικαρδίου LV/RV με τη χρήση των DS16, SUN09 και RV12.
Αυτή η μέθοδος απέδωσε καλύτερα από τον ειδικό καρδιολόγο σε αυτή τη μελέτη, ειδικά για την κατάτμηση του RV\@.

Ορισμένες άλλες μέθοδοι βασίστηκαν σε μοντέλα παλινδρόμησης.
Στο άρθρο τους οι Du et al.~\cite{du2018deep} δημιούργησαν ένα πλαίσιο κατάτμησης παλινδρόμησης για να οριοθετήσουν το LV/RV\@.
Πρώτον, εξάγονται χαρακτηριστικά DAISY και στη συνέχεια χρησιμοποιήθηκε μια μέθοδος αναπαράστασης βάσει σημείων, για την απεικόνιση των ορίων.
Τέλος, τα χαρακτηριστικά DAISY χρησιμοποιήθηκαν ως είσοδος και τα σημεία ορίων ως ετικέτες για την εκπαίδευση του μοντέλου παλινδρόμησης με βάση το DBN\@.
Η απόδοση του μοντέλου αξιολογείται χρησιμοποιώντας διαφορετικά χαρακτηριστικά από το DAISY (GIST, ιστόγραμμα πυραμίδων προσανατολισμένων διαβαθμίσεων) και επίσης συγκρίνεται με την παλινδρόμηση διανυσμάτων υποστήριξης (Support Vector Regression, SVR) και άλλες παραδοσιακές μεθόδους (γραφήματα, ενεργά περιγράμματα, level set), επιτυγχάνοντας καλύτερα αποτελέσματα.

\subsection{Κατάτμηση της καρδιάς}
Το MICCAI 2016 HVSMR (HVS16) χρησιμοποιήθηκε για την κατάτμηση της καρδιάς η οποία περιέχει εικόνες MRI από 20 ασθενείς.
Οι Wolterink et al.~\cite{wolterink2016dilated} εκπαίδευσαν ένα CNN δέκα \textit{επιπέδων} με αυξανόμενα επίπεδα διαστολής για την κατάτμηση του μυοκαρδίου και του αίματος στις αξονικές, σαγματοειδής και στεφανιαίες εικόνες.
Επίσης, χρησιμοποιούν βαθιά επίβλεψη~\cite{lee2015deeply} για να επιλύσουν το πρόβλημα του vanishing gradients και να βελτιώσουν την αποτελεσματικότητα της εκπαίδευσης του δικτύου τους χρησιμοποιώντας ένα μικρό σύνολο δεδομένων.
Τα πειράματα που πραγματοποίησαν με και χωρίς διαστολές (dilations) σε αυτήν την αρχιτεκτονική έδειξαν τη χρησιμότητα της.
Στο άρθρο τους οι Li et al.~\cite{li2016automatic} ξεκίνησαν με ένα 3D FCN με επισήμανση των voxel και στη συνέχεια εισήγαγαν διαστελλόμενα συνελικτικά επίπεδα στο βασικό μοντέλο για να επεκτείνουν το δεκτικό πεδίο.
Έπειτα χρησιμοποιούν μονοπάτια βαθιά επίβλεψης, για την επιταχύνουν την εκπαίδευση και την αξιοποίηση πληροφοριών πολλαπλών κλιμάκων.
Σύμφωνα με τους συγγραφείς το μοντέλο παρουσιάζει καλή ακρίβεια κατάτμησης, σε συνδυασμό με χαμηλό υπολογιστικό κόστος.
Οι Yu et al.~\cite{yu20163d} δημιούργησαν ένα φράκταλ δίκτυο 3D FCN για κατάτμηση της καρδιάς και των μεγάλων αγγείων.
Εφαρμόζοντας αναδρομικά έναν κανόνα απλής επέκτασης, κατασκευάζουν το φράκταλ δίκτυο συνδυάζοντας ιεραρχικές ενδείξεις για ακριβή κατάτμηση.
Επιτυγχάνουν επίσης καλά αποτελέσματα με χαμηλό υπολογιστικό κόστος (12 δευτερόλεπτα ανά όγκο).

Μια άλλη βάση δεδομένων που χρησιμοποιήθηκε για την κατάτμηση της καρδιάς ήταν η MM17 η οποία περιέχει 120 πολυτροπικές εικόνες από καρδιακή MRI/CT\@.
Η μέθοδος των Payer et al.~\cite{payer2017multi} βασίζεται σε δύο FCN για τον εντοπισμό και την κατάτμηση της καρδιάς.
Αρχικά, το CNN εντοπισμού βρίσκει το κέντρο του πλαισίου οριοθέτησης γύρω από όλες τις δομές της καρδιάς, έτσι ώστε το CNN κατάτμησης να μπορεί να επικεντρωθεί σε αυτήν την περιοχή.
Έπειτα, το CNN κατάτμησης μετατρέπει τις προβλέψεις ενδιάμεσων ετικετών σε θέσεις άλλων ετικετών.
Επομένως το δίκτυο μαθαίνει από τις σχετικές θέσεις μεταξύ των επισημάνσεων και επικεντρώνεται στις ανατομικά εφικτές διαμορφώσεις.
Το μοντέλο συγκρίθηκε με το u-net επιτυγχάνοντας καλύτερα αποτελέσματα, ειδικά στο σύνολο δεδομένων του MRI\@.
Οι Mortazi et al.~\cite{mortazi2017multi} εκπαίδευσαν ένα πολυεπίπεδο CNN με μια προσαρμοστική στρατηγική σύντηξης, για την κατάτμηση επτά περιοχών της καρδιάς.
Σχεδίασαν τρία CNN (ένα για κάθε κάθετο επίπεδο) με την ίδια αρχιτεκτονική και τα εκπαίδευσαν για επισήμανση των voxel.
Από τα πειράματά τους καταλήγουν στο συμπέρασμα ότι το μοντέλο τους οριοθετεί τις καρδιακές δομές με υψηλή ακρίβεια και αποτελεσματικά.
Στο~\cite{yang2017hybrid} οι συγγραφείς χρησιμοποίησαν ένα FCN, το οποίο συνδύασαν με 3D τελεστές, μεταφορά μάθησης και έναν μηχανισμό βαθιάς επίβλεψης για την απόσταξη 3D συμφραζομένων πληροφοριών και την επίλυση πιθανών δυσκολιών στην εκπαίδευση.
Χρησιμοποιήθηκε υβριδική απώλεια που καθοδηγεί τη διαδικασία εκπαίδευσης για την εξισορρόπηση των κατηγοριών και διατηρεί τις λεπτομέρειες των ορίων.
Σύμφωνα με τα πειράματά τους, η χρήση της υβριδικής απώλειας επιτυγχάνει καλύτερα αποτελέσματα από το Dice μέτρο.

\subsection{Άλλες εφαρμογές}
Μέθοδοι βαθιάς μάθησης έχουν επίσης χρησιμοποιηθεί και για ανίχνευση άλλων καρδιακών δομών με MRI\@.
Οι Yang et al.~\cite{yang2017segmenting} δημιούργησαν μια μέθοδο διάδοσης πολλαπλών ατλάντων, για να ενσωματώσουν την ανατομική δομή του μυοκαρδίου του αριστερού κόλπου και των πνευμονικών φλεβών.
Αυτό ακολουθήθηκε από ένα μη-επιτηρούμενο εκπαιδευμένο SSAE με ένα softmax για την κατάτμηση της κολπικής ίνωσης, χρησιμοποιώντας 20 εικόνες από ασθενείς με AF\@.
Στο άρθρο τους οι Zhang et al.~\cite{zhang2016automated} προσπάθησαν να ανιχνεύσουν εικόνες έλλειψης apical και basal.
Ελέγχουν την παρουσία τυπικών basal και apical μοτίβων στις τελευταίες και πρώτες εικόνες της βάσης δεδομένων και εκπαιδεύουν δύο CNN για να κατασκευάσουν ένα σύνολο διακριτικών χαρακτηριστικών.
Τα πειράματά τους έδειξαν ότι το μοντέλο με τέσσερα \textit{επίπεδα}, έχει καλύτερη απόδοση από τις SAE και τις Βαθιές Μηχανές Boltzmann.

\begin{sidewaystable}
	\caption{Εφαρμογές βαθιάς μάθησης με χρήση MRI, για κατάτμηση της καρδιάς και άλλα}
	\label{table:imaging3}
	\centering
	\begin{tabular}{l c l l}
		\toprule
		\thead{Αναφορά}                           & \thead{Μέθοδος} & \thead{Εφαρμογή/Σημειώσεις\footnote{Σε παρένθεση οι βάσεις δεδομένων που χρησιμοποιήθηκαν.}}    & \thead{Dice\footnote{το ($\S$) υποδηλώνει `για κάθε βάση', το ($*$) υποδηλώνει μέσο τετραγωνικό σφάλμα για EF το ($+$) υποδηλώνει `για ενδοκαρδιακά και επικαρδιακά', το ($-$) υποδηλώνει ακρίβεια, και το ($\#$) υποδηλώνει `για CT και MRI'}} \\
		\midrule
		\multicolumn{4}{l}{\thead{Τμηματοποίηση όλης της καρδιάς}}                                                                                                                                                                                                                                                                                                                                                      \\
		\midrule
		Wolterink 2016~\cite{wolterink2016dilated} & CNN             & διαστελλόμενα CNN με ορθογώνια patches (HVS16)                                                  & 80\%, 93\%                                                                                                                                                                                                                                      \\
		Li 2016~\cite{li2016automatic}             & CNN             & βαθιά επιβλεπώμενο 3D FCN με διαστολές (HVS16)                                                  & 69.5\%                                                                                                                                                                                                                                          \\
		Yu 2017~\cite{yu20163d}                    & CNN             & βαθιά επιβλεπώμενο 3D FCN κατασκευασμένο με αυτο-επιβλεπώμενο φρακταλ τρόπο (HVS16)             & \textit{πολλαπλά}                                                                                                                                                                                                                               \\
		Payer~\cite{payer2017multi}                & CNN             & δύο ξεχωριστά FCN για εντοπισμό και κατάτμηση (MM17)                                            & 90.7\%, 87\%$^\#$                                                                                                                                                                                                                               \\
		Mortazi 2017~\cite{mortazi2017multi}       & CNN             & πολυ-επίπεδο FCN (MM17)                                                                         & 90\%, 85\%$^\#$                                                                                                                                                                                                                                 \\
		Yang~\cite{yang2017hybrid}                 & CNN             & βαθιά επιβλεπώμενο 3D FCN εκπαιδευμένο με μεταφορά μάθησης (MM17)                               & 84.3\%, 77.8\%$^\#$                                                                                                                                                                                                                             \\
		\midrule
		\multicolumn{4}{l}{\thead{Άλλες εφαρμογές}}                                                                                                                                                                                                                                                                                                                                                                     \\
		\midrule
		Yang 2017~\cite{yang2017segmenting}        & SSAE            & κατάτμηση κολπικής ίνωσης με διάδοση πολλαπλών ατλάντων, SSAE, softmax (μη-δημόσια)      & 82\%                                                                                                                                                                                                                                            \\
		Zhang 2016~\cite{zhang2016automated}       & CNN             & ανίχνευση έλλειψης apical και basal με δύο CNNs τεσσάρων \textit{επιπέδων} (UKBDB)              & \textit{πολλαπλά}                                                                                                                                                                                                                               \\
		Kong 2016~\cite{kong2016recognizing}       & CNN, RNN        & CNN για χωρική πληροφορία και RNN για χρονική πληροφορία                                        & \textit{πολλαπλά}                                                                                                                                                                                                                               \\
		Yang 2017~\cite{yang2017convolutional}     & CNN             & CNN για την ανίχνευση του τέλους της διαστολής και συστολής απο το LV (STA11, μη-δημόσια)       & 76.5\%$^-$                                                                                                                                                                                                                                      \\
		Xu 2017~\cite{xu2017direct}                & Multiple        & ανίχνευση MI με χρήση Fast R-CNN για εντοπισμό της καρδιάς, LSTM και SAE (μη-δημόσια)           & 94.3\%$^-$                                                                                                                                                                                                                                      \\
		Xue 2018~\cite{xue2018full}                & CNN, LSTM       & CNN, δύο παράλληλα LSTMs και Bayesian μοντέλο για ποσοτικοποίηση του LV (μη-δημόσια) & \textit{πολλαπλά}                                                                                                                                                                                                                               \\
		Zhen 2016~\cite{zhen2016multi}             & RBM             & πολυ-κλιμακωτό συνελικτικό RBM και RF για δι-κολπική εκτίμηση του όγκου (μη-δημόσια)            & 3.87\%$^*$                                                                                                                                                                                                                                      \\
		Biffi 2016~\cite{biffi2018learning}        & CNN             & ανίχνευση υπερτροφικής καρδιομυοπάθειας με χρήση μεταβολικών AE (AC17, μη-δημόσια)              & 90\%$^-$                                                                                                                                                                                                                                        \\
		Oktay 2016~\cite{oktay2016multi}           & CNN             & αύξηση αναλυτικότητας εικόνας με residual CNN (μη-δημόσια)                                      & \textit{πολλαπλά}                                                                                                                                                                                                                               \\
		\bottomrule
	\end{tabular}
\end{sidewaystable}

Άλλα ιατρικά προβλήματα με MRI μελετήθηκαν επίσης, όπως η ανίχνευση εικόνων ακραίας συστολής και διαστολής.
Οι Kong et al.~\cite{kong2016recognizing} δημιούργησαν ένα χρονικό δίκτυο παλινδρόμησης που είχε προεκπαιδευτεί στο ImageNet με την ενσωμάτωση ενός CNN με ένα RNN, για να προσδιορίσει τις εικόνες της τελικής διαστολής και της τελικής συστολής από τις ακολουθίες MRI\@.
Το CNN κωδικοποιεί τις χωρικές πληροφορίες μίας καρδιακής αλληλουχίας ενώ το RNN αποκωδικοποιεί τις χρονικές.
Επίσης σχεδίασαν μια συνάρτηση απώλειας για να περιορίσουν τη δομή των προβλεπόμενων επισημάνσεων.
Το μοντέλο τους επιτυγχάνει καλύτερη μέση διαφορά εικόνων από τις προηγούμενες μεθόδους.
Στο άρθρο τους οι Yang et al.~\cite{yang2017convolutional} χρησιμοποίησαν ένα CNN για να ανιχνεύσουν τις εικόνες της τελικής διαστολής και της τελικής συστολής από την LV, επιτυγχάνοντας ακρίβεια 76.5\%.

Δημιουργήθηκαν επίσης μέθοδοι ποσοτικοποίησης διαφόρων καρδιαγγειακών χαρακτηριστικών.
Στο~\cite{xu2017direct} οι συγγραφείς εντοπίζουν τη θέση και το σχήμα του ΜΙ χρησιμοποιώντας ένα μοντέλο που αποτελείται από τρία επίπεδα; πρώτον, το επίπεδο εντοπισμού καρδιάς είναι ένα Fast R-CNN το οποίο απομονώνει τις ακολουθίες ROI συμπεριλαμβανομένου του LV\@; δεύτερον, τα στατιστικά επίπεδα κίνησης, τα οποία κατασκευάζουν μια αρχιτεκτονική χρονοσειράς για να καταγράψουν τα χαρακτηριστικά τοπικής κίνησης που παράγονται από το LSTM-RNN και τα χαρακτηριστικά κίνησης που παράγονται από βαθιές οπτικές ροές από την ακολουθία ROI\@; τρίτον, τα FNN διάκρισης, τα οποία χρησιμοποιούν το SAE για να μάθουν περαιτέρω τα χαρακτηριστικά από το προηγούμενο επίπεδο και τέλος έναν ταξινομητή softmax\@.
Οι Xue et al.~\cite{xue2018full} εκπαίδευσαν ένα δίκτυο βαθιάς μάθησης πολλαπλών προβλημάτων, σε MRI από 145 άτομα με 20 εικόνες ο καθένας για πλήρη ποσοτικοποίηση του LV\@.
Αποτελείται από ένα CNN τριών \textit{επιπέδων} που εξάγει τις καρδιακές αναπαραστάσεις, και στη συνέχεια δύο παράλληλα LSTM-RNN για τη μοντελοποίηση της χρονικής δυναμικής των καρδιακών ακολουθιών.
Τέλος, τοποθετείται ένα Bayesian πλαίσιο ικανό να μάθει τις σχέσεις μεταξύ των προβλημάτων και ένας ταξινομητής softmax.
Εκτεταμένες συγκρίσεις με προηγούμενες μεθόδους δείχνουν την αποτελεσματικότητα αυτής της μεθόδου όσον αφορά το μέσο απόλυτο σφάλμα.
Στο~\cite{zhen2016multi} οι συγγραφείς δημιούργησαν μια μέθοδο μη-επιβλεπώμενης μάθησης καρδιακής απεικόνισης χρησιμοποιώντας πολυεπίπεδο συνελικτικό RBM και μια άμεση εκτίμηση όγκου των δύο κοιλοτήτων χρησιμοποιώντας RF\@.
Σύγκριναν το μοντέλο τους με ένα Bayesian μοντέλο, ένα μοντέλο βασισμένο σε χειροποίητα χαρακτηριστικά, τα level-set και την περικοπή γραφήματος, επιτυγχάνοντας καλύτερα αποτελέσματα από πλευράς συντελεστή συσχέτισης για όγκους LV/RV και εκτίμηση σφάλματος του EF\@.

Άλλες μέθοδοι χρησιμοποιήθηκαν επίσης για την ανίχνευση υπερτροφικής καρδιομυοκαρδιοπάθειας ή για την αύξηση της αναλυτικότητας των MRI\@.
Οι Biffi et al.~\cite{biffi2018learning} εκπαίδευσαν ένα VΑΕ για την ταυτοποίηση ασθενών με υπερτροφική μυοκαρδιοπάθεια χρησιμοποιώντας ένα ισορροπημένο σύνολο δεδομένων 1365 ασθενών και την AC17.
Δείχνουν επίσης ότι το δίκτυο είναι σε θέση να απεικονίσει και να ποσοτικοποιήσει τα πρότυπα αναδιαμόρφωσης που είναι σχετικά με την παθολογία στον αρχικό χώρο εισόδου των εικόνων, αυξάνοντας έτσι την ερμηνευσιμότητα του μοντέλου.
Στο~\cite{oktay2016multi} οι συγγραφείς δημιούργησαν μια μέθοδο επαύξησης της αναλυτικότητας της εικόνας βασισμένη σε residual CNN η οποία επιτρέπει τη χρήση δεδομένων εισόδου που αποκτήθηκαν από διαφορετικά επίπεδα προβολής για βελτιωμένη απόδοση.
Σύγκριναν με άλλες μεθόδους παρεμβολής (γραμμική, spline, ταίριασμα patch πολλαπλών ατλάντων, ρηχό CNN, CNN), επιτυγχάνοντας καλύτερα αποτελέσματα όσον αφορά το PSNR\@.
Συγγραφείς από την ίδια ομάδα πρότειναν μια στρατηγική εκπαίδευσης~\cite{oktay2018anatomically} που ενσωματώνει ανατομική προηγούμενη γνώση σε CNNs μέσω ενός μοντέλου συστηματοποίησης, ενθαρρύνοντας την να ακολουθήσει την ανατομία μέσω μη-γραμμικών παραστάσεων του σχήματος.

\subsection{Συμπεράσματα της χρήσης βαθιάς μάθησης με MRI}
Υπάρχει ένα ευρύ φάσμα αρχιτεκτονικών που έχουν εφαρμοστεί στα MRI\@.
Οι περισσότερες είναι CNNs ή u-net οι οποίες είτε χρησιμοποιούνται αποκλειστικά είτε σε συνδυασμό με RNNs, AEs ή ensemble.
Το πρόβλημα είναι ότι οι περισσότερες από αυτές δεν εκπαιδεύονται από-άκρο-σε-άκρο. Βασίζονται σε προεπεξεργασία, χρήση χειροποίητων χαρακτηριστικών, ενεργά περιγράμματα, level-set και σε άλλες μη-διαφοροποιήσιμες μεθόδους, χάνοντας έτσι μερικώς τη δυνατότητα κλιμάκωσης στην παρουσία νέων δεδομένων.
Κύριος στόχος αυτού του τομέα πρέπει να είναι η δημιουργία μοντέλων από-άκρο-σε-άκρο, ακόμη και αν αυτό σημαίνει μικρότερη ακρίβεια βραχυπρόθεσμα; πιο αποδοτικές αρχιτεκτονικές θα μπορούσαν να καλύψουν το χάσμα στο μέλλον.

Ένα ενδιαφέρον εύρημα σχετικά με την κατάτμηση της καρδιάς έγινε στο~\cite{konukoglu2018exploration} όπου οι συγγραφείς διερεύνησαν την καταλληλότητα των προηγούμενων 2D, 3D CNN αρχιτεκτονικών και των τροποποιήσεών τους.
Διαπίστωσαν ότι η επεξεργασία ανά εικόνα χρησιμοποιώντας δίκτυα 2D ήταν καλύτερη λόγω του μεγάλου πάχους του τμήματος.
Ωστόσο, η επιλογή της αρχιτεκτονικής δικτύου διαδραματίζει μικρό ρόλο.

\section{Fundus}
Η απεικόνιση Fundus είναι ένα κλινικό εργαλείο για την αξιολόγηση της αμφιβληστροειδοπάθειας σε ασθενείς στην οποία η ένταση αντιπροσωπεύει την ποσότητα του ανακλώμενου φωτός συγκεκριμένης ζώνης κυμάτων~\cite{abramoff2010retinal}.
Μία από τις πιο ευρέως διαδεδομένες βάσεις δεδομένων στο Fundus είναι η DRIVE, η οποία περιέχει 40 εικόνες και τις αντίστοιχες επισημάνσεις της μάσκας των αγγείων.

\subsection{Κατάτμηση αγγείων}
Τα CNN χρησιμοποιήθηκαν για την κατάτμηση αγγείων σε εικόνες Fundus.
Στο~\cite{wang2015hierarchical} οι συγγραφείς αρχικά χρησιμοποίησαν ισορροπία ιστογράμματος και φιλτράρισμα Gauss για τη μείωση του θορύβου.
Στη συνέχεια χρησιμοποιήθηκε ένα CNN τριών \textit{επιπέδων} ως εξαγωγέας χαρακτηριστικών και ένα RF ως ταξινομητής.
Σύμφωνα με τα πειράματά τους, η καλύτερη απόδοση επιτεύχθηκε από ένα ensemble νικητής-τα-παίρνει-όλα, σε σύγκριση με ένα μέσο, σταθμισμένο και διάμεσο ensemble.
Οι Zhou et al.~\cite{zhou2017improving} εφάρμοσαν προεπεξεργασία εικόνας για την εξάλειψη των ισχυρών άκρων γύρω από το οπτικό πεδίο και κανονικοποίησαν την φωτεινότητα και την αντίθεση μέσα σε αυτό.
Στη συνέχεια, εκπαίδευσαν ένα CNN για να παράξουν χαρακτηριστικά για γραμμικά μοντέλα και εφάρμοσαν φίλτρα για την ενίσχυση των λεπτών αγγείων, μειώνοντας τη διαφορά έντασης μεταξύ λεπτών και ευρέων αγγείων.
Στη συνέχεια ένας πυκνός CRF προσαρμόστηκε για να επιτευχθεί η τελική κατάτμηση του αγγείου, λαμβάνοντας τα διακριτικά χαρακτηριστικά για μοναδιαία δυναμικά και την εικόνα με τα ενισχυμένα λεπτά αγγεία.
Μεταξύ των αποτελεσμάτων τους, στα οποία παρουσιάζουν μεγαλύτερη ακρίβεια από τις περισσότερες μεθόδους τελευταίας τεχνολογίας, παρέχουν επίσης στοιχεία υπέρ της χρήσης των πληροφοριών RGB του Fundus αντί για χρήση μόνο του πράσινου καναλιού.
Οι Chen et al.~\cite{chen2017labeling} σχεδίασαν ένα σύνολο κανόνων για τη δημιουργία τεχνητών δειγμάτων εκπαίδευσης με πρότερη γνώση και χωρίς χειροκίνητη επισήμανση.
Εκπαίδευσαν ένα FCN με ένα skip-connection που επιτρέπει την υψηλού επιπέδου πληροφορία να καθοδηγεί την εργασία σε χαμηλότερα επίπεδα.
Αξιολογούν το μοντέλο τους στις DRIVE και STARE, επιτυγχάνοντας συγκρίσιμα αποτελέσματα με άλλες μεθόδους που χρησιμοποιούν πραγματική επισήμανση.
Στο~\cite{maji2016ensemble} οι συγγραφείς εκπαίδευσαν ένα ensemble 12 CNNs με τρία \textit{επίπεδα} το καθένα στη DRIVE, όπου κατά τη διάρκεια του συμπερασμού οι απαντήσεις των CNNs υπολογίζονται κατά μέσο όρο για να σχηματίσουν την τελική κατάτμηση.
Δείχνουν ότι το μοντέλο τους επιτυγχάνει υψηλότερη μέση ακρίβεια από τις προηγούμενες μεθόδους.
Οι Fu et al.~\cite{fu2016retinal} εκπαιδεύουν ένα CNN στις DRIVE και STARE δημιουργώντας ένα χάρτη πιθανοτήτων και έπειτα χρησιμοποιούν έναν πλήρως συνδεδεμένο CRF για να συνδυάσουν τους χάρτες πιθανότητας και τις αλληλεπιδράσεις μεγάλης εμβέλειας μεταξύ των εικονοστοιχείων.
Στο~\cite{wu2016deep} οι συγγραφείς χρησιμοποίησαν ένα CNN για να μάθουν τα χαρακτηριστικά και μια αναζήτηση πλησιέστερων γειτόνων βασισμένη στο PCA που χρησιμοποιήθηκε για την εκτίμηση της τοπικής κατανομής δομών.
Εκτός από την παρουσίαση καλών αποτελεσμάτων υποστηρίζουν ότι είναι σημαντικό για το CNN να ενσωματώσει πληροφορίες σχετικά με τη δομή δέντρων όσον αφορά την ακρίβεια.

\begin{sidewaystable}
	\caption{Εφαρμογές βαθιάς μάθησης με χρήση Fundus, για κατάτμηση αγγείων}
	\label{table:imaging4}
	\centering
	\begin{tabular}{l c l l}
		\toprule
		\thead{Αναφορά}                                  & \thead{Μέθοδος} & \thead{Εφαρμογή/Σημειώσεις\footnote{Σε παρένθεση οι βάσεις δεδομένων που χρησιμοποιήθηκαν.}}                  & \thead{AUC\footnote{Το ($*$) υποδηλώνει ακρίβεια.}} \\
		\midrule
		Wang 2015~\cite{wang2015hierarchical}             & CNN, RF         & CNN τριών \textit{επιπέδων} συνδυασμένα με ensemble RF (DRIVE, STARE)                                         & 0.9475                                              \\
		Zhou 2017~\cite{zhou2017improving}                & CNN, CRF        & CNN για εξαγωγή χαρακτηριστικών και CRF για τελικό αποτέλεσμα (DRIVE, STARE, CHDB)                     & 0.7942                                              \\
		Chen 2017~\cite{chen2017labeling}                 & CNN             & τεχνητά δεδομένα, FCN (DRIVE, STARE)                                                                          & 0.9516                                              \\
		Maji 2016~\cite{maji2016ensemble}                 & CNN             & 12 CNNs ensemble με τρία \textit{επίπεδα} (DRIVE)                                                             & 0.9283                                              \\
		Fu 2016~\cite{fu2016retinal}                      & CNN, CRF        & CNN και CRF (DRIVE, STARE)                                                                                    & 94.70\%$^*$                                         \\
		Wu 2016~\cite{wu2016deep}                         & CNN             & τμηματοποίηση αγγείου και ανίχνευση βρόχου με CNN και PCA (DRIVE)                                             & 0.9701                                              \\
		Li 2016~\cite{li2016cross}                        & SDAE            & FNN και SDAE (DRIVE, STARE, CHDB)                                                                             & 0.9738                                              \\
		Lahiri 2016~\cite{lahiri2016deep}                 & SDAE            & ensemble δύο επιπέδων SDAE (DRIVE)                                                        & 95.30\%$^*$                                         \\
		Oliveira 2017~\cite{oliveira2017augmenting}       & u-net           & επαύξηση δεδομένων και u-net (DRIVE)                                                                          & 0.9768                                              \\
		Leopold 2017~\cite{leopold2017use}                & CNN             & CNN ως ένα πολυ-καναλικός ταξινομητής και φίλτρα Gabor (DRIVE)                                                & 94.78\%$^*$                                         \\
		Leopold 2017~\cite{leopold2017pixelbnn}           & AE              & πλήρως residual AE με περιφραγμένο ρεύμα βασισμένο στο u-net (DRIVE, STARE, CHDB)                             & 0.8268                                              \\
		Mo 2017~\cite{mo2017multi}                        & CNN             & επαυξημένοι ταξινομητές και μεταφορά μάθησης (DRIVE, STARE, CHDB)                                             & 0.9782                                              \\
		Melinscak 2015~\cite{melinvsvcak2015retinal}      & CNN             & CNN τεσσάρων \textit{επιπέδων} (DRIVE)                                                                        & 0.9749                                              \\
		Sengur 2017~\cite{sengur2017retinal}              & CNN             & CNN δύο \textit{επιπέδων} με dropout (DRIVE)                                                                  & 0.9674                                              \\
		Meyer 2017~\cite{meyer2017deep}                   & u-net           & τμηματοποίηση αγγείου με χρήση u-net στην SLO (IOSTAR, RC-SLO)                                                & 0.9771                                              \\
		\bottomrule
	\end{tabular}
\end{sidewaystable}

Τα ΑΕ έχουν επίσης χρησιμοποιηθεί για την κατάτμηση των αγγείων.
Οι Li et al.~\cite{li2016cross} εκπαίδευσαν ένα FNN και ένα AE αποθορυβοποίησης με τις DRIVE, STARE και CHDB\@.
Υποστηρίζουν ότι τα χαρακτηριστικά του μοντέλου τους είναι πιο ανθεκτικά στο θόρυβο και τις διαφορετικές συνθήκες απεικόνισης, επειδή η διαδικασία μάθησης εκμεταλλεύεται τα χαρακτηριστικά των αγγείων σε όλες τις εικόνες εκπαίδευσης.
Στο~\cite{lahiri2016deep} οι συγγραφείς χρησιμοποίησαν μη-επιβλεπώμενα ιεραρχικά χαρακτηριστικά χρησιμοποιώντας ένα ensemble δύο \textit{επιπέδων} SDAE\@.
Το επίπεδο εκπαίδευσης εξασφαλίζει αποσύνδεση και το επίπεδο του ensemble εξασφαλίζει την αρχιτεκτονική αναθεώρηση.
Δείχνουν επίσης ότι η εκπαίδευση του ensemble των ΑΕ ενισχύει την ποικιλομορφία στο λεξικό των χαρακτηριστικών που έχουν μαθευτεί για την κατάτμηση των αγγείων.
Ο ταξινομητής Softmax χρησιμοποιήθηκε στη συνέχεια για το fine-tuning κάθε ΑΕ και διερευνήθηκαν στρατηγικές για συγχώνευση δύο επιπέδων μελών του ensemble.

Άλλες αρχιτεκτονικές χρησιμοποιήθηκαν επίσης για την κατάτμηση των αγγείων.
Στο άρθρο τους οι Oliveira et al.~\cite{oliveira2017augmenting} εκπαίδευσαν ένα u-net με την DRIVE, παρουσιάζοντας καλά αποτελέσματα και ενδείξεις των πλεονεκτημάτων της επαύξησης των δεδομένων εκπαίδευσης με χρήση ελαστικών μετασχηματισμών.
Οι Leopold et al.~\cite{leopold2017use} διερεύνησαν τη χρήση ενός CNN ως ταξινομητή πολλαπλών καναλιών και τη χρήση φίλτρων Gabor για να ενισχύσουν την ακρίβεια της μεθόδου που περιγράφεται στο~\cite{leopold2017segmentation}.
Εφάρμοσαν το μέσο μιας σειράς φίλτρων Gabor με διάφορες συχνότητες και τιμές σίγμα στην έξοδο του δικτύου για να καθορίσουν εάν ένα εικονοστοιχείο αντιπροσωπεύει ένα αγγείο ή όχι.
Εκτός από τη διαπίστωση ότι τα βέλτιστα φίλτρα διαφέρουν μεταξύ των καναλιών, οι συγγραφείς δηλώνουν επίσης την `ανάγκη' να επιβάλλουν στα δίκτυα να ευθυγραμμίζονται με την ανθρώπινη αντίληψη, στο πλαίσιο της χειρωνακτικής επισήμανσης, ακόμη και αν απαιτεί πληροφορίες υποδειγματοληψίας, οι οποίες διαφορετικά θα μείωναν το υπολογιστικό κόστος.
Οι ίδιοι συγγραφείς~\cite{leopold2017pixelbnn} δημιούργησαν το PixelBNN, το οποίο είναι ένα πλήρως residual AE\@.
Είναι πάνω από οκτώ φορές χρονικά αποδοτικό από τις προηγούμενες μεθόδους κατά την διάρκεια του συμπερασμού με καλά αποτελέσματα, λαμβάνοντας υπόψη τη σημαντική μείωση των πληροφοριών από την αλλαγή μεγέθους των εικόνων κατά την προεπεξεργασία.
Στο άρθρο τους οι Mo et al.~\cite{mo2017multi} χρησιμοποίησαν βαθιά επίβλεψη με βοηθητικούς ταξινομητές στα ενδιάμεσα επίπεδα του δικτύου, για να βελτιώσουν τη διακριτική ικανότητα των χαρακτηριστικών στα χαμηλότερα επίπεδα του δικτύου και να καθοδηγήσουν το backpropagation να ξεπεράσει τα vanishing gradients.
Επιπλέον, μεταφορά μάθησης χρησιμοποιήθηκε για να ξεπεραστεί το ζήτημα των ανεπαρκών δεδομένων εκπαίδευσης.

\subsection{Ανίχνευση μικροανευρυσμάτων και αιμορραγίας}
Οι Haloi et al.~\cite{haloi2015improved} εκπαίδευσαν ένα CNN τριών \textit{επιπέδων} με dropout και maxout για ανίχνευση ΜΑ.
Τα πειράματα στις ROC και DIA έδειξαν θετικά αποτελέσματα.
Στο~\cite{giancardo2017representation} οι συγγραφείς δημιούργησαν ένα μοντέλο που μαθαίνει έναν γενικό περιγραφέα της μορφολογίας των αγγείων χρησιμοποιώντας την εσωτερική αναπαράσταση ενός variational u-net.
Έπειτα, εξέτασαν τις αγγειακές ενσωματώσεις (embeddings) σε ένα παρόμοιο πρόβλημα ανάκτησης εικόνων σύμφωνα με το αγγειακό σύστημα και σε ένα πρόβλημα ταξινόμησης της διαβητικής αμφιβληστροειδοπάθειας, στο οποίο δείχνουν πως τα embeddings των αγγείων μπορούν να βελτιώσουν την ταξινόμηση μιας μεθόδου η οποία βασίζεται στην ανίχνευση ΜΑ.
Στο~\cite{orlando2018ensemble} οι συγγραφείς συνδυάζουν ενισχυμένα χαρακτηριστικά από ένα CNN με χειροποίητα χαρακτηριστικά.
Αυτό το ensemble διάνυσμα χαρακτηριστικών χρησιμοποιήθηκε στη συνέχεια για την αναγνώριση των υποψηφίων αιμορραγίας και των ΜΑ, χρησιμοποιώντας έναν ταξινομητή RF\@.
Η ανάλυσή τους με χρήση t-SNE καταδεικνύει ότι τα χαρακτηριστικά από το CNN έχουν μεγάλη λεπτομέρεια όπως ένδειξη του προσανατολισμού της βλάβης ενώ τα χειροποίητα χαρακτηριστικά είναι σε θέση να διακρίνουν βλάβες χαμηλής αντίθεσης όπως αιμορραγίες.
Στο~\cite{van2016fast} οι συγγραφείς εκπαίδευσαν ένα CNN πέντε \textit{επιπέδων}, για ανίχνευση αιμορραγίας χρησιμοποιώντας 6679 εικόνες από τις βάσεις δεδομένων DS16 και Messidor.
Εφάρμοσαν επιλεκτική δειγματοληψία σε ένα CNN, κάτι το οποίο αύξησε την ταχύτητα της εκπαίδευσης με δυναμική επιλογή λανθασμένα ταξινομημένων δειγμάτων κατά τη διάρκεια της εκπαίδευσης.
Τα βάρη αποδίδονται στα δείγματα εκπαίδευσης και τα ενημερωμένα δείγματα περιλαμβάνονται στην επόμενη επανάληψη εκπαίδευσης.

\subsection{Άλλες εφαρμογές}
Το Fundus έχει χρησιμοποιηθεί και για ταξινόμηση αρτηριών/φλεβών.
Στο άρθρο τους οι Girard et al.~\cite{girard2017artery} εκπαίδευσαν ένα CNN τεσσάρων \textit{επιπέδων}, που ταξινομεί τα εικονοστοιχεία των αγγείων σε αρτηρίες/φλέβες χρησιμοποιώντας επαύξηση των δεδομένων εκπαίδευσης με περιστροφικό μετασχηματισμό.
Στη συνέχεια κατασκευάστηκε ένα γράφημα από το αγγειακό δίκτυο του αμφιβληστροειδούς, όπου οι κόμβοι ορίζονται ως οι κλάδοι των αγγείων και κάθε άκρη συνδέεται με ένα κόστος που εκτιμά εάν οι δύο κλάδοι θα πρέπει να έχουν την ίδια ετικέτα.
Η ταξινόμηση του CNN διαδόθηκε μέσω του ελάχιστου δένδρου του γραφήματος.
Τα πειράματα κατέδειξαν την αποτελεσματικότητα της μεθόδου, ιδίως στην παρουσία εμφραγμάτων.
Οι Welikala et al.~\cite{welikala2017automated} εκπαίδευσαν και αξιολόγησαν ένα CNN τριών \textit{επιπέδων}, χρησιμοποιώντας εικονοστοιχεία της κεντρικής γραμμής που προέρχονται από εικόνες αμφιβληστροειδούς.
Από τα πειράματά τους διαπίστωσαν ότι η επαύξηση των δεδομένων εκπαίδευσης με χρήση περιστροφικών και κλιμακωτών μετασχηματισμών, δεν βοήθησε στην αύξηση της ακρίβειας αποδίδοντάς το στην παρεμβολή μεταξύ των εντάσεων των εικονοστοιχείων, η οποία είναι προβληματική λόγω της ευαισθησίας του CNN στην κατανομή των εικονοστοιχείων.

\begin{sidewaystable}
	\caption{Εφαρμογές βαθιάς μάθησης με χρήση Fundus, εκτός της κατάτμησης αγγείων}
	\label{table:imaging5}
	\centering
	\begin{tabular}{l c l l}
		\toprule
		\thead{Αναφορά}                                  & \thead{Μέθοδος} & \thead{Εφαρμογή/Σημειώσεις\footnote{Σε παρένθεση οι βάσεις δεδομένων που χρησιμοποιήθηκαν.}}                  & \thead{AUC\footnote{Το ($*$) υποδηλώνει ακρίβεια.}} \\
		\midrule
		\multicolumn{4}{l}{\thead{Μικροανεύρισμα και ανίχνευση αιμορραγίας}}                                                                                                                                                                     \\
		\midrule
		Haloi 2015~\cite{haloi2015improved}               & CNN             & ανίχνευση MA με χρήση CNN με dropout και maxout (ROC, Messidor, DIA)                                          & 0.98                                                \\
		Giancardo 2017~\cite{giancardo2017representation} & u-net           & ανίχνευση MA με χρήση εσωτερικών αναπαραστάσεων με εκπαιδευμένα u-net (DRIVE, Messidor)                       & \textit{πολλαπλά}                                   \\
		Orlando 2018~\cite{orlando2018ensemble}           & CNN             & ανίχνευση MA και αιμορραγιών με χρήση χαρακτηριστικών και CNN (DIA, e-optha, Messidor)       & \textit{πολλαπλά}                                   \\
		van Grinsven 2017~\cite{van2016fast}              & CNN             & ανίχνευση αιμορραγιών με δειγματοληψία και χρήση CNN πέντε \textit{επιπέδων} (KR15, Messidor) & \textit{πολλαπλά}                                   \\
		\midrule
		\multicolumn{4}{l}{\thead{Άλλες εφαρμογές}}                                                                                                                                                                                              \\
		\midrule
		Girard 2017~\cite{girard2017artery}               & CNN             & ταξινόμηση αρτηριών/φλεβών με χρήση CNN και διάδοση πιθανοφάνειας (DRIVE, Messidor)                           & \textit{πολλαπλά}                                   \\
		Welikala 2017~\cite{welikala2017automated}        & CNN             & ταξινόμηση αρτηριών/φλεβών με χρήση ενός CNN τριών \textit{επιπέδων} (UKBDB)                                  & 82.26\%$^*$                                         \\
		Pratt 2017~\cite{pratt2017automatica}             & ResNet          & ταξινόμηση διασταυρώσεων με χρήση ενός ResNet18 (DRIVE, IOSTAR)                                               & \textit{πολλαπλά}                                   \\
		Poplin 2017~\cite{poplin2017predicting}           & Inception       & πρόβλεψη παραγόντων καρδιαγγειακών κινδύνου (UKBDB, μη-δημόσια)                                               & \textit{πολλαπλά}                                   \\
		\bottomrule
	\end{tabular}
\end{sidewaystable}

Υπάρχουν επίσης και άλλες εφαρμογές του Fundus όπως η αναγνώριση διακλάδωσης/διέλευσης.
Οι Pratt et al.~\cite{pratt2017automatica} εκπαίδευσαν ένα ResNet 18 \textit{επιπέδων}, για να εντοπίσουν μικρά patches που περιλαμβάνουν είτε διακλάδωση είτε διέλευση.
Ένα άλλο ResNet18 εκπαιδεύτηκε σε patches που έχουν ταξινομηθεί ώστε να έχουν διακλαδώσεις και διελεύσεις έτσι ώστε να διακρίνουν σε ποια κατηγορία ανήκει.
Παρόμοια επίλυση σε αυτό το πρόβλημα έχει γίνει από τους ίδιους συγγραφείς~\cite{pratt2017automaticb} χρησιμοποιώντας ένα CNN\@.

Ένα σημαντικό αποτέλεσμα στον τομέα της Καρδιολογίας χρησιμοποιώντας τη Fundus είναι από τους Poplin et al.~\cite{poplin2017predicting} που χρησιμοποίησαν ένα Inception-v3 για να προβλέψουν παράγοντες καρδιαγγειακού κινδύνου (ηλικία, φύλο, κατάσταση καπνίσματος, HbA1c, SBP) και μείζον καρδιακά επεισόδια.
Τα μοντέλα τους χρησιμοποίησαν ξεχωριστές πτυχές της ανατομίας για τη δημιουργία κάθε πρόβλεψης, όπως ο οπτικός δίσκος ή τα αιμοφόρα αγγεία, όπως αποδείχθηκε χρησιμοποιώντας την τεχνική soft-attention.
Τα περισσότερα αποτελέσματα ήταν σημαντικά καλύτερα από ότι θεωρούνταν προηγουμένως δυνατό με τη Fundus (\textgreater{70\%} AUC).

\subsection{Συμπεράσματα της χρήσης βαθιάς μάθησης με Fundus}
Όσον αφορά τη χρήση των αρχιτεκτονικών, υπάρχει σαφής προτίμηση στα CNN ειδικά στον τομέα της κατάτμησης των αγγείων, ενώ μια ενδιαφέρουσα προσέγγιση από ορισμένες δημοσιεύσεις είναι η χρήση των CRF μετά την επεξεργασία για τη περαιτέρω βελτίωση της κατάτμησης των αγγείων.
Το γεγονός ότι υπάρχουν πολλές βάσεις δεδομένων που είναι διαθέσιμες και ότι η βάση δεδομένων DRIVE χρησιμοποιείται κατά κύριο λόγο στην πλειοψηφία της βιβλιογραφίας, καθιστά τον τομέα αυτό ευκολότερο στη σύγκριση και επικύρωση νέων αρχιτεκτονικών.
Επιπλέον, η μη-επεμβατική φύση του Fundus και η πρόσφατη χρήση του ως εργαλείου για την εκτίμηση καρδιαγγειακού κινδύνου, το καθιστά μια πολλά υποσχόμενη απεικονιστική αυξημένης χρησιμότητας στον τομέα της καρδιολογίας.

\section{Ηλεκτρονική τομογραφία}
Η ηλεκτρονική τομογραφία (CT) είναι μια μη-επεμβατική μέθοδος για την ανίχνευση της αποφρακτικής αρτηριακής νόσου.
Ορισμένες από τις περιοχές στις οποίες εφαρμόστηκε η βαθιά μάθηση στην CT, περιλαμβάνουν την αξιολόγηση της βαθμολογίας ασβεστίου της στεφανιαίας αρτηρίας, τον εντοπισμό και τον κατάτμηση των καρδιακών περιοχών.

\begin{sidewaystable}
	\caption{Εφαρμογές βαθιάς μάθησης με χρήση CT}
	\label{table:imaging6}
	\centering
	\begin{tabular}{l c l}
		\toprule
		\thead{Αναφορά}                              & \thead{Μέθοδος} & \thead{Εφαρμογή/Σημειώσεις\footnote{Αποτελέσματα από αυτές τις απεικονιστικές τεχνικές δεν δημοσιεύονται σε αυτήν την βιβλιογραφική αναφορά καθώς υπάρχει υψηλή μεταβλητότητα σε σχέση με το ερευνητικό ερώτημα προς προς απάντηση και στην χρήση των μετρήσεων. Επιπλέον όλες οι δημοσιεύσεις χρησιμοποιούν μη-δημόσιες βάσεις δεδομένων εκτός της~\cite{liu2017left}.}} \\
		\midrule
		Lessman 2016~\cite{lessmann2016deep}          & CNN             & ανίχνευση στεφανιαίου ασβεστίου με χρήση τριών ανεξάρτητα εκπαιδευμένων CNNs                                                                                                                                                                                                                                                                                                                \\
		Shadmi 2018~\cite{shadmi2018fully}            & DenseNet        & συνέκριναν το DenseNet και το u-net για τον εντοπισμό στεφανιαίου ασβεστίου                                                                                                                                                                                                                                                                                                                 \\
		Cano 2018~\cite{cano2018automated}            & CNN             & 3D CNN παλινδρόμησης για τον υπολογισμό της βαθμολογίας Agatston                                                                                                                                                                                                                                                                                                                            \\
		Wolterink 2016~\cite{wolterink2016automatic}  & CNN             & ανίχνευση στεφανιαίου ασβεστίου με χρήση τριών CNNs για τον εντοπισμό και δύο CNN για την ανίχνευση                                                                                                                                                                                                                                                                                         \\
		Santini 2017~\cite{santini2017automatic}      & CNN             & ανίχνευση στεφανιαίου ασβεστίου με χρήση ενός CNN επτά \textit{επιπέδων} σε patches εικόνας                                                                                                                                                                                                                                                                                                 \\
		Lopez 2017~\cite{lopez2017dcnn}               & CNN             & χαρακτηρισμός όγκου θρόμβου με χρήση ενός 2D CNN και μετα-επεξεργασία                                                                                                                                                                                                                                                                                                                       \\
		Hong 2016~\cite{hong2016automatic}            & DBN             & ανίχνευση, τμηματοποίηση, ταξινόμηση αορτικού ανευρύσματος με χρήση DBN και patches εικόνων                                                                                                                                                                                                                                                                                                 \\
		Liu 2017~\cite{liu2017left}                   & CNN             & τμηματοποίηση αριστερού κόλπου καρδιάς με CNN δώδεκα \textit{επιπέδων} και μοντέλα ενεργού σχήματος (STA13)                                                                                                                                                                                                                                                                           \\
		de Vos 2016~\cite{de20162d}                   & CNN             & 3D εντοπισμός των ανατομικών δομών με χρήση τριών CNNs, ένα για κάθε ορθογώνιο επίπεδο                                                                                                                                                                                                                                                                                                      \\
		Moradi 2016~\cite{moradi2016hybrid}           & CNN             & ανίχνευση θέσης στο CT με προεκπαιδευμένο VGGnet, χειροποίητα χαρακτηριστικά και SVM                                                                                                                                                                                                                                                            \\
		Zheng 2015~\cite{zheng20153d}                 & Multiple        & ανίχνευση διακλάδωσης της καρωτίδας με χρήση πολυ-επίπεδων perceptrons και πιθανοτικά boosting-trees                                                                                                                                                                                                                                                                             \\
		Montoya 2018~\cite{montoya2018deep}           & ResNet          & 3D ανακατασκευή αγγειογραφείας με χρήση ενός ResNet-30 \textit{επιπέδων}                                                                                                                                                                                                                                                                                                                    \\
		Zreik 2018~\cite{zreik2018deep}               & CNN, AE         & ανίχνευση στεφανιαίας αρτηριακής στένωσης με CNN για LV τμηματοποίηση και AE, SVM για ταξινόμηση                                                                                                                                                                                                                                                                              \\
		Commandeur 2018~\cite{commandeur2018deep}     & CNN             & ποσοτικοποιήση του επικάρδιου και θωρακικού λιπώδες ιστού από χωρίς-αντίθεση CT                                                                                                                                                                                                                                                                                                             \\
		Gulsun 2016~\cite{gulsun2016coronary}         & CNN             & εξαγωγή στεφανιαίων κεντρικών γραμμών με χρήση βέλτιστου μονοπατιού από πεδίο ροών και ένα CNN                                                                                                                                                                                                                                                                                              \\
		\bottomrule
	\end{tabular}
\end{sidewaystable}

Η μέθοδος των Lessman et al.~\cite{lessmann2016deep} για τη βαθμολόγηση του στεφανιαίου ασβεστίου χρησιμοποιεί τρία ανεξάρτητα εκπαιδευμένα CNNs για να εκτιμήσει ένα πλαίσιο οριοθέτησης γύρω από την καρδιά, στο οποίο τα συνδεδεμένα συστατικά (connected components) πάνω από ένα όριο μονάδας Hounsfield θεωρούνται υποψήφια για CACs.
Η ταξινόμηση των voxels πραγματοποιήθηκε τροφοδοτώντας δισδιάστατα patches από τρία ορθογώνια επίπεδα ταυτόχρονα σε τρία CNNs, για να διαχωριστούν από άλλες περιοχές υψηλής έντασης.
Οι ασθενείς ταξινομήθηκαν σε μία από τις πέντε πρότυπες κατηγορίες καρδιαγγειακού κινδύνου βάσει της βαθμολογίας Agatston.
Συγγραφείς από την ίδια ομάδα δημιούργησαν μια μέθοδο~\cite{lessmann2017automatic} για την ανίχνευση ασβεστοποιήσεων σε θωρακικό CT χαμηλής δόσης χρησιμοποιώντας ένα CNN για ανατομική θέση και ένα άλλο CNN για την ανίχνευση της ασβεστοποίησης.
Στο~\cite{shadmi2018fully} οι συγγραφείς σύγκριναν ένα u-net και DenseNet για τον υπολογισμό της βαθμολογίας Agatston χρησιμοποιώντας πάνω από 1000 εικόνες θωρακικού CT\@.
Οι συγγραφείς επεξεργάστηκαν έντονα τις εικόνες χρησιμοποιώντας κατωφλίωση, ανάλυση connected components και μορφολογικές λειτουργίες για την ανίχνευση των πνευμόνων, της τραχείας και της τροπίδας.
Τα πειράματά τους έδειξαν ότι το DenseNet είχε καλύτερη απόδοση όσον αφορά την ακρίβεια.
Οι Cano et al.~\cite{cano2018automated} εκπαίδευσαν ένα 3D CNN παλινδρόμησης που υπολόγισε την βαθμολογία Agatston χρησιμοποιώντας 5973 εικόνες CT χωρίς ECG περίφραξης επιτυγχάνοντας συσχέτιση Pearson 0.932.
Στο~\cite{wolterink2016automatic} οι συγγραφείς δημιούργησαν μια μέθοδο για την ανίχνευση και τον ποσοτικό προσδιορισμό της CAC, χωρίς την εξαγωγή της στεφανιαίας αρτηρίας.
Η ανίχνευση του πλαισίου οριοθέτησης γύρω από την καρδιά χρησιμοποιεί τρία CNNs, όπου το καθένα ανιχνεύει την καρδιά στο αξονικό, ισχαιμικό και στεφανιαίο επίπεδο.
Ένα άλλο ζεύγος από CNNs χρησιμοποιήθηκε για την ανίχνευση CAC\@.
Το πρώτο CNN αναγνωρίζει κύστεις τύπου CAC, απορρίπτοντας έτσι την πλειονότητα των voxels που δεν είναι υποψήφια CAC όπως ο πνεύμονας και ο λιπώδης ιστός.
Τα αναγνωρισμένα voxels τύπου CAC ταξινομούνται περαιτέρω από το δεύτερο CNN, το οποίο διακρίνει τα CAC\@.
Αν και τα CNNs μοιράζονται την αρχιτεκτονική, δεδομένου ότι έχουν διαφορετικό έργο, δεν μοιράζονται βάρη.
Επιτυγχάνουν μια συσχέτιση Pearson 0.95, συγκρίσιμη με προηγούμενες καλύτερες τεχνικές.
Οι Santini et al.~\cite{santini2017automatic} εκπαίδευσαν ένα CNN επτά \textit{επιπέδων} χρησιμοποιώντας patches, για την κατάτμηση και ταξινόμηση των στεφανιαίων περιοχών στις εικόνες CT\@.
Εκπαίδευσαν, επικύρωσαν και δοκίμασαν το δίκτυό τους σε 45, 18 και 56 CT όγκους αντιστοίχως, επιτυγχάνοντας μια συσχέτιση Pearson 0.983.

Το CT έχει χρησιμοποιηθεί επίσης και για την κατάτμηση διαφόρων καρδιακών περιοχών.
Οι Lopez et al.~\cite{lopez2017dcnn} εκπαίδευσαν ένα 2D CNN για την εκτίμηση του όγκου αορτής θρόμβου από προεγχειρητικές και μετεγχειρητικές κατατμήσεις, χρησιμοποιώντας περιστροφικές και κατοπτρικές επαυξήσεις δεδομένων.
Μεταγενέστερη επεξεργασία περιλαμβάνει Gaussian φιλτράρισμα και ομαδοποίηση κ-μέσων.
Στο άρθρο τους, οι Hong et al.~\cite{hong2016automatic} εκπαίδευσαν ένα DBN χρησιμοποιώντας patches εικόνας για την ανίχνευση, τμηματοποίηση και ταξινόμηση της σοβαρότητας του κοιλιακού αορτικού ανευρύσματος σε CT εικόνες.
Οι Liu et al.~\cite{liu2017left} χρησιμοποίησαν ένα FCN με δώδεκα \textit{επίπεδα} για την κατάτμηση του αριστερού κόλπου σε 3D όγκους CT και στη συνέχεια βελτίωσαν τα αποτελέσματα τμηματοποίησης του FCN με ένα μοντέλο ενεργού σχήματος επιτυγχάνοντας Dice 93\%.

Το CT έχει επίσης χρησιμοποιηθεί για τον εντοπισμό των καρδιακών περιοχών.
Στο~\cite{de20162d} οι συγγραφείς δημιούργησαν μια μέθοδο για την ανίχνευση ανατομικών ROI (καρδιά, αορτική αψίδα και φθίνουσα αορτή) σε εικόνες από θωρακικό CT προκειμένου να εντοπιστούν σε 3D.
Κάθε ROI εντοπίστηκε χρησιμοποιώντας έναν συνδυασμό τριών CNNs, κάθε ένα αναλύοντας ένα ορθογώνιο επίπεδο εικόνας.
Ενώ ένα CNN προέβλεψε την παρουσία ενός συγκεκριμένου ROI στο δεδομένο επίπεδο, ο συνδυασμός των αποτελεσμάτων τους παρείχε ένα 3D πλαίσιο οριοθέτησης γύρω του.
Στο άρθρο τους οι Moradi et al.~\cite{moradi2016hybrid} αντιμετωπίζουν το πρόβλημα της ανίχνευσης κατακόρυφης θέσης για μια δεδομένη καρδιακή εικόνα CT\@.
Διαχωρίζουν την περιοχή του σώματος που απεικονίζεται στο θωρακικό CT σε εννέα σημασιολογικές κατηγορίες, οι οποίες κάθε μια αντιπροσωπεύουν μία περιοχή που είναι σχετική με τη μελέτη αντίστοιχων νόσων.
Χρησιμοποιώντας ένα σύνολο χειροποίητων χαρακτηριστικών εικόνας μαζί με τα χαρακτηριστικά που προέρχονται από ένα προεκπαιδευμένο VGGnet με πέντε \textit{επίπεδα}, χτίζουν ένα σχήμα ταξινόμησης για να αντιστοιχίσουν μια δεδομένη εικόνα CT στο σχετικό επίπεδο.
Κάθε ομάδα χαρακτηριστικών χρησιμοποιήθηκε για την εκπαίδευση ενός ξεχωριστού ταξινομητή SVM και οι προβλεπόμενες ετικέτες στη συνέχεια συνδυάζονται σε ένα γραμμικό μοντέλο, το οποίο επίσης αντλήθηκε από τα δεδομένα εκπαίδευσης.

Η βαθιά μάθηση χρησιμοποιήθηκε σε συνδυασμό με το CT και για άλλες περιοχές εκτός από την καρδιάς.
Οι Zheng et al.~\cite{zheng20153d} δημιούργησαν μια μέθοδο για 3D ανίχνευση σε ογκομετρικά δεδομένα, τα οποία αξιολογήθηκαν ποσοτικά για την ανίχνευση της διακλάδωσης της καρωτιδικής αρτηρίας σε CT\@.
Χρησιμοποιήθηκε ένα δίκτυο κρυφών επιπέδων για τον αρχικό έλεγχο όλων των voxels για να αποκτηθεί ένας μικρός αριθμός υποψηφίων, ακολουθούμενος από μια ακριβέστερη ταξινόμηση με ένα βαθύ δίκτυο.
Τα χαρακτηριστικά από το δίκτυο συνδυάζονται περαιτέρω με τα χαρακτηριστικά κυματιδίων Haar, για την αύξηση της ακρίβειας ανίχνευσης.
Οι Montoya et al.~\cite{montoya2018deep} εκπαίδευσαν ένα ResNet με 30 επίπεδα, για να δημιουργήσουν 3D αγγειογραφήματα χρησιμοποιώντας τρεις τύπους ιστών (αγγειακό σύστημα, οστό και μαλακό ιστό).
Δημιούργησαν τις επισημάνσεις χρησιμοποιώντας κατωφλίωση και connected components σε 3D, έχοντας ένα συνδυασμένο σύνολο 13790 εικόνων.

Το CT έχει επίσης χρησιμοποιηθεί για την επίλυση και άλλων προβλημάτων.
Οι Zreik et al.~\cite{zreik2018deep} δημιούργησαν μια μέθοδο για την ταυτοποίηση ασθενών με στένωση στεφανιαίας αρτηρίας από το μυοκάρδιο του LV από CT\@.
Χρησιμοποίησαν ένα CNN πολλαπλών κλιμάκων για να τμηματοποιήσουν το μυοκάρδιο του LV και έπειτα το κωδικοποίησαν χρησιμοποιώντας ένα μη-επιβλεπώμενο συνελικτικό ΑΕ.
Η τελική ταξινόμηση έγινε με έναν ταξινομητή SVM με βάση τα εξαγόμενα και ομαδοποιημένα clustering.
Παρόμοια δουλειά έχει γίνει από τους ίδιους συγγραφείς~\cite{zreik2016automatic}, οι οποίοι χρησιμοποίησαν τρία CNNs για να ανιχνεύσουν ένα πλαίσιο οριοθέτησης γύρω από την LV και εφάρμοσαν ταξινόμηση voxel LV μέσα στο κιβώτιο οριοθέτησης.
Οι Commandeur et al.~\cite{commandeur2018deep} χρησιμοποίησαν ένα συνδυασμό δύο βαθιών δικτύων για τον ποσοτικό προσδιορισμό του επικαρδιακού και θωρακικού λιπώδες ιστού, με CT από 250 ασθενείς με 55 εικόνες ανά ασθενή κατά μέσο όρο.
Το πρώτο δίκτυο είναι ένα CNN έξι \textit{επιπέδων}, που ανιχνεύει την περιοχή που βρίσκεται μέσα στα όρια της καρδιάς και διαχωρίζει τις θωρακικές και επικαρδιακές-παρακαρδιακές μάσκες.
Το δεύτερο δίκτυο είναι ένα CNN πέντε \textit{επιπέδων}, που ανιχνεύει τη γραμμή του περικαρδίου από την αξονική τομογραφία σε κυλινδρικές συντεταγμένες.
Στη συνέχεια ένα στατιστικό μοντέλο συστηματοποίησης μαζί με εφαρμογή κατωφλίωσης και φιλτράρισμα διαμέσου, παρέχουν τις τελικές κατατμήσεις.
Οι Gulsun et al.~\cite{gulsun2016coronary} δημιούργησαν μια μέθοδο για την εξαγωγή των κεντρικών γραμμών των αγγείων στην CT\@.
Πρώτον, οι βέλτιστες διαδρομές ανιχνεύονται σε ένα υπολογισμένο πεδίο ροής και στη συνέχεια χρησιμοποιείται ένας ταξινομητής CNN για την αφαίρεση των εξωτερικών διαδρομών στις ανιχνευμένες κεντρικές γραμμές.
Η μέθοδος ενισχύθηκε χρησιμοποιώντας μια ανίχνευση βασισμένη στο μοντέλο των συγκεκριμένων στεφανιαίων περιοχών και των κύριων κλάδων για να περιορίσουν τον χώρο αναζήτησης.

\section{Ηχοκαρδιογράφημα}
Το ηχοκαρδιογράφημα είναι μια μέθοδος απεικόνισης της περιοχής της καρδιάς χρησιμοποιώντας υπερηχητικά κύματα.
Χρήσεις της βαθιάς μάθησης στο ηχοκαρδιογράφημα περιλαμβάνουν κυρίως την κατάτμηση του LV και την αξιολόγηση της ποιότητας της βαθμολογίας της εικόνας, μεταξύ άλλων.

\begin{sidewaystable}
	\caption{Εφαρμογές βαθιάς μάθησης με χρήση Ηχοκαρδιογραφήματος}
	\label{table:imaging7}
	\centering
	\begin{tabular}{l c l}
		\toprule
		\thead{Αναφορά}                              & \thead{Μέθοδος} & \thead{Εφαρμογή/Σημειώσεις\footnote{Αποτελέσματα από αυτές τις απεικονιστικές τεχνικές δεν δημοσιεύονται σε αυτήν την βιβλιογραφική αναφορά καθώς υπάρχει υψηλή μεταβλητότητα σε σχέση με το ερευνητικό ερώτημα προς απάντηση και στην χρήση των μέτρων. Επιπλέον όλες οι δημοσιεύσεις χρησιμοποιούν μη-δημόσιες βάσεις δεδομένων.}} \\
		\midrule
		Carneiro 2012~\cite{carneiro2012segmentation} & DBN             & τμηματοποίηση LV με αποσύνδεση στοιβαρών και μη-στοιβαρών ανιχνεύσεων με χρήση DBN σε 480 εικόνες                                                                                                                                                                                                                                                                                           \\
		Nascimento 2016~\cite{nascimento2016multi}    & DBN             & τμηματοποίηση LV με μάθηση επιφανειών και ένα DBN                                                                                                                                                                                                                                                                                                                                     \\
		Chen 2016~\cite{chen2016iterative}            & CNN             & τμηματοποίηση LV με συστηματοποιημένο FCN πολλαπλών τομέων και μεταφορά μάθησης                                                                                                                                                                                                                                                                                                      \\
		Madani 2018~\cite{madani2018fast}             & CNN             & ταξινόμηση όψης διαθωρακικού ηχοκαρδιογραφήματος με CNN έξι \textit{επιπέδων} \\
		Silva 2018~\cite{silva2018ejection}           & CNN             & ταξινόμηση κλάσματος εξώθησης με ένα residual 3D CNN και διαθωρακικά ηχοκαρδιογραφήματα                                                                                                                                                                                                                                                                                               \\
		Gao 2017~\cite{gao2017fused}                  & CNN             & ταξινόμηση όψης με σύντηξη δύο CNNs με επτά \textit{επίπεδα} το καθένα                                                                                                                                                                                                                                                                                                                      \\
		Abdi 2017~\cite{abdi2017quality}              & CNN, LSTM       & αξιολόγηση ποιότητας βαθμολογίας με συνελικτικά και επαναληπτικά επιπέδα                                                                                                                                                                                                                                                                                                           \\
		Ghesu 2016~\cite{ghesu2016marginal}           & CNN             & τμηματοποίηση αορτικής βαλβίδας με χρήση 2891 3D διαεισοφαγικών ηχοκαρδιογραφικών εικόνων                                                                                                                                                                                                                                                                                                   \\
		Perrin 2017~\cite{perrin2017application}      & CNN             & ταξινόμηση εκ-γενετής καρδιακής ασθένειας με CNN                                                                                                                                                                                                                                                                                                                                      \\
		Moradi 2016~\cite{moradi2016cross}            & VGGnet, doc2vec & δημιουργία σημαντικών περιγραφέων για εικόνες                                                                                                                                                                                                                                                                                                                                               \\
		\bottomrule
	\end{tabular}
\end{sidewaystable}

Τα DBN έχουν χρησιμοποιηθεί για την κατάτμηση της LV στα ηχοκαρδιογραφήματα.
Στο~\cite{carneiro2012segmentation} οι συγγραφείς δημιούργησαν μια μέθοδο που αποσυνδέει τις άκαμπτες και μη-άκαμπτες ανιχνεύσεις, με ένα DBN που μοντελοποιεί το LV δείχνοντας ότι είναι πιο ισχυρό από τα level-sets και τα παραμορφώσιμα μοντέλα.
Οι Nascimento et al.~\cite{nascimento2016multi} χρησιμοποιούν μάθηση του πολυειδούς που χωρίζει τα δεδομένα σε patches όπου το κάθε ένα προτείνει μια κατάτμηση της LV\@.
Η σύντηξη των patches πραγματοποιήθηκε από έναν πολλαπλό ταξινομητή DBN ο οποίος αποδίδει ένα βάρος σε κάθε patch.
Με τον τρόπο αυτό, η μέθοδος δεν βασίζεται σε μια μόνο κατάτμηση και η διαδικασία εκπαίδευσης παράγει ισχυρά μοντέλα χωρίς την ανάγκη μεγάλων βάσεων δεδομένων εκπαίδευσης.
Στο~\cite{chen2016iterative} οι συγγραφείς χρησιμοποίησαν ένα συστηματοποιημένο FCN και μεταφορά μάθησης.
Συγκρίνουν τη μέθοδο τους με απλούστερες αρχιτεκτονικές FCN και μια προηγούμενη μέθοδο παρουσιάζοντας καλύτερα αποτελέσματα.

Το ηχοκαρδιογράφημα έχει επίσης χρησιμοποιηθεί για την ταξινόμηση των όψεων της απεικόνισης.
Οι Madani et al.~\cite{madani2018fast} εκπαίδευσαν CNN έξι \textit{επιπέδων} για να ταξινομήσουν 15 προβολές (12 βίντεο και 3 ακίνητα) με διαθωρακικό υπερηχογράφημα, επιτυγχάνοντας καλύτερα αποτελέσματα από τους ειδικούς ηχοκαρδιογραφήματος.
Στο~\cite{silva2018ejection} οι συγγραφείς δημιούργησαν ένα residual 3D CNN για ταξινόμηση κλάσματος εξώθησης από εικόνες διαθωρακικού ηχοκαρδιογραφήματος.
Χρησιμοποίησαν 8715 εξετάσεις κάθε μια με 30 διαδοχικές εικόνες του apical θαλάμου για να εκπαιδεύσουν και να δοκιμάσουν τη μέθοδο τους επιτυγχάνοντας ικανοποιητικά αποτελέσματα.
Οι Gao et al.~\cite{gao2017fused} ενσωμάτωσαν χωρική και χρονική πληροφορία από τις εικόνες του βίντεο της κινούμενης καρδιάς, με τη σύντηξη δύο CNNs με επτά \textit{επίπεδα} το καθένα.
Η μέτρηση της επιτάχυνσης σε κάθε σημείο υπολογίστηκε με τη χρήση μεθόδου πυκνής οπτικής ροής, για την απεικόνιση πληροφοριών χρονικής κίνησης.
Στη συνέχεια η σύντηξη των CNN πραγματοποιήθηκε χρησιμοποιώντας γραμμικές ενσωματώσεις των διανυσμάτων των εξόδων τους.
Συγκρίσεις έγιναν με προηγούμενες προσεγγίσεις βασισμένες σε χειροκίνητη δημιουργία χαρακτηριστικών, καταδεικνύοντας καλύτερα αποτελέσματα.

Τέλος, η αξιολόγηση της ποιότητας των εικόνων και άλλα προβλήματα επιλύθηκαν με τη χρήση του ηχοκαρδιογραφήματος.
Στο~\cite{abdi2017quality} οι συγγραφείς δημιούργησαν μια μέθοδο για τη μείωση της μεταβλητότητας των δεδομένων κατα τη διάρκεια της επισήμανσης από τον χειριστή, υπολογίζοντας μια βαθμολογία ποιότητας που ανατροφοδοτείται σε πραγματικό χρόνο.
Το μοντέλο συνίσταται από συνελικτικά επίπεδα για την εξαγωγή χαρακτηριστικών από την είσοδο και επαναλαμβανόμενα επίπεδα για χρήση των χρονικών πληροφοριών.
Η μέθοδος των Ghesu et al.~\cite{ghesu2016marginal} ανίχνευσης αντικειμένων και κατάτμησης στο πλαίσιο της ογκομετρικής ανάλυσης εικόνων, γίνεται με την επίλυση της ανατομικής εκτίμησης της θέσης και της οριοθέτησης.
Για το σκοπό αυτό εισάγουν βαθιά μάθηση της οριοθέτησης, η οποία παρέχει υψηλής απόδοσης χρόνο εκτέλεσης, μαθαίνοντας ταξινομητές σε περιοχές υψηλής πιθανότητας και σε χώρους σταδιακά αυξανόμενων διαστάσεων.
Δεδομένου του εντοπισμού των αντικειμένων, προτείνουν ένα συνδυασμένο μοντέλο ενεργού σχήματος βαθιάς μάθησης για την εκτίμηση του μη-άκαμπτου ορίου αντικειμένου.
Στο άρθρο τους οι Perrin et al.~\cite{perrin2017application} εκπαίδευσαν και αξιολόγησαν το AlexNet με 59151 εικόνες, για να ταξινομήσουν μεταξύ πέντε παιδιατρικών πληθυσμών με συγγενή καρδιακή νόσο.
Οι Moradi et al.~\cite{moradi2016cross} δημιούργησαν μια μέθοδο που βασίζεται σε VGGnet και doc2vec για να παράξουν σημασιολογικούς περιγραφείς εικόνων, οι οποίοι μπορούν να χρησιμοποιηθούν ως ασθενώς επισημασμένες περιπτώσεις ή να διορθωθούν από ιατρικούς εμπειρογνώμονες.
Το μοντέλο τους ήταν σε θέση να αναγνωρίσει το 91\% των ασθενειών και 77\% των σοβαρών νόσων από εικόνες Doppler καρδιακών βαλβίδων.

\section{Τομογραφία οπτικής συνοχής}
Η Τομογραφία Οπτικής Συνοχής (Optical Coherence Tomography, OCT) είναι μια ενδοαγγειακή απεικόνιση που παρέχει εικόνες αρτηριών υψηλής ανάλυσης και ποσοτικές μετρήσεις της στεφανιαίας γεωμετρίας στο κλινικό περιβάλλον~\cite{kubo2013oct}.

Στο άρθρο τους οι Roy et al.~\cite{roy2016multiscale} χαρακτηρίζουν τον ιστό στο OCT, μαθαίνοντας την πολυκλιμακωτή κατανομή των δεδομένων με ένα AE\@.
Ο κανόνας μάθησης του δικτύου εισάγει μια παράμετρο κλίμακας που συνδέεται με το backpropagation.
Σε σύγκριση με τρία προεκπαιδευμένα ΑΕ βάσης αναφοράς επιτυγχάνεται καλύτερη απόδοση όσον αφορά την ακρίβεια στην ανίχνευση πλακών/κανονικών εικονοστοιχείων.
Οι Yong et al.~\cite{yong2017linear} δημιούργησαν ένα CNN γραμμικής παλινδρόμησης με τέσσερα \textit{επίπεδα}, για να χωρίσουν τον αυλό του αγγείου παραμετροποιημένο σε όρους ακτινικών αποστάσεων από το κέντρο του καθετήρα στις πολικές συντεταγμένες.
Η υψηλή ακρίβεια αυτής της μεθόδου μαζί με την υπολογιστική αποτελεσματικότητά της (40.6ms/εικόνα), υποδεικνύουν τη δυνατότητα χρήσης σε πραγματικό κλινικό περιβάλλον.
Στο~\cite{xu2017fibroatheroma} οι συγγραφείς σύγκριναν την διακριτική ικανότητα των βαθιών χαρακτηριστικών που εξάγονται από τα AlexNet, GoogleNet, VGGnet16 και VGGnet19 για την ταυτοποίηση του ινωαθερώματος.
Η επαύξηση δεδομένων εφαρμόστηκε σε μια βάση δεδομένων OCT για κάθε σχήμα ταξινόμησης και επίσης εφαρμόστηκε γραμμικό SVM για την ταξινόμηση των εικόνων ως κανονικών ή ινωαθερωματικών.
Τα αποτελέσματα υποδεικνύουν ότι το VGGnet19 είναι καλύτερο για τον εντοπισμό εικόνων που περιέχουν ινωαθέρωμα.
Οι Abdolmanafi et al.~\cite{abdolmanafi2017deep} ταξινομούν τον ιστό στο OCT χρησιμοποιώντας ένα προεκπαιδευμένο AlexNet ως εξαγωγέα χαρακτηριστικών και συγκρίνουν τις προβλέψεις τριών ταξινομητών, CNN, RF και SVM, με το πρώτο να επιτυγχάνει τα καλύτερα αποτελέσματα.

\section{Άλλες απεικονιστικές τεχνικές}
Το ενδοαγγειακό υπερηχογράφημα (Intravascular Ultrasound, IVUS) χρησιμοποιεί μετασχηματιστές τοποθετημένους σε τροποποιημένους ενδοστεφανιαίους καθετήρες για την παροχή ακτινικής ανατομικής απεικόνισης ενδοστεφανιαίας ασβεστοποίησης και σχηματισμού πλάκας~\cite{parrillo2013critical}.
Οι Lekadir et al.~\cite{lekadir2017convolutional} χρησιμοποίησαν ένα βασισμένο σε patches CNN τεσσάρων \textit{επιπέδων}, για τον χαρακτηρισμό της σύνθεσης της πλάκας σε καρωτιδικές εικόνες υπερήχων.
Τα πειράματα που έγιναν από τους συγγραφείς έδειξαν ότι το μοντέλο επιτυγχάνει καλύτερη ακρίβεια από SVM μονών και πολλαπλών κλιμάκων.
Στο~\cite{tajbakhsh2017automatic} οι συγγραφείς αυτοματοποίησαν ολόκληρη τη διαδικασία της ερμηνείας του πάχους του καρωτιδικού intima-media.
Εκπαίδευσαν ένα CNN δύο \textit{επιπέδων} με δύο εξόδους για την επιλογή της εικόνας και ένα CNN δύο \textit{επιπέδων} με τρεις εξόδους για τον εντοπισμό του ROI και τις μετρήσεις πάχους intima-media.
Το μοντέλο αυτό αποδίδει καλύτερα από προηγούμενη μέθοδο από τους ίδιους συγγραφείς, το οποίο δικαιολογεί την ικανότητα του CNN να μαθαίνει την εμφάνιση του QRS και του ROI, αντί να στηρίζεται σε εφαρμογή κατωφλίωσης με κατώτατο πλάτος και καμπυλότητα.
Οι Tom et al.~\cite{tom2018simulating} δημιούργησαν μια μέθοδο βασισμένη στα Δίκτυα Γεννήτριας-Διάκρισης (Generative Adversarial Networks, GANs) για χρονικά αποδοτική προσομοίωση ρεαλιστικού IVUS\@.
Η προσομοίωση στο πρώτο στάδιο αφορούσε τον προσομοιωτή ψευδότροπου IVUS και την χαρτογράφηση του στίγματος της ψηφιακά καθορισμένης εικόνας.
Στο δεύτερο στάδιο βελτιώθηκαν οι αντιστοιχίσεις για τη διατήρηση των συγκεκριμένων εντάσεων στίγματος με ιστούς χρησιμοποιώντας ένα GAN με τέσσερα residual \textit{επίπεδα}, ενώ στο τελευταίο στάδιο το GAN παρήξε εικόνες υψηλής αναλυτικότητας με παθορεαλιστικά προφίλ στίγματος.

\begin{sidewaystable}
	\caption{Εφαρμογές βαθιάς μάθησης με χρήση OCT και άλλων απεικονιστικών τεχνικών}
	\label{table:imaging8}
	\centering
	\begin{tabular}{l c l}
		\toprule
		\thead{Αναφορά}                             & \thead{Μέθοδος} & \thead{Εφαρμογή/Σημειώσεις\footnote{Αποτελέσματα από αυτές τις απεικονιστικές τεχνικές δεν δημοσιεύονται σε αυτήν την βιβλιογραφική αναφορά καθώς υπάρχει υψηλή μεταβλητότητα σε σχέση με το ερευνητικό ερώτημα προς προς απάντηση και στην χρήση των μετρήσεων. Επιπλέον όλες οι δημοσιεύσεις χρησιμοποιούν μη-δημόσιες βάσεις δεδομένων εκτός της~\cite{tom2018simulating}.}} \\
		\midrule
		\multicolumn{3}{l}{\thead{OCT}}                                                                                                                                                                                                                                                                                                                                                                                                                             \\
		\midrule
		Roy 2016~\cite{roy2016multiscale}            & AE              & χαρακτηρισμός ιστού με χρήση ενός AE διατήρησης της κατανομής                                                                                                                                                                                                                                                                                                                               \\
		Yong 2017~\cite{yong2017linear}              & CNN             & τμηματοποίηση του lumen με χρήση ενός CNN γραμμικής-παλινδρόμησης με τέσσερα \textit{επίπεδα}                                                                                                                                                                                                                                                                                               \\
		Xu 2017~\cite{xu2017fibroatheroma}           & CNN             & παρουσία του ινωαθερώματος με χρήση χαρακτηριστικών εξαγόμενων από προηγούμενες αρχιτεκτονικές και ένα SVM                                                                                                                                                                                                                                                                                   \\
		Abdolmanafi 2017~\cite{abdolmanafi2017deep}  & CNN             & τμηματοποίηση του intima-media με χρήση ενός προεκπαιδευμένου AlexNet και σύγκριση διαφόρων ταξινομητών                                                                                                                                                                                                                                                                                     \\
		\midrule
		\multicolumn{3}{l}{\thead{Άλλες απεικονιστικές τεχνικές}}                                                                                                                                                                                                                                                                                                                                                                                                   \\
		\midrule
		Lekadir 2017~\cite{lekadir2017convolutional} & CNN             & χαρακτηρισμός καρωτιδικής πλάκας με χρήση CNN τεσσάρων \textit{επιπέδων} με υπερήχους                                                                                                                                                                                                                                                                                                       \\
		Tajbakhsh 2017~\cite{tajbakhsh2017automatic} & CNN             & ερμηνεία video του πάχους των καρωτιδικών intima-media με χρήση δύο CNNs δύο \textit{επίπεδα} με υπερήχους                                                                                                                                                                                                                                                                                 \\
		Tom 2017~\cite{tom2018simulating}            & GAN             & δημιουργία IVUS εικόνων με χρήση δύο GANs (IV11)                                                                                                                                                                                                                                                                                                                                            \\
		Wang 2017~\cite{wang2017detecting}           & CNN             & ασβεστοποίηση αρτηριών μαστού με χρήση CNN δέκα \textit{επιπέδων} σε μαστογραφίες                                                                                                                                                                                                                                                                                                           \\
		Liu 2017~\cite{liu2017coronary}              & CNN             & ανίχνευση CAC με χρήση CNNs σε 1768 X-Rays                                                                                                                                                                                                                                                                                                                                                  \\
		Pavoni 2017~\cite{pavoni2017image}           & CNN             & αποθορυβοποίηση του διαδερμικού αυλού στεφανιαίων εικόνων αγγειοπλαστικής με ένα CNN τεσσάρων \textit{επιπέδων}                                                                                                                                                                                                                                                                             \\
		Nirschl 2018~\cite{nirschl2018deep}          & CNN             & CNN έξι \textit{επιπέδων} βασισμένο σε patch για ανίχνευση καρδιακής ανεπάρκειας σε ενδομυοκαρδιακές εικόνες βιοψίας                                                                                                                                                                                                                                                                        \\
		Betancur 2018~\cite{betancur2018deep}        & CNN             & CNN τριών \textit{επιπέδων} για προβλεψη CAD από απεικονίσεις του μυοκαρδίου                                                                                                                                                                                                                                                                                                                \\
		\bottomrule
	\end{tabular}
\end{sidewaystable}

Άλλες εφαρμογές καρδιολογίας με χρήση μεθόδων βαθιάς μάθησης, περιλαμβάνουν μαστογραφίες, ακτίνες Χ, διαδερμική διαφραγματική αγγειοπλαστική, εικόνες βιοψίας και απεικόνιση διάχυσης του μυοκαρδίου.
Στο άρθρο τους οι Wang et al.~\cite{wang2017detecting} εφάρμοσαν μια διαδικασία βασισμένη σε εικονοστοιχεία και patches για την ανίχνευση ασβεστοποιητικού αρτηριακού μαστού σε μαστογραφίες, χρησιμοποιώντας ένα CNN δέκα \textit{επιπέδων} και μορφολογικές λειτουργίες για μετα-επεξεργασία.
Οι συγγραφείς χρησιμοποίησαν 840 εικόνες και τα πειράματά τους οδήγησαν σε ένα μοντέλο που πέτυχε συντελεστή προσδιορισμού 96.2\%.
Οι Liu et al.~\cite{liu2017coronary} εκπαίδευσαν ένα CNN χρησιμοποιώντας 1768 εικόνες ακτίνων Χ με αντίστοιχες επισημάνσεις διαγνώσεων.
Η μέση διαγνωστική ακρίβεια των μοντέλων έφτασε στο 0.89 όταν το βάθος του δικτύου ήταν οκτώ επίπεδα; μετά από αυτό η αύξηση της ακρίβειας ήταν περιορισμένη.
Στο~\cite{pavoni2017image} οι συγγραφείς δημιούργησαν μια μέθοδο για την αποθορυβοποίηση εικόνων διαδερμικής αγγειοπλαστικής στεφανιαίας.
Δοκίμασαν το μέσο τετραγωνικό σφάλμα και τη δομική ομοιότητα ως συναρτήσεις απωλειών σε δύο patch CNN με τέσσερα \textit{επίπεδα} και τα σύγκριναν με διαφορετικούς τύπους και επίπεδα θορύβου.
Οι Nirschl et al.~\cite{nirschl2018deep} χρησιμοποίησαν εικόνες ενδομυοκαρδιακής βιοψίας από 209 ασθενείς για να εκπαιδεύσουν και να δοκιμάσουν ένα patch CNN έξι \textit{επιπέδων}, για τον εντοπισμό της καρδιακής ανεπάρκειας.
Επίσης εφάρμοσαν περιστροφική επαύξηση δεδομένων, ενώ υπολογίστηκε ο μέσος όρος των εξόδων του CNN σε κάθε patch για να ληφθεί η πιθανότητα για κάθε εικόνα.
Αυτό το μοντέλο έδειξε καλύτερα αποτελέσματα από τα AlexNet, Inception και ResNet50.
Στο~\cite{betancur2018deep} οι συγγραφείς εκπαίδευσαν ένα CNN τριών \textit{επιπέδων} για την πρόβλεψη της αποφρακτικής αιμάτωσης μυοκαρδίου του CAD από 1638 ασθενείς και το συνέκριναν με το συνολικό έλλειμμα διάχυσης.
Το μοντέλο υπολογίζει την πιθανότητα ανά αγγείο κατά τη διάρκεια της εκπαίδευσης, ενώ κατά τη διάρκεια της δοκιμής χρησιμοποιούνται οι μέγιστες πιθανότητες ανά αρτηρία ανά βαθμολογία ασθενούς.
Τα αποτελέσματα δείχνουν ότι αυτή η μέθοδος υπερέχει του συνολικού ελλείμματος διάχυσης στο πρόβλημα της πρόβλεψης των αγγείων και των ασθενών.

\section{Συμπεράσματα της χρήσης βαθιάς μάθησης με CT, ηχοκαρδιογράφημα, OCT και άλλων απεικονιστικών τεχνικών}
Δημοσιεύσεις που χρησιμοποίησαν αυτούς τους τρόπους απεικόνισης παρουσίασαν υψηλή μεταβλητότητα από την πλευρά του ερευνητικού τους ερωτήματος και ήταν κατά πλειοψηφία ασυμβίβαστα σε σχέση με τη χρήση μέτρων αξιολόγησης για τα αποτελέσματα που ανέφεραν.
Αυτές οι μέθοδοι απεικόνισης επίσης δεν έχουν διαθέσιμες δημόσιες βάσεις δεδομένων, περιορίζοντας έτσι τις ευκαιρίες για δημιουργία νέων αρχιτεκτονικών από ερευνητικές ομάδες που δεν έχουν κλινικούς συνεργάτες.
Από την άλλη πλευρά, υπάρχει σχετικά υψηλή ομοιομορφία όσον αφορά τη χρήση αρχιτεκτονικών με το πιο διαδεδομένο το CNN, ειδικά τις προεκπαιδευμένες αρχιτεκτονικές του διαγωνισμού ImageNet (AlexNet, VGGnet, GoogleNet, ResNet).

\section{Συζήτηση και μελλοντικές κατευθύνσεις}
\label{sec4:discussion}

\begin{sidewaystable}
	\caption{Βιβλιογραφικές αναφορές βαθιάς μάθησης με εφαρμογή στην καρδιολογία}
	\label{table:reviews}
	\centering
	\begin{tabular}{l l}
		\toprule
		\thead{Αναφορά}                                    & \thead{Εφαρμογή/Σημειώσεις}                                                                                     \\
		\midrule
		Mayer 2015~\cite{mayer2015big}                      & Big data στην καρδιολογία αλλάζει τον τρόπο με τον οποίο δημιουργούνται καινούργιες ιδέες                       \\
		Austin 2016~\cite{austin2016application}            & σφαιρική εικόνα των big data, τα πλεονεκτήματα, πιθανά μειονεκτήματα και μελλοντική συνεισφορά στην καρδιολογία \\
		Greenspan 2016~\cite{greenspan2016guest}            & ανίχνευση ιστών, τμηματοποίηση και μοντελοποίηση ενεργού σχήματος                                               \\
		Miotto 2017~\cite{miotto2017deep}                   & απεικονιστικές, EHR, γονιδιωματική, δεδομένα από φορητές συσκευές και η ανάγκη για αύξηση της ερμηνευσιμότητας  \\
		Krittanawong 2017~\cite{krittanawong2017rise}       & μελέτες πάνω στις τεχνολογίες αναγνώρισης εικόνας που προβλέπουν καλύτερα από τους ειδικούς                     \\
		Litjens 2017~\cite{litjens2017survey}               & ταξινόμηση εικόνων, εντοπισμός αντικειμένου, τμηματοποίηση και εγγραφή (registration)                           \\
		Qayyum 2017~\cite{qayyum2017medical}                & μέθοδοι βασισμένοι στο CNN στην τμηματοποίηση εικόνας, ταξινόμηση, διάγνωση και ανάκτηση εικόνας                \\
		Hengling 2017~\cite{henglin2017machine}             & επίπτωση που θα έχει η μηχανική μάθηση στο μέλλον της καρδιαγγειακής απεικόνισης                                \\
		Blair 2017~\cite{blair2017advanced}                 & πρόοδος στη νευροεπιστήμη με MRI στην νόσο των μικρών αγγείων                                                   \\
		Slomka 2017~\cite{slomka2017cardiac}                & πυρηνική καρδιολογία, CT αγγειογραφία Ηχοκαρδιογράφημα, MRI                                                      \\
		Carneiro 2017~\cite{carneiro2017review}             & μαστογραφία, καρδιαγγειακή και μικροσκοπική απεικόνιση                                                          \\
		Johnson 2018~\cite{johnson2018artificial}           & AI στην καρδιολογία, έννοιες προβλεπτικών μοντέλων, κοινοί αλγόριθμοι και χρήση της βαθιάς μάθησης               \\
		Jiang 2017~\cite{jiang2017artificial}               & εφαρμογές του AI στην αναγνώριση της καρδιακής προσβολής, διάγνωση, θεραπεία, πρόβλεψη και εκτίμηση πρόγνωσης   \\
		Lee 2017~\cite{lee2017deepb}                        & AI στην απεικονιστική καρδιακής προσβολής εστιασμένο στις τεχνικές αρχές και κλινικές εφαρμογές                 \\
		Loh 2017~\cite{loh2017deep}                         & διάγνωση καρδιακής ασθένειας και διαχείριση μέσα στο πλαίσιο της υγειονομικής περίθαλψης                        \\
		Krittanawong 2017~\cite{krittanawong2017artificial} & καρδιαγγειακή κλινική περίθαλψη και ο ρόλος της ακριβής καρδιαγγειακής ιατρικής                                 \\
		Gomez 2018~\cite{gomez2018new}                      & πρόσφατη πρόοδος στην αυτοματοποίηση και ποσοτική ανάλυση στην πυρηνική καρδιολογία                             \\
		Shameer 2018~\cite{shameer2018machine}              & υποσχέσεις και περιορισμοί της υλοποίησης της μηχανικής μάθησης στην καρδιαγγειακή ιατρική                      \\
		Shrestha 2018~\cite{shrestha2018machine}            & εφαρμογές μηχανικής μάθησης στην πυρηνική καρδιολογία                                                           \\
		Kikuchi 2018~\cite{kikuchi2018future}               & εφαρμογές του AI στην πυρηνική καρδιολογία και το πρόβλημα των περιορισμένων αριθμών δεδομένων                  \\
		Awan 2018~\cite{awan2018machine}                    & εφαρμογές μηχανικής μάθησης στην διάγνωση καρδιακής ανεπάρκειας, ταξινόμηση και πρόβλεψη επανεισδοχής           \\
		Faust 2018~\cite{faust2018deep}                     & εφαρμογές βαθιάς μάθησης στα φυσιολογικά δεδομένα συμπεραλαμβανομένου του ECG                                   \\
		\bottomrule
	\end{tabular}
\end{sidewaystable}

Είναι προφανές από τη βιβλιογραφία ότι οι μέθοδοι βαθιάς μάθησης θα αντικαταστήσουν τα συστήματα που βασίζονται σε χειροποίητους κανόνες και την παραδοσιακή μηχανική μάθηση.
Στο~\cite{awan2018machine} οι συγγραφείς υποστηρίζουν ότι η βαθιά μάθηση είναι καλύτερη στην απεικόνιση πολύπλοκων μοτίβων κρυμμένων σε ιατρικά δεδομένα υψηλής διαστάσεων.
Οι Krittanawong et al.~\cite{krittanawong2017artificial} υποστηρίζουν ότι η αυξανόμενη διαθεσιμότητα αυτοματοποιημένων εργαλείων AI πραγματικού χρόνου στα EHRs, θα μειώσει την ανάγκη για συστήματα βαθμολόγησης όπως το σκορ Framingham.
Στο~\cite{krittanawong2017rise} οι συγγραφείς υποστηρίζουν ότι οι προγνωστικές αναλύσεις του AI και η εξατομικευμένη κλινική υποστήριξη για την αναγνώριση του ιατρικού κινδύνου, είναι ανώτερες από τις ανθρώπινες γνωσιακές ικανότητες.
Επιπλέον, το AI μπορεί να διευκολύνει την επικοινωνία μεταξύ ιατρών και ασθενών, μειώνοντας τους χρόνους επεξεργασίας και αυξάνοντας έτσι την ποιότητα της περίθαλψης των ασθενών.
Οι Loh et al.~\cite{loh2017deep} υποστηρίζουν ότι οι τεχνολογίες βαθιάς μάθησης και κινητής τηλεφωνίας θα επιταχύνουν τον πολλαπλασιασμό των υπηρεσιών υγειονομικής περίθαλψης σε εκείνους που βρίσκονται σε φτωχές περιοχές, γεγονός που με τη σειρά του οδηγεί σε περαιτέρω μείωση των ποσοστών ασθενειών.
Οι Mayer et al.~\cite{mayer2015big} δηλώνουν ότι τα big data υπόσχονται να αλλάξουν την καρδιολογία μέσω της αύξησης των δεδομένων που συλλέγονται, αλλά ο αντίκτυπος τους υπερβαίνει τη βελτίωση των υπαρχουσών μεθόδων όπως η αλλαγή του τρόπου με τον οποίο δημιουργούνται νέες ιδέες.

Η βαθιά μάθηση απαιτεί δεδομένα εκπαίδευσης μεγάλου όγκου για την επίτευξη αποτελεσμάτων υψηλής ποιότητας~\cite{krizhevsky2012imagenet}.
Αυτό είναι ιδιαίτερα δύσκολο με τα ιατρικά δεδομένα, λόγω του ότι η διαδικασία επισήμανσης των ιατρικών δεδομένων είναι δαπανηρή; απαιτεί χειρωνακτική εργασία από ιατρικούς εμπειρογνώμονες.
Επιπλέον, τα περισσότερα ιατρικά δεδομένα ανήκουν στις κανονικές περιπτώσεις αντί στις μη-κανονικές, καθιστώντας τα εξαιρετικά μη-ισορροπημένα.
Άλλες προκλήσεις της εφαρμογής της βαθιάς μάθησης στην ιατρική που έχει εντοπίσει η βιβλιογραφία είναι τα ζητήματα τυποποίησης/διαθεσιμότητας/διαστάσεων/όγκου/ποιότητας, δυσκολία στην απόκτηση των αντίστοιχων επισημάνσεων και θόρυβος στις επισημάνσεις~\cite{litjens2017survey, greenspan2016guest, miotto2017deep, slomka2017cardiac}.
Πιο συγκεκριμένα, στο~\cite{blair2017advanced} οι συγγραφείς σημειώνουν ότι οι εφαρμογές βαθιάς μάθησης στην ασθένεια μικρών αγγείων, έχουν αναπτυχθεί χρησιμοποιώντας μόνο λίγα αντιπροσωπευτικά σύνολα δεδομένων και πρέπει να αξιολογηθούν σε μεγάλα σύνολα δεδομένων πολλαπλών κέντρων.
Οι Kikuchi et al.~\cite{kikuchi2018future} αναφέρουν ότι σε σύγκριση με τα CT και MRI, οι απεικονιστικές της πυρηνικής καρδιολογίας έχουν περιορισμένο αριθμό εικόνων ανά ασθενή και απεικονίζεται μόνο συγκεκριμένος αριθμός οργάνων.
Ο Liebeskind~\cite{liebeskind2018artificial} αναφέρει ότι οι μέθοδοι μηχανικής μάθησης δοκιμάζονται σε επίλεκτα και ομοιογενή κλινικά δεδομένα, αλλά η γενικευσιμότητα θα προέκυπτε χρησιμοποιώντας ετερογενή και πολύπλοκα δεδομένα.
Το ισχαιμικό αγγειακό εγκεφαλικό επεισόδιο αναφέρεται ως παράδειγμα μιας ετερογενούς και σύνθετης ασθένειας, όπου η απόφραξη της μεσαίας εγκεφαλικής αρτηρίας μπορεί να οδηγήσει σε αποκλίνουσες μορφές απεικόνισης.
Στο~\cite{gomez2018new} οι συγγραφείς καταλήγουν στο συμπέρασμα ότι απαιτούνται πρόσθετα δεδομένα που επικυρώνουν αυτές τις εφαρμογές σε μη-ελεγχόμενες κλινικές ρυθμίσεις πολλαπλών κέντρων, πριν από την κλινική εφαρμογή τους.
Πρέπει επίσης να διερευνηθεί ο αντίκτυπος αυτών των εργαλείων στην λήψη αποφάσεων, χρήση πόρων, και κόστος.
Επιπλέον, η παρούσα βιβλιογραφία έδειξε ότι υπάρχει μια άνιση κατανομή των διαθέσιμων στο κοινό βάσεων δεδομένων ανάμεσα στις διάφορες απεικονιστικές τεχνικές στην καρδιολογία (π.χ. δεν υπάρχει διαθέσιμη δημόσια βάση δεδομένων για τα OCT σε αντίθεση με τα MRI).

Η προηγούμενη βιβλιογραφία αναφέρει ότι τα προβλήματα που σχετίζονται με τα δεδομένα μπορούν να λυθούν με τεχνικές επαύξησης δεδομένων, την ανοικτή συνεργασία μεταξύ των ερευνητικών οργανισμών και την αύξηση της χρηματοδότησης.
Οι Hengling et al.~\cite{henglin2017machine} υποστηρίζουν ότι θα χρειαστούν σημαντικές επενδύσεις για τη δημιουργία επισημασμένων βάσεων δεδομένων υψηλής ποιότητας, οι οποίες είναι απαραίτητες για την επιτυχία των επιβλεπώμενων μεθόδων βαθιάς μάθησης.
Στο~\cite{austin2016application} οι συγγραφείς υποστηρίζουν ότι η επιτυχία αυτού του τομέα εξαρτάται από τις τεχνολογικές εξελίξεις στην πληροφορική και την αρχιτεκτονική των υπολογιστών, καθώς και τη συνεργασία και την ανοικτή ανταλλαγή δεδομένων μεταξύ των γιατρών και άλλων ενδιαφερομένων.
Οι Lee et al.~\cite{lee2017deepb} καταλήγουν στο συμπέρασμα ότι απαιτείται διεθνής συνεργασία για την κατασκευή ενός υψηλής ποιότητας πολυτροπικού συνόλου δεδομένων για απεικόνιση εγκεφαλικών επεισοδίων.
Μια άλλη λύση για την καλύτερη αξιοποίηση των δεδομένων μεγάλου όγκου στην καρδιολογία είναι η εφαρμογή μη-επιβλεπώμενων μεθόδων μάθησης, οι οποίες δεν απαιτούν επισημάνσεις.
Η παρούσα βιβλιογραφική αναφορά έδειξε ότι η μη-επιβλεπώμενη μάθηση δεν χρησιμοποιείται ευρέως, καθώς η πλειονότητα των μεθόδων σε όλες τις απεικονιστικές τεχνικές είναι επιβλεπώμενες.

Όσον αφορά το πρόβλημα της έλλειψης ερμηνευσιμότητας όπως υποδεικνύεται από τον Hinton~\cite{hinton2018deep}, είναι γενικά ανέφικτο να ερμηνευτούν τα μη-γραμμικά χαρακτηριστικά των βαθιών δικτύων, επειδή το νόημά τους εξαρτάται από πολύπλοκες αλληλεπιδράσεις με μη-ερμηνεύσιμα χαρακτηριστικά από άλλα επίπεδα.
Επιπλέον, αυτά τα μοντέλα είναι στοχαστικά, που σημαίνει ότι κάθε φορά που ένα δίκτυο εκπαιδεύεται με τα ίδια δεδομένα αλλά με διαφορετικά αρχικά βάρη, μαθαίνει διαφορετικά χαρακτηριστικά.
Πιο συγκεκριμένα σε μια εκτενή ανασκόπηση~\cite{betancur2018deep} του αν επιλύεται το πρόβλημα της τμηματοποίησης LV/RV, οι συγγραφείς δηλώνουν ότι αν και η πτυχή ταξινόμησης του προβλήματος επιτυγχάνει σχεδόν τέλεια αποτελέσματα, η χρήση ενός `διαγνωστικού μαύρου κουτιού' δεν μπορεί να ενσωματωθεί στην κλινική πρακτική.
Οι Miotto et al.~\cite{miotto2017deep} αναφέρουν την ερμηνευσιμότητα ως μία από τις κύριες προκλήσεις που αντιμετωπίζει η κλινική εφαρμογή της βαθιάς μάθησης στην υγειονομική περίθαλψη.
Στο~\cite{lee2017deepb} οι συγγραφείς σημειώνουν ότι η ιδιότητα του μαύρου κουτιού που έχουν μερικές ΑΙ μέθοδοι, όπως η βαθιά μάθηση, είναι αντίθετη με την έννοια της τεκμηριωμένης ιατρικής και εγείρει νομικά και ηθικά ζητήματα στη χρήση τους στην κλινική πρακτική.
Αυτή η έλλειψη ερμηνευσιμότητας είναι ο κύριος λόγος για τον οποίο οι ιατρικοί εμπειρογνώμονες αντιστέκονται στη χρήση αυτών των μοντέλων και υπάρχουν επίσης νομικοί περιορισμοί όσον αφορά την ιατρική χρήση των μη-ερμηνευόμενων εφαρμογών~\cite{slomka2017cardiac}.
Από την άλλη πλευρά, κάθε μοντέλο μπορεί να τοποθετηθεί σε έναν άξονα~\cite{beam2018big} `ανθρώπου-μηχανής', συμπεριλαμβανομένων στατιστικών που οι ιατρικοί εμπειρογνώμονες βασίζονται στην καθημερινή λήψη κλινικών αποφάσεων.
Για παράδειγμα, οι ανθρώπινες αποφάσεις όπως η επιλογή των μεταβλητών που πρέπει να συμπεριληφθούν στο μοντέλο, η σχέση εξαρτημένων και ανεξάρτητων μεταβλητών και μεταβλητών μετασχηματισμών, μετακινούν τον αλγόριθμο προς τον ανθρώπινο άξονα αποφάσεων, καθιστώντας τον πιο ερμηνεύσιμο αλλά ταυτόχρονα περισσότερο πιο επιρρεπή σε σφάλματα.

Όσον αφορά την επίλυση του προβλήματος ερμηνευσιμότητας, όταν νέες μέθοδοι είναι απαραίτητες οι ερευνητές είναι προτιμητέο να δημιουργούν απλούστερες μεθόδους βαθιάς μάθησης (από-άκρο-σε-άκρο και μη-ensembles) για να αυξήσουν την πιθανότητα κλινικής εφαρμογής τους, ακόμα κι αν αυτό σημαίνει μειωμένη αναφερόμενη ακρίβεια.
Υπάρχουν επίσης επιχειρήματα κατά της δημιουργίας νέων μεθόδων, επικεντρώνοντας στην επικύρωση των ήδη υφιστάμενων.
Στο~\cite{damen2016prediction} οι συγγραφείς καταλήγουν στο συμπέρασμα ότι υπάρχει πληθώρα μοντέλων που προβλέπουν περιστατικά CVD στο γενικό πληθυσμό.
Η χρησιμότητα των περισσότερων μοντέλων είναι ασαφής λόγω σφαλμάτων στη μεθοδολογία και έλλειψης εξωτερικών μελετών επικύρωσης.
Αντί να αναπτυχθούν νέα μοντέλα πρόβλεψης κινδύνου CVD, η μελλοντική έρευνα πρέπει να επικεντρωθεί στην επικύρωση και σύγκριση υφιστάμενων μοντέλων και να διερευνήσει εάν μπορούν να βελτιωθούν περαιτέρω.

Μια δημοφιλής μέθοδος που χρησιμοποιείται για ερμηνεύσιμα μοντέλα είναι τα δίκτυα προσοχής~\cite{bahdanau2014neural}.
Τα δίκτυα προσοχής είχαν ως αρχική έμπνευση την ικανότητα εστίασης της ανθρώπινης όρασης σε ένα συγκεκριμένο σημείο με υψηλή ανάλυση και την αντίληψη του περιβάλλοντος με χαμηλή ανάλυση.
Έχουν χρησιμοποιηθεί από αρκετές δημοσιεύσεις στην καρδιολογία στην πρόβλεψη του ιατρικού ιστορικού~\cite{kim2017highrisk}, ταξινόμηση χτύπων ECG~\cite{schwab2017beat} και την πρόβλεψη CVD με χρήση Fundus~\cite{poplin2017predicting}.
Ένα άλλο απλούστερο εργαλείο για την ερμηνεία είναι οι χάρτες αξιοπιστίας (saliency maps)~\cite{simonyan2013deep} που χρησιμοποιούν την κλίση της εξόδου σε σχέση με την είσοδο, η οποία δείχνει διαισθητικά τις περιοχές που συνεισφέρουν περισσότερο στην έξοδο.

Εκτός από την επίλυση των προβλημάτων των δεδομένων και της ερμηνείας, οι ερευνητές στην καρδιολογία θα μπορούσαν να χρησιμοποιήσουν τις ήδη καθιερωμένες αρχιτεκτονικές βαθιάς μάθησης που δεν έχουν εφαρμοστεί ευρέως στην καρδιολογία, όπως τα δίκτυα καψουλών (Capsule Networks, CapsNets).
Τα CapsNet~\cite{sabour2017dynamic} είναι βαθιά νευρωνικά δίκτυα που χρειάζονται λιγότερα δεδομένα εκπαίδευσης από τα CNN και τα επίπεδά τους αποτυπώνουν τον `προσανατολισμό' των χαρακτηριστικών, καθιστώντας έτσι τις εσωτερικές τους λειτουργίες πιο ερμηνεύσιμες και πιο κοντά στον ανθρώπινο τρόπο αντίληψης.
Εντούτοις, ένα σημαντικό μειονέκτημα που έχουν, το οποίο τους περιορίζει από την επίτευξη ευρύτερης χρήσης, είναι το υψηλό υπολογιστικό κόστος σε σύγκριση με τα CNN λόγω του αλγορίθμου `δρομολόγησης με συμφωνία'.
Μεταξύ των πρόσφατων χρήσεών τους στην ιατρική περιλαμβάνονται η ταξινόμηση όγκων στον εγκέφαλο~\cite{afshar2018brain} και η ταξινόμηση του καρκίνου του μαστού~\cite{iesmantas2018convolutional}.
Τα CapsNet δεν έχουν χρησιμοποιηθεί ακόμη σε καρδιολογικά δεδομένα.

Μια άλλη λιγότερο χρησιμοποιούμενη αρχιτεκτονική βαθιάς μάθησης στην καρδιολογία είναι τα GANs~\cite{goodfellow2014generative}, τα οποία αποτελούνται από ένα δίκτυο γεννήτρια που δημιουργεί ψεύτικες εικόνες από θόρυβο και ένα δίκτυο διάκρισης που είναι υπεύθυνο για τη διαφοροποίηση μεταξύ πλαστών εικόνων από τη γεννήτρια και πραγματικών εικόνων.
Και τα δύο δίκτυα προσπαθούν να βελτιστοποιήσουν μια απώλεια σε ένα παιχνίδι με μηδενικό άθροισμα, με αποτέλεσμα μια γεννήτρια που παράγει ρεαλιστικές εικόνες.
Τα GAN έχουν χρησιμοποιηθεί μόνο για την προσομοίωση παθο-ρεαλιστικών εικόνων IVUS~\cite{tom2018simulating} και ο τομέας της καρδιολογίας έχει πολλά να κερδίσει από τη χρήση αυτού του είδους των μοντέλων, ειδικά λόγω της έλλειψης επισημασμένων δεδομένων υψηλής ποιότητας.

Οι ερευνητές θα μπορούσαν επίσης να χρησιμοποιήσουν CRF, τα οποία είναι γραφικά μοντέλα που συλλαμβάνουν πληροφορίες από γειτονικές περιοχές και είναι σε θέση να ενσωματώσουν στατιστικά στοιχεία υψηλότερης τάξης, τα οποία οι παραδοσιακές μέθοδοι βαθιάς μάθησης δεν είναι σε θέση να κάνουν.
Εκπαιδευμένα CRFs από κοινού με CNNs έχουν χρησιμοποιηθεί σε εκτίμηση του βάθους στην ενδοσκόπηση~\cite{mahmood2018deep} και την κατάτμηση του ήπατος στην CT~\cite{christ2016automatic}.
Υπάρχουν επίσης εφαρμογές καρδιολογίας που χρησιμοποίησαν CRF με βαθιά μάθηση για βελτίωση της τμηματοποίησης στα Fundus~\cite{zhou2017improving} και σε LV/RV~\cite{bai2017semi}.
Η πολυτροπική βαθιά μάθηση~\cite{ngiam2011multimodal} μπορεί επίσης να χρησιμοποιηθεί για τη βελτίωση των διαγνωστικών αποτελεσμάτων, π.χ. τη δυνατότητα συνδυασμού δεδομένων fMRI και ECG\@.
Ειδικές βάσεις δεδομένων πρέπει να δημιουργηθούν για να αυξηθεί η έρευνα στον τομέα αυτό, καθώς σύμφωνα με την τρέχουσα ανασκόπηση υπάρχουν μόνο τρεις βάσεις δεδομένων καρδιολογίας με πολυτροπικά δεδομένα.
Εκτός από τις προηγούμενες βάσεις δεδομένων, η MIMIC-III έχει επίσης χρησιμοποιηθεί για πολυτροπική βαθιά μάθηση από~\cite{purushotham2018benchmarking} για την πρόβλεψη ενδονοσοκομειακής, βραχυπρόθεσμης/μακροπρόθεσμης θνησιμότητας και πρόβλεψης του κώδικα ICD-9.

\clearpage
\bibliography{chapter3.bib}
\bibliographystyle{unsrt}
