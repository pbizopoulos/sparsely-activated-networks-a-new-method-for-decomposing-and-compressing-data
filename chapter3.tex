\chapter{Βαθιά μάθηση με δομημένα δεδομένα και σήματα}
\label{chapter3}
\graphicspath{{./images/deep-learning-in-cardiology/}}

\section{Εισαγωγή}
Οι καρδιαγγειακές παθήσεις (Cardiovascular Diseases, CVDs) είναι η κύρια αιτία θανάτου παγκοσμίως αντιπροσωπεύοντας το 30\% των θανάτων το 2014 στις Ηνωμένες Πολιτείες~\cite{benjamin2017heart}, το 45\% των θανάτων στην Ευρώπη και εκτιμάται ότι κοστίζουν 210 δισ. Ευρώ ετησίως μόνο για την Ευρωπαϊκή Ένωση~\cite{wilkins2017european}.

Στα επόμενα δύο κεφάλαια θα παρουσιάσουμε μια συστηματική βιβλιογραφική ανασκόπηση, που έγινε στο πλαίσιο της παρούσας διδακτορικής διατριβής, των εφαρμογών της βαθιάς μάθησης σε δομημένα δεδομένα, σήματα και απεικόνιση από την καρδιολογία, που σχετίζονται με καρδιακές δομές και αγγεία.
Η φράση αναζήτησης της βιβλιογραφίας είναι η συνδυασμός καθενός από τους όρους καρδιολογίας και βαθιάς μάθησης που αναφέρονται στα Ακρωνύμια, χρησιμοποιώντας το Google Scholar\footnote{\url{https://scholar.google.com}}, το Pubmed\footnote{\url{https://ncbi.nlm.nih.gov/pubmed/}} και το Scopus\footnote{\url{https://www.scopus.com/search/form.uri?=display=basic}}.
Έπειτα τα αποτελέσματα επιλέγονται έτσι ώστε να ταιριάζουν με τα κριτήρια επιλογής της ανασκόπησης τα οποία συνοψίζονται σε δύο κύριους άξονες: την αρχιτεκτονική του νευρωνικού δικτύου και ο τύπος δεδομένων που χρησιμοποιήθηκε για εκπαίδευση/επικύρωση/δοκιμή.
Οι αξιολογήσεις αναφέρονται για περιοχές που χρησιμοποίησαν ένα συνεκτικό σύνολο μετρήσεων με την ίδια μη-τροποποιημένη βάση δεδομένων και το ίδιο ερευνητικό ερώτημα.
Δημοσιεύσεις που δεν παρέχουν πληροφορίες σχετικά με την αρχιτεκτονική του νευρωνικού δικτύου ή δημοσιεύσεις που απλά αναπαράγουν μεθόδους προηγούμενων δημοσιεύσεων ή προκαταρκτικές δημοσιεύσεις αποκλείονται από αυτήν την ανασκόπηση.
Όταν αναφέρεται η λέξη \textit{πολλαπλά} στη στήλη `Αποτελέσματα', αυτά αναφέρονται στο κυρίως κείμενο όπου είναι κατάλληλο και ειδικά για τις αρχιτεκτονικές συνελικτικών νευρωνικών δικτύων, η χρήση του όρου \textit{επίπεδο} συνεπάγεται `συνελικτικό επίπεδο' για χάρη συντομίας.
Οι πίνακες~\ref{table:cardiologypublicdatabases1},~\ref{table:cardiologypublicdatabases2} και~\ref{table:cardiologypublicdatabases3} παρέχουν μια επισκόπηση των δημόσια διαθέσιμων βάσεων δεδομένων καρδιολογίας που έχουν χρησιμοποιηθεί για εκπαίδευση μοντέλων βαθιάς μάθησης.

\begin{sidewaystable}
	\centering
	\caption{Δημόσιες καρδιολογικές βάσεις δεδομένων, δομημένων δεδομένων και σημάτων}
	\label{table:cardiologypublicdatabases1}
	\begin{tabular}{l c r l}
		\toprule
		\thead{Βάση δεδομένων}                                & \thead{Ακρωνύμιο} & \thead{Ασθενείς}                                                               & \thead{Πρόβλημα}                                                      \\
		\midrule
		\multicolumn{4}{l}{\thead{Δομημένες βάσεις δεδομένων}}                                                                                                                                                                                                                                                           \\
		\midrule
		Medical Information Mart for Intensive Care III~\cite{johnson2016mimic}                                                              & MIMIC             & 38597                                                                          & 53423 νοσοκομειακές εισαγωγές για πρόβλεψη ICD-9     \\
		KNHANES-VI~\cite{kweon2014data}                                                                                                      & KNH               & 8108                                                                           & δημογραφικά, τεστ αίματος και τρόπος ζωής \\
		\midrule
		\multicolumn{4}{l}{\thead{Βάσεις δεδομένων σημάτων (όλα ECG εκτός του~\cite{karlen2013multiparameter})}}                                                                                                                                                                                                          \\
		\midrule
		IEEE-TBME PPG Respiratory Rate Benchmark Dataset~\cite{karlen2013multiparameter}                                                     & PPGDB             & 42                                                                             & εκτίμηση ρυθμού αναπνοής με χρήση PPG                                 \\
		Creighton University Ventricular Tachyarrhythmia~\cite{nolle1986crei}                                                                & CREI              & 35                                                                             & εντοπισμός κοιλιακής ταχυαρρυθμίας                                    \\
		MIT-BIH Atrial Fibrillation Database~\cite{moody1983new}                                                                             & AFDB              & 25                                                                             & πρόβλεψη HF                                                           \\
		BIH Deaconess Medical Center CHF Database~\cite{baim1986survival}                                                                    & CHFDB             & 15                                                                             & ταξινόμηση CHF                                                        \\
		St.Petersburg Institute of Cardiological Technics~\cite{goldberger2000physiobank}                                                    & INDB              & 32                                                                             & εντοπισμός QRS και ταξινόμηση χτύπων ECG                              \\
		Long-Term Atrial Fibrillation Database~\cite{petrutiu2007abrupt}                                                                     & LTAFDB            & 84                                                                             & εντοπισμός QRS και ταξινόμηση χτύπων ECG                              \\
		Long-Term ST Database~\cite{jager2003long}                                                                                           & LTSTDB            & 80                                                                             & ανίχνευση χτύπων ST και ταξινόμηση                                    \\
		MIT-BIH Arrhythmia Database~\cite{moody2001impact}                                                                                   & MITDB             & 47                                                                             & ανίχνευση αρρυθμιών                                                   \\
		MIT-BIH Noise Stress Test Database~\cite{moody1984noise}                                                                             & NSTDB             & 12                                                                             & χρήση για δοκιμή αντοχής θορύβου μοντέλων                             \\
		MIT-BIH Normal Sinus Rhythm Database~\cite{goldberger2000physiobank}                                                                 & NSRDB             & 18                                                                             & ανίχνευση αρρυθμιών                                                   \\
		MIT-BIH Normal Sinus Rhythm RR Interval Database~\cite{goldsmith1992comparison}                                                      & NSR2DB            & 54                                                                             & ταξινόμηση χτύπων ECG                                                 \\
		Fantasia Database~\cite{iyengar1996age}                                                                                              & FAN               & 40                                                                             & ταξινόμηση χτύπων ECG                                                 \\
		AF Classification short single lead ECG Physionet 2017~\cite{moody2001impact}                                                        & PHY17             & ---\footnote{Ο αριθμός των ασθενών δεν αναφέρεται.} & ταξινόμηση χτύπων ECG (12186 σήματα μονού καναλιού)                   \\
		Physionet 2016 Challenge~\cite{liu2016open}                                                                                          & PHY16             & 1297                                                                           & ταξινόμηση ήχων καρδιάς με χρήση PCG (3126 σήματα)                    \\
		Physikalisch-Technische Bundesanstalt ECG Database~\cite{bousseljot1995nutzung}                                                      & PTBDB             & 268                                                                            & διάγνωση καρδιακών ασθενειών                                          \\
		QT Database~\cite{laguna1997database}                                                                                                & QTDB              & 105                                                                            & ανίχνευση χτύπων QT και ταξινόμηση                                    \\
		MIT-BIH Supraventricular Arrhythmia Database~\cite{greenwald1990improved}                                                            & SVDB              & 78                                                                             & ανίχνευση υπερκοιλιακών αρρυθμιών                                     \\
		Non-Invasive Fetal ECG Physionet Challenge Dataset~\cite{silva2013noninvasive}                                                       & PHY13             & 447                                                                            & μέτρηση εμβρυϊκού HR, διαστήματος RR και QT                           \\
		DeepQ Arrhythmia Database~\cite{wu2017deepq} & DeepQ             & 299                                                                            & ταξινόμηση χτύπων ECG (897 σήματα)                                    \\
		\bottomrule
	\end{tabular}
\end{sidewaystable}

\section{Βαθιά μάθηση με δομημένα δεδομένα}
\label{sec3:structured}
Τα δομημένα δεδομένα περιλαμβάνουν κυρίως EHRs και συνήθως διατηρούνται σε σχεσιακές βάσεις δεδομένων.
Μια σύνοψη των εφαρμογών βαθιάς μάθησης που χρησιμοποιούν δομημένα δεδομένα παρουσιάζεται στον Πίνακα~\ref{table:structured}.

\begin{sidewaystable}
	\centering
	\caption{Εφαρμογές βαθιάς μάθησης με χρήση δομημένων δεδομένων}
	\label{table:structured}
	\begin{tabular}{l c l l}
		\toprule
		\thead{Αναφορά}                                    & \thead{Μέθοδος} & \thead{Εφαρμογή/Σημειώσεις\footnote{Σε παρένθεση οι βάσεις δεδομένων που χρησιμοποιήθηκαν.}}                        & \thead{Αποτέλεσμα\footnote{\label{structuredlabel}Υπάρχει μεγάλη μεταβλητότητα στην αναφορά αποτελεσμάτων. Όλα τα αποτελέσματα ειναι ακρίβειες εκτός από το~\cite{choi2016using} που αναφέρει AUC και το~\cite{hsiao2016deep} που είναι στατιστική μελέτη.}} \\
		\midrule
		Gopalswamy 2017~\cite{gopalswamy2017deep}           & LSTM            & πρόβλεψη BP και διάρκεια περίθαλψης με χρήση πολλαπλών βιοδεικτών (μη-δημόσια)                                      & 73.1\%                                                                                                                                                                                                                                                       \\
		Choi 2016~\cite{choi2016using}                      & GRU             & διάγνωση του HF με χρήση GRU και παράθυρου παρατήρησης (μη-δημόσια)                               & 0.883\textsuperscript{b}                                                                                                                                                                                                                                        \\
		Purushotham 2018~\cite{purushotham2018benchmarking} & FNN, GRU        & προβλήματα πρόβλεψης της MIMIC με χρήση ενός FNN και ενός μοντέλου GRU (MIMIC)                                      & \textit{πολλαπλά}                                                                                                                                                                                                                                            \\
		Kim 2017~\cite{kim2017highrisk}                     & GRU, CNN        & πρόβλεψη υψηλού ρίσκου αγγειακής ασθένειας με ένα Bi-GRU και ενός 1D CNN (μη-δημόσια) & \textit{πολλαπλά}                                                                                                                                                                                                                                            \\
		Hsiao 2016~\cite{hsiao2016deep}                     & AE              & ανάλυση ρίσκου CVD με AE και softmax σε ασθενείς, μετεωρολογικά δεδομένα (μη-δημόσια)          & \_\_\textsuperscript{b}                                                                                                                                                                                                                                              \\
		Kim 2017~\cite{kim2017statistics}                   & DBN             & πρόβλεψη καρδιαγγειακού ρίσκου με χρήση DBN (KNH)                                                                   & 83.9\%                                                                                                                                                                                                                                                       \\
		Huang 2018~\cite{huang2018regularized}              & SDAE            & πρόβλεψη ACS με SDAE με δύο περιορισμούς συστηματοποίησης και softmax (μη-δημόσια)          & 73.0\%                                                                                                                                                                                                                                                       \\
		\bottomrule
	\end{tabular}
\end{sidewaystable}

Τα RNNs έχουν χρησιμοποιηθεί για τη διάγνωση καρδιαγγειακών νοσημάτων με χρήση δομημένων δεδομένων.
Στο~\cite{gopalswamy2017deep} οι συγγραφείς προβλέπουν την πίεση του αίματος (Blood Pressure, BP) κατά τη διάρκεια χειρουργικής επέμβασης και μετά, χρησιμοποιώντας LSTM\@.
Πραγματοποίησαν πειράματα σε ένα σύνολο 12036 χειρουργικών επεμβάσεων που περιέχουν πληροφορίες για ενδοεγχειρητικά σήματα (θερμοκρασία σώματος, αναπνευστικός ρυθμός, καρδιακός ρυθμός, διαστολική BP (Diastolic Blood Pressure, DBP), συστολική BP (Systolic Blood Pressure, SBP), κλάσμα εισπνεόμενου $O_2$), επιτυγχάνοντας καλύτερα αποτελέσματα από τα KNN και SVM\@.
Οι Choi et al.~\cite{choi2016using} εκπαίδευσαν ένα GRU με διαχρονικά δεδομένα EHR, ανιχνεύοντας σχέσεις μεταξύ χρονικών συμβάντων (διάγνωση ασθενειών, εντολές φαρμάκων κ.λπ.), χρησιμοποιώντας ένα παράθυρο παρατήρησης.
Κάνουν διάγνωση καρδιακής ανεπάρκειας (Heart Failure, HF) με AUC 0.777 για παράθυρο 12 μηνών και 0.883 για παράθυρο 18 μηνών, καλύτερα από τα MLP, SVM και KNN\@.
Οι Purushotham et al.~\cite{purushotham2018benchmarking} σύγκριναν τον super-learner (σύνολο ρηχών αλγορίθμων μηχανικής μάθησης)~\cite{polley2010super} με FNN, RNN και ένα πολυτροπικό μοντέλο βαθιάς μάθησης που προτάθηκε από τους συγγραφείς στη βάση δεδομένων MIMIC\@.
Το προτεινόμενο πλαίσιο χρησιμοποιεί FNN και GRU για το χειρισμό των μη-χρονικών και χρονικών χαρακτηριστικών αντίστοιχα, μαθαίνοντας έτσι τις κοινές λανθάνουσες αναπαραστάσεις τους για πρόβλεψη.
Τα αποτελέσματα δείχνουν ότι οι μέθοδοι βαθιάς μάθησης υπερβαίνουν κατά πολύ τον super-learner στην πλειονότητα των προβλημάτων πρόβλεψης της MIMIC (πρόβλεψη θνησιμότητας εντός νοσοκομείου με AUC 0.873, πρόβλεψη βραχυπρόθεσμης θνησιμότητας με AUC 0.871, πρόβλεψη μακροπρόθεσμης θνησιμότητας με AUC 0.87 και πρόβλεψη κωδικού ICD-9 με AUC 0.777).
Οι Kim et al.~\cite{kim2017highrisk} δημιούργησαν δύο μοντέλα πρόγνωσης ιατρικού ιστορικού χρησιμοποιώντας δίκτυα προσοχής και τα αξιολόγησαν σε 50000 ασθενείς με υπέρταση.
Έδειξαν ότι η χρήση ενός GRU διπλής κατεύθυνσης παρέχει καλύτερη διακριτική ικανότητα από ένα αντίστοιχο συνελικτικό δίκτυο, το οποίο όμως έχει μικρότερο χρόνο εκπαίδευσης με ανταγωνιστική ακρίβεια.

Τα ΑΕ χρησιμοποιήθηκαν για τη διάγνωση καρδιαγγειακών νοσημάτων με δομημένα δεδομένα.
Οι Hsiao et al.~\cite{hsiao2016deep} εκπαίδευσαν ένα ΑΕ και ένα επίπεδο softmax για την ανάλυση κινδύνου τεσσάρων κατηγοριών καρδιαγγειακών παθήσεων.
Η είσοδος περιελάμβανε δημογραφικά στοιχεία, κωδικούς ICD-9 από αρχεία εξωτερικών ασθενών, ρύπους συγκέντρωσης και μετεωρολογικές παραμέτρους από περιβαλλοντικά αρχεία.
Οι Huang et al.~\cite{huang2018regularized} εκπαίδευσαν ένα SDAE χρησιμοποιώντας ένα σύνολο δεδομένων EHR 3464 ασθενών για να προβλέψουν Οξύ Στεφανιαίο Σύνδρομο (Acute Coronary Syndrome, ACS).
Το SDAE έχει δύο περιορισμούς συστηματοποίησης (regularization) που κάνουν τις ανακατασκευασμένες αναπαραστάσεις χαρακτηριστικών να περιέχουν περισσότερες πληροφορίες σχετικές με το επίπεδο κινδύνου, καταγράφοντας έτσι τα χαρακτηριστικά των ασθενών σε παρόμοια επίπεδα κινδύνου και διατηρώντας τις διακριτικές πληροφορίες σε διαφορετικά επίπεδα κινδύνου.
Στη συνέχεια, τοποθέτησαν ένα επίπεδο softmax, το οποίο προσαρμόζεται στο πρόβλημα της κλινικής πρόβλεψης κινδύνου.

Τα DBN έχουν επίσης χρησιμοποιηθεί σε συνδυασμό με δομημένα δεδομένα εκτός από τα RNNs και AEs.
Στο~\cite{kim2017statistics} οι συγγραφείς πρώτα εφήρμοσαν μια στατιστική μελέτη ενός συνόλου δεδομένων 4244 εγγραφών για την εύρεση μεταβλητών που σχετίζονται με καρδιαγγειακές παθήσεις (ηλικία, φύλο, χοληστερόλη, λιποπρωτεΐνη υψηλής πυκνότητας, SBP, DBP, κάπνισμα, διαβήτης).
Στη συνέχεια, ανέπτυξαν ένα μοντέλο DBN για την πρόβλεψη καρδιαγγειακών παθήσεων (υπέρταση, υπερλιπιδαιμία, μυοκαρδιακό έμφραγμα (Myocardial Infarction, MI), στηθάγχη).
Σύγκριναν το μοντέλο τους με Naive Bayes, λογιστική παλινδρόμηση, SVM, RF και ένα βασικό DBN πετυχαίνοντας καλύτερα αποτελέσματα.

Σύμφωνα με τη βιβλιογραφία, τα RNN χρησιμοποιούνται ευρέως σε δομημένα δεδομένα καρδιολογίας επειδή είναι σε θέση να βρουν χρονικά χαρακτηριστικά καλύτερα από άλλες μεθόδους βαθιάς/μηχανικής μάθησης.
Από την άλλη πλευρά, οι εφαρμογές σε αυτόν τον τομέα είναι σχετικά λίγες και αυτό συμβαίνει κυρίως επειδή υπάρχει μικρός αριθμός δημόσιων βάσεων δεδομένων, κάτι το οποίο εμποδίζει την περαιτέρω αξιολόγηση και σύγκριση διαφορετικών αρχιτεκτονικών.
Επιπλέον, οι δομημένες βάσεις δεδομένων λόγω σχεδιασμού τους περιέχουν λιγότερες πληροφορίες για έναν μεμονωμένο ασθενή και επικεντρώνονται περισσότερο σε ομάδες ασθενών, καθιστώντας αυτές πιο κατάλληλες για επιδημιολογικές μελέτες παρά για τον τομέα της καρδιολογίας.

\section{Βαθιά μάθηση με σήματα}
\label{sec3:signals}
Τα σήματα περιλαμβάνουν χρονοσειρές όπως ηλεκτροκαρδιογράφημα (Electrocardiogram, ECG), φωνοκαρδιογράφημα (Phonocardiogram, PCG), παλμομετρικά δεδομένα και δεδομένα από φορητά εξαρτήματα.
Ένας λόγος που η παραδοσιακή μηχανική μάθηση έχει δουλέψει αρκετά καλά σε αυτόν τον τομέα τα προηγούμενα χρόνια οφείλεται στη χρήση χειροποίητων και προσεκτικά σχεδιασμένων χαρακτηριστικών από εμπειρογνώμονες, όπως τα στατιστικά μέτρα από το ECG beats και το διάστημα RR~\cite{faziludeen2013ecg}.
Η βαθιά μάθηση μπορεί να βελτιώσει τα μοντέλα όταν οι επισημάνσεις των ειδικών είναι χαμηλής ποιότητας ή όταν είναι δύσκολο να δημιουργηθεί ένα μοντέλο με χρήση χειροποίητων χαρακτηριστικών.
Μια σύνοψη των εφαρμογών βαθιάς μάθησης που χρησιμοποιούν σήματα εμφανίζεται στους πίνακες~\ref{table:signals1},~\ref{table:signals2},~\ref{table:signals3} και~\ref{table:signals4}.

\begin{sidewaystable}
	\centering
	\caption{Εφαρμογές βαθιάς μάθησης για ανίχνευση αρρυθμιών με την MITDB}
	\label{table:signals1}
	\begin{tabular}{l c l l}
		\toprule
		\thead{Αναφορά}                                & \thead{Μέθοδος} & \thead{Εφαρμογή/Σημειώσεις\footnote{Σε παρένθεση οι βάσεις δεδομένων που χρησιμοποιήθηκαν.}}          & \thead{Ακρίβεια\footnote{Υπάρχει μεγάλη μεταβλητότητα στην αναφορά αποτελεσμάτων. Τα αποτελέσματα του~\cite{kiranyaz2016real} είναι για κοιλιακούς/υπερκοιλιακούς εκτοπικούς χτύπους, το~\cite{isin2017cardiac} είναι για τρεις τύπους αρρυθμίας, το~\cite{wu2016novel} είναι για πέντε τύπους αρρυθμίας.}} \\
		\midrule
		Zubair 2016~\cite{zubair2016automated}          & CNN             & μη-γραμμικός μετασχηματισμός για ανίχνευση R-peak και 1D CNN με μεταβλητό ρυθμό μάθησης           & 92.7\%                                                                                                                                                                                                                                                                                                                                                                                                                                                                                                                                                                                                                                                                                                                                                                                                                                                                           \\
		Li 2017~\cite{li2017classification}             & CNN             & WT για αποθορυβοποίηση και ανίχνευση των R-peak και 1D CNN δύο \textit{επιπέδων}                      & 97.5\%                                                                                                                                                                                                                                                                                                                                                                                                                                                                                                                                                                                                                                                                                                                                                                                                                                                                           \\
		Kiranyaz 2016~\cite{kiranyaz2016real}           & CNN             & CNN ειδικό για κάθε ασθενή με χρήση προσαρμόσιμων 1D συνελικτικών επιπέδων                            & 99\%,97.6\%\textsuperscript{b}                                                                                                                                                                                                                                                                                                                                                                                                                                                                                                                                                                                                                                                                                                                                                                                                                                              \\
		Isin 2017~\cite{isin2017cardiac}                & CNN             & φίλτρα αποθορυβοποίησης, Pan-Tomkins, AlexNet για χαρακτηριστικά και PCA για ταξινόμηση      & 92.0\%\textsuperscript{b}                                                                                                                                                                                                                                                                                                                                                                                                                                                                                                                                                                                                                                                                                                                                                                                                                                                   \\
		Luo 2017~\cite{luo2017patient}                  & SDAE            & φίλτρα αποθορυβοποίησης, ανίχνευση R-peak με χρήση παραγώγων, WT, SDAE και softmax                    & 97.5\%                                                                                                                                                                                                                                                                                                                                                                                                                                                                                                                                                                                                                                                                                                                                                                                                                                                                           \\
		Jiang 2017~\cite{jiang2017heartbeat}            & SDAE            & φίλτρα αποθορυβοποίησης, Pan-Tomkins, SDAE και FNN                                                    & 97.99\%                                                                                                                                                                                                                                                                                                                                                                                                                                                                                                                                                                                                                                                                                                                                                                                                                                                                          \\
		Yang 2017~\cite{yang2017novel}                  & SSAE            & κανονικοποίηση του ECG, SSAE                                                                          & 99.45\%                                                                                                                                                                                                                                                                                                                                                                                                                                                                                                                                                                                                                                                                                                                                                                                                                                                                          \\
		Wu 2016~\cite{wu2016novel}                      & DBN             & φίλτρα αποθορυβοποίησης, ecgpuwave, δύο τύποι RBMs                                                    & 99.5\%\textsuperscript{b}                                                                                                                                                                                                                                                                                                                                                                                                                                                                                                                                                                                                                                                                                                                                                                                                                                                   \\
		\bottomrule
	\end{tabular}
\end{sidewaystable}

\subsection{Ηλεκτροκαρδιογράφημα}
Το ECG είναι η μέθοδος μέτρησης των ηλεκτρικών δυναμικών της καρδιάς για τη διάγνωση καρδιακών προβλημάτων~\cite{badnjevic2017inspection}.
Είναι μη-επεμβατική, εύκολη στην απόκτηση και παρέχει χρήσιμες πληροφορίες για τη διάγνωση ασθενειών.
Έχει χρησιμοποιηθεί κυρίως για την ανίχνευση αρρυθμιών χρησιμοποιώντας τον μεγάλο αριθμό δημόσιων βάσεων δεδομένων ECG όπως φαίνεται στον Πίνακα~\ref{table:cardiologypublicdatabases1}.

\subsubsection{Ανίχνευση αρρυθμιών με την MITDB}
Τα CNN έχουν χρησιμοποιηθεί για την ανίχνευση αρρυθμιών με την MITDB\@.
Οι Zubair et al.~\cite{zubair2016automated} ανίχνευσαν τις κορυφές R χρησιμοποιώντας ένα μη-γραμμικό μετασχηματισμό και τμηματοποιούν χτύπους γύρω από αυτό.
Στη συνέχεια, χρησιμοποίησαν τα τμήματα για να εκπαιδεύσουν ένα 1D CNN τριών \textit{επιπέδων} με μεταβλητό ρυθμό μάθησης, βασισμένο στο μέσο τετραγωνικό σφάλμα πετυχαίνοντας καλύτερα αποτελέσματα από τις προηγούμενες καλύτερες μεθόδους.
Οι Li et al.~\cite{li2017classification} χρησιμοποίησαν Μετασχηματισμό Κυματιδίων (Wavelet Transform, WT) για την απομάκρυνση του θορύβου υψηλής συχνότητας και της μετατόπισης της γραμμής βάσης και διορθογωνικού spline για την ανίχνευση των κορυφών R.
Έπειτα, δημιούργησαν τμήματα γύρω από την κορυφή R τα οποία τροφοδότησαν σε ένα 1D CNN δυο \textit{επιπέδων}.
Στο άρθρο τους οι Kiranyaz et al.~\cite{kiranyaz2016real} εκπαίδευσαν CNNs που μπορούν να χρησιμοποιηθούν για την ταξινόμηση δεδομένων ECG μακράς διάρκειας και για την παρακολούθηση ECG σε πραγματικό χρόνο, ως μέρος συστήματος έγκαιρης προειδοποίησης σε φορητή συσκευή.
Το CNN αποτελούνταν από τρία προσαρμόσιμα 1D συνελικτικά \textit{επίπεδα}.
Επιτεύχθηκε 99\% και 97.6\% για την ταξινόμηση κοιλιακών και υπερκοιλιακών εκτοπικών παλμών αντίστοιχα.
Στο~\cite{isin2017cardiac} οι συγγραφείς χρησιμοποίησαν αφαίρεση μέσου όρου για απομάκρυνση της βάσης αναφοράς, κινητό φίλτρο μέσου όρου για απομάκρυνση των υψηλών συχνοτήτων, διαφορικό φίλτρο για απομάκρυνση της γραμμής βάσης και φίλτρο χτένας (comb) για την απομάκρυνση του θορύβου γραμμής ισχύος.
Ανίχνευσαν τα QRS με τον αλγόριθμο Pan-Tompkins~\cite{pan1985real}, εξήγαγαν τμήματα χρησιμοποιώντας δείγματα μετά την κορυφή R τα οποία μετέτρεψαν σε $256\times 256\times 3$ δυαδικές εικόνες.
Οι εικόνες έπειτα τροφοδοτήθηκαν σε ένα εξαγωγέα χαρακτηριστικών AlexNet προεκπαιδευμένο στην βάση δεδομένων ImageNet και έπειτα στην Ανάλυση Κύριων Συνιστωσών (Principal Component Analysis, PCA).
Πέτυχαν υψηλή ακρίβεια στην ταξινόμηση τριών τύπων αρρυθμιών της MITDB\@.

Τα ΑΕ έχουν επίσης χρησιμοποιηθεί για ανίχνευση αρρυθμιών με MITDB\@.
Στο άρθρο τους οι Luo et al.~\cite{luo2017patient} χρησιμοποίησαν αξιολόγηση ποιότητας για την αφαίρεση των καρδιακών παλμών χαμηλής ποιότητας, δύο φίλτρα μέσης τιμής για την αφαίρεση του θορύβου της γραμμής ισχύος, του θορύβου υψηλής συχνότητας και της μετατόπισης της γραμμής βάσης.
Στη συνέχεια, χρησιμοποίησαν έναν διαφορικό αλγόριθμο για να ανιχνεύσουν τις κορυφές R και τα χρονικά παράθυρα για την κατάτμηση κάθε καρδιακού παλμού.
Χρησιμοποίησαν επίσης WT για τον υπολογισμό του φάσματος κάθε καρδιακού παλμού και ενός SDAE για την εξαγωγή χαρακτηριστικών από το φάσμα.
Έπειτα, δημιούργησαν έναν ταξινομητή τεσσάρων αρρυθμιών από τον κωδικοποιητή του SDAE και ένα softmax, επιτυγχάνοντας ακρίβεια 97.5\%.
Στο~\cite{jiang2017heartbeat} οι συγγραφείς αποθορυβοποίησαν τα σήματα με ένα χαμηλοπερατό, ένα ζωνοπερατό και ένα φίλτρο διάμεσου.
Εντόπισαν κορυφές R χρησιμοποιώντας τον αλγόριθμο Pan-Tomkins και κατάτμησαν/ανακατασκεύασαν τους καρδιακούς παλμούς.
Χαρακτηριστικά εξήχθησαν από το σήμα της καρδιάς χρησιμοποιώντας ένα SDAE και χρησιμοποιήθηκε ένα FNN για την ταξινόμηση καρδιακών παλμών από 16 τύπους αρρυθμίας.
Παρατηρήθηκε συγκρίσιμη απόδοση σε σχέση με προηγούμενες μεθόδους βασισμένες στην χειροκίνητη εξαγωγή χαρακτηριστικών.
Οι Yang et al.~\cite{yang2017novel} κανονικοποίησαν το ECG και στη συνέχεια το τροφοδότησαν σε ένα Στοιβαγμένο Αραιό AE (Stacked Sparse Autoencoder, SSAE).
Ταξινομούν έξι τύπους αρρυθμιών επιτυγχάνοντας ακρίβεια 99.5\% ενώ ταυτόχρονα αποδεικνύουν την ανθεκτικότητα του έναντι στο θόρυβο με χρήση τεχνητά προστιθέμενου θορύβου.

Τα DBN έχουν επίσης χρησιμοποιηθεί για αυτό το πρόβλημα εκτός από CNN και AE\@.
Οι Wu et al.~\cite{wu2016novel} χρησιμοποίησαν φίλτρα διάμεσου για την απομάκρυνση της γραμμής βάσης, ένα χαμηλοπερατό φίλτρο για την απομάκρυνση του θορύβου γραμμής ισχύος και του θορύβου υψηλής συχνότητας.
Εντόπισαν κορυφές R χρησιμοποιώντας το λογισμικό ecgpuwave από την Physionet και κατάτμησαν τους ECG χτύπους.
Δύο τύποι RBMs, εκπαιδεύτηκαν για την εξαγωγή χαρακτηριστικών από το ECG για ανίχνευση αρρυθμιών.
Επιτεύχθηκε ακρίβεια 99.5\% σε πέντε κατηγορίες της MITDB\@.

\subsubsection{Ανίχνευση αρρυθμιών με άλλες βάσεις δεδομένων}
Τα CNN χρησιμοποιήθηκαν για ανίχνευση αρρυθμιών χρησιμοποιώντας άλλες βάσεις δεδομένων εκτός της MITDB\@.
Στο~\cite{wu2018personalizing} οι συγγραφείς δημιούργησαν ένα CNN δύο \textit{επιπέδων} χρησιμοποιώντας την DeepQ και την MITDB για να ταξινομήσουν τέσσερις τύπους αρρυθμιών.
Τα σήματα υπόκεινται σε έντονη προεπεξεργασία με φίλτρα απομάκρυνσης (μεσαία, υψηλή, και χαμηλή διέλευση, αφαίρεση ακραίων τιμών) και κατατμήσονται σε 0.6 δευτερόλεπτα γύρω από την κορυφή R.
Στη συνέχεια, τροφοδοτούνται στο CNN μαζί με το διάστημα RR για εκπαίδευση.
Οι συγγραφείς χρησιμοποιούν επίσης μια ενεργή μέθοδο μάθησης για την επίτευξη εξατομικευμένων αποτελεσμάτων και βελτιωμένης ακρίβειας, επιτυγχάνοντας υψηλή ευαισθησία και θετική προβλεψιμότητα και στα δύο σύνολα δεδομένων.
Οι Hannun et al.~\cite{hannun2019cardiologist} δημιούργησαν ένα σύνολο δεδομένων με ECG από φορητές συσκευές το οποίο περιέχει τον μεγαλύτερο αριθμό ασθενών (30000) σε σχέση με προηγούμενες βάσεις δεδομένων και το χρησιμοποίησαν για να εκπαιδεύσουν ένα Residual-CNN με 34 \textit{επίπεδα}.
Το μοντέλο τους ανιχνεύει ένα ευρύ φάσμα αρρυθμιών συνολικά 14 κατηγορίες, ξεπερνώντας τον μέσο καρδιολόγο σε ακρίβεια.
Στο άρθρο τους, οι Acharya et al.~\cite{acharya2017automateda} εκπαίδευσαν ένα CNN τεσσάρων \textit{επιπέδων} στην AFDB, MITDB και CREI, για την ταξινόμηση μεταξύ φυσιολογικού, AF, κολπικού πτερυγισμού και κοιλιακής μαρμαρυγής.
Χωρίς την ανίχνευση του QRS, πέτυχαν συγκρίσιμες επιδόσεις με προηγούμενες μεθόδους που βασίστηκαν στην ανίχνευση κορυφών R και στην χειροκίνητη εξαγωγή χαρακτηριστικών.
Οι ίδιοι συγγραφείς επίσης εκπαίδευσαν την προηγούμενη CNN αρχιτεκτονική για τον εντοπισμό των απινιδωτικών και μη-απινιδωτικών κοιλιακών αρρυθμιών~\cite{acharya2018automated}, εντοπισμό CAD ασθενών με τις FAN και INDB~\cite{acharya2017automatedb}, ταξινόμηση CHF με τις CHFDB, NSTDB, FAN~\cite{acharya2018deep} και την δοκιμασία της αντοχής τους στο θόρυβο με αποθορυβοποίηση WT~\cite{acharya2017application}.

\begin{sidewaystable}
	\centering
	\caption{Εφαρμογές βαθιάς μάθησης με χρήση ECG για ανίχνευση αρρυθμιών και AF}
	\label{table:signals2}
	\begin{tabular}{l c l l}
		\toprule
		\thead{Αναφορά}                                & \thead{Μέθοδος} & \thead{Εφαρμογή/Σημειώσεις\footnote{Σε παρένθεση οι βάσεις δεδομένων που χρησιμοποιήθηκαν.}}          & \thead{Ακρίβεια\footnote{Υπάρχει μεγάλη μεταβλητότητα στην αναφορά αποτελεσμάτων. Τα αποτελέσματα του~\cite{hannun2019cardiologist} είναι precision, το~\cite{xiong2015denoising} αναφέρει SNR και πολλαπλά αποτελέσματα ανάλογα με τον προστιθέμενο θόρυβο, το αποτέλεσμα του~\cite{taji2017false} αφορά μελέτη αντοχής θορύβου.}} \\
		\midrule
		\multicolumn{4}{l}{\thead{Ανίχνευση αρρυθμιών}}                                                                                                                                                                                                                                                                                                                                                                                                                                                                                                                                                                                                                                                                                                                                                                                                                                                                                                                                                                                                             \\
		\midrule
		Wu 2018~\cite{wu2018personalizing}              & CNN             & ενεργή μάθηση και CNN δύο \textit{επιπέδων} με ECG και διαστήματα RR (MITDB, DeepQ) & \textit{πολλαπλά}                                                                                                                                                                                                                                                                                                                                                                                                                                                                                                                                                                                                                                                                                                                                                                                                                                                                \\
		Hannun 2019~\cite{hannun2019cardiologist} & CNN             & CNN 34-\textit{επιπέδων} (μη-δημόσια)                                                                 & 80\%\textsuperscript{b}                                                                                                                                                                                                                                                                                                                                                                                                                                                                                                                                                                                                                                                                                                                                                                                                                                                     \\
		Acharya 2017~\cite{acharya2017automateda}       & CNN             & CNN τεσσάρων \textit{επιπέδων} (AFDB, MITDB, Creighton)                                               & 92.5\%                                                                                                                                                                                                                                                                                                                                                                                                                                                                                                                                                                                                                                                                                                                                                                                                                                                                           \\
		Schwab 2017~\cite{schwab2017beat}               & RNN             & ensemble από RNNs με μηχανισμό προσοχής (PHY17)                                                       & 79\%                                                                                                                                                                                                                                                                                                                                                                                                                                                                                                                                                                                                                                                                                                                                                                                                                                                                             \\
		\midrule
		\multicolumn{4}{l}{\thead{Ανίχνευση AF}}                                                                                                                                                                                                                                                                                                                                                                                                                                                                                                                                                                                                                                                                                                                                                                                                                                                                                                                                                                                                                    \\
		\midrule
		Yao 2017~\cite{yao2017atrial}                   & CNN             & πολυ-κλιμακωτό CNN (AFDB, LTAFDB, μη-δημόσια)                                                         & 98.18\%                                                                                                                                                                                                                                                                                                                                                                                                                                                                                                                                                                                                                                                                                                                                                                                                                                                                          \\
		Xia 2018~\cite{xia2018detecting}                & CNN             & CNN με φασματογράφημα από STFT ή στατικού WT (AFDB)                                                   & 98.29\%                                                                                                                                                                                                                                                                                                                                                                                                                                                                                                                                                                                                                                                                                                                                                                                                                                                                          \\
		Andersen 2018~\cite{andersen2018deep}           & CNN, LSTM       & διαστήματα RR με CNN-LSTM (MITDB, AFDB, NSRDB)                                                        & 87.40\%                                                                                                                                                                                                                                                                                                                                                                                                                                                                                                                                                                                                                                                                                                                                                                                                                                                                          \\
		Xiong 2015~\cite{xiong2015denoising}            & AE              & κατωφλίωση προσαρμοσμένης κλίμακας WT και AE αποθορυβοποίησης (MITDB, NSTDB)                          & $\sim$18.7\textsuperscript{b}                                                                                                                                                                                                                                                                                                                                                                                                                                                                                                                                                                                                                                                                                                                                                                                                                                               \\
		Taji 2017~\cite{taji2017false}                  & DBN             & μείωση false alarm κατά τη διάρκεια ανίχνευσης AF σε θορυβώδη ECG σήματα (AFDB, NSTDB)                & 87\%\textsuperscript{b}                                                                                                                                                                                                                                                                                                                                                                                                                                                                                                                                                                                                                                                                                                                                                                                                                                                     \\
		\bottomrule
	\end{tabular}
\end{sidewaystable}

Μια εφαρμογή των RNN σε αυτή την περιοχή είναι από τους Schwab et al.~\cite{schwab2017beat} που δημιούργησαν ένα ensemble από RNN που διακρίνει μεταξύ φυσιολογικών φλεβοκομβικών ρυθμών, AF, άλλων τύπων αρρυθμίας και θορυβώδους σήματος.
Εισήγαγαν μια μορφοποίηση του προβλήματος κατά την οποία τμηματοποιούν το ECG σε καρδιακούς παλμούς για να μειώσουν τον αριθμό των χρονικών βημάτων ανά ακολουθία.
Επέκτειναν επίσης τα RNNs με έναν μηχανισμό προσοχής που τους επιτρέπει να εκτιμήσουν σε ποιους καρδιακούς παλμούς επικεντρώνεται το RNN για να λάβει τις αποφάσεις του και να επιτύχει συγκρίσιμη αποτελεσματικότητα με άλλες μεθόδους χρησιμοποιώντας λιγότερες παραμέτρους.

\subsubsection{Ανίχνευση AF}
Τα CNN χρησιμοποιήθηκαν για την ανίχνευση AF\@.
Οι Yao et al.~\cite{yao2017atrial} εξήγαγαν την ακολουθία του στιγμιαίου καρδιακού ρυθμού, η οποία τροφοδοτείται σε ένα CNN πολλαπλής κλίμακας που εξάγει το αποτέλεσμα ανίχνευσης AF, επιτυγχάνοντας καλύτερα αποτελέσματα από τις προηγούμενες μεθόδους όσον αφορά την ακρίβεια.
Οι Xia et al.~\cite{xia2018detecting} σύγκριναν δύο CNN, με τρία και δύο \textit{επίπεδα}, τα οποία τροφοδοτήθηκαν με φάσματα σημάτων από την AFDB χρησιμοποιώντας βραχυπρόθεσμο μετασχηματισμό Fourier (Short-Time Fourier Transform, STFT) και στατικό WT αντίστοιχα.
Τα πειράματά τους κατέληξαν στο συμπέρασμα ότι η χρήση του στατικού WT επιτυγχάνει μια ελαφρώς καλύτερη ακρίβεια για αυτό το πρόβλημα.

Εκτός από τα CNN έχουν χρησιμοποιηθεί και άλλες αρχιτεκτονικές για την ανίχνευση AF\@.
Οι Andersen et al.~\cite{andersen2018deep} μετέτρεψαν τα σήματα ECG από την AFDB σε διαστήματα RR για να τα ταξινομήσουν για ανίχνευση AF\@.
Στη συνέχεια, κατάτμησαν τα διαστήματα RR σε 30 δείγματα το καθένα και τα τροφοδότησαν σε ένα δίκτυο με δύο \textit{επίπεδα} ακολουθούμενα από ένα επίπεδο συγκέντρωσης (pooling) και ένα επίπεδο LSTM με 100 νευρώνες.
Η μέθοδος επικυρώθηκε στην MITDB και την NSRDB επιτυγχάνοντας ακρίβεια που δείχνει ότι μπορεί να γενικεύσει.
Στο~\cite{xiong2015denoising} οι συγγραφείς πρόσθεσαν σήματα θορύβου από την NSTDB στην MITDB και στη συνέχεια χρησιμοποίησαν κατωφλίωση προσαρμοσμένης κλίμακας WT, για να απομακρύνουν το μεγαλύτερο μέρος του θορύβου και ένα SDAE για να αφαιρεθεί ο υπολειπόμενος θόρυβος.
Τα πειράματά τους έδειξαν ότι η αύξηση του αριθμού των δεδομένων εκπαίδευσης σε 1000, αυξάνει δραματικά το λόγο σήματος προς θόρυβο μετά την αποθορυβοποίηση.
Οι Taji et al.~\cite{taji2017false} εκπαίδευσαν ένα DBN για να ταξινομήσουν αποδεκτά από μη-αποδεκτά τμήματα ECG, έτσι ώστε να μειώσουν το ποσοστό ψευδούς συναγερμού που προκαλείται από κακής ποιότητας ECG κατά την ανίχνευση AF\@.
Οκτώ διαφορετικά επίπεδα ποιότητας ECG παρέχονται, προσθέτοντας στο ECG θόρυβο κίνησης από την NSTDB\@.
Με SNR $-20dB$ στο ECG, η μέθοδος τους πέτυχε αύξηση 22\% της ακρίβειας σε σύγκριση με ένα μοντέλο βάσης αναφοράς.

\subsubsection{Άλλες εφαρμογές με δημόσιες βάσεις δεδομένων}
Ταξινόμηση χτύπων του ECG πραγματοποιήθηκε επίσης από μια σειρά μελετών που χρησιμοποίησαν δημόσιες βάσεις δεδομένων.
Στο~\cite{xiao2018monitoring} οι συγγραφείς μετεκπαίδευσαν ένα Inception-v3 προεκπαιδευμένο στην ImageNet, χρησιμοποιώντας σήματα από την LTSTDB για την ταξινόμηση συμβάντων ST\@.
Τα δείγματα εκπαίδευσης ήταν πάνω από 500000 τμήματα ST και μη-ST ECG σημάτων με διάρκεια δέκα δευτερολέπτων που έπειτα μετατράπηκαν σε εικόνες.
Επιτυγχάνουν συγκρίσιμες επιδόσεις με προηγούμενες πολύπλοκες μεθόδους βασισμένες σε χειροκίνητο ορισμό κανόνων.
Οι Rahhal et al.~\cite{al2016deep} εκπαίδευσαν ένα SDAEs με περιορισμό αραιότητας και ένα softmax για την ταξινόμηση των ECG χτύπων.
Σε κάθε επανάληψη, ο εμπειρογνώμονας επισημάνει τους πιο αβέβαιους χτύπους ECG στο σετ δοκιμών, τα οποία στη συνέχεια χρησιμοποιούνται για εκπαίδευση, ενώ η έξοδος του δικτύου εκχωρεί τα μέτρα εμπιστοσύνης σε κάθε χτύπο.
Πειράματα που εκτελούνται στις MITDB, INDB, SVDB δείχνουν την αξιοπιστία και την υπολογιστική αποτελεσματικότητα της μεθόδου.
Στο~\cite{abrishami2018p} οι συγγραφείς εκπαίδευσαν τρεις ξεχωριστές αρχιτεκτονικές για να προσδιορίσουν τα κύματα P-QRS-T στο ECG με την QTDB\@.
Σύγκριναν ένα FNN δύο \textit{επιπέδων}, ένα CNN δύο \textit{επιπέδων} και ένα CNN δύο \textit{επιπέδων} με dropout, με το δεύτερο να πετυχαίνει τα καλύτερα αποτελέσματα.

\begin{sidewaystable}
	\centering
	\caption{Εφαρμογές βαθιάς μάθησης με χρήση ECG σε άλλες εφαρμογές}
	\label{table:signals3}
	\begin{tabular}{l c l l}
		\toprule
		\thead{Αναφορά}                       & \thead{Μέθοδος} & \thead{Εφαρμογή/Σημειώσεις\footnote{Σε παρένθεση οι βάσεις δεδομένων που χρησιμοποιήθηκαν.}} & \thead{Ακρίβεια\footnote{Υπάρχει μεγάλη μεταβλητότητα στην αναφορά αποτελεσμάτων. Τα αποτελέσματα του~\cite{xiao2018monitoring} αναφέρει AUC, το~\cite{al2016deep} αναφέρει πολλαπλές ακρίβειες για υπερκοιλιακούς/κοιλιακούς εκτοπικούς χτύπους, το~\cite{wu2016myocardial} αναφέρει ευαισθησία και εξειδίκευση (specificity), το~\cite{hwang2018deep} αναφέρει αποτελέσματα για δύο περιπτώσεις.}} \\
		\midrule
		\multicolumn{4}{l}{\thead{Άλλες εφαρμογές}}                                                                                                                                                                                                                                                                                                                                                                                                                                                                                                                                                                                                                                                                                                                                                                                                                                                                                                                                                                                               \\
		\midrule
		Xiao 2018~\cite{xiao2018monitoring}    & CNN             & ταξινόμηση γεγονότων ST από ECG με μεταφοράς μάθησης στο Inception-v3 (LTSTDB)               & 0.867\textsuperscript{b}                                                                                                                                                                                                                                                                                                                                                                                                                                                                                                                                                                                                                                                                                                                                                                                                                                                    \\
		Rahhal 2016~\cite{al2016deep}          & SDAE            & SDAE με περιορισμό αραιότητας και softmax (MITDB, INDB, SVDB)                                 & \textgreater{99\%}\textsuperscript{b}                                                                                                                                                                                                                                                                                                                                                                                                                                                                                                                                                                                                                                                                                                                                                                                                                                       \\
		Abrishami 2018~\cite{abrishami2018p}   & Multiple        & σύγκριναν ένα FNN, CNN και CNN με dropout για εντοπισμό κυμάτων ECG (QTDB)                & 96.2\%                                                                                                                                                                                                                                                                                                                                                                                                                                                                                                                                                                                                                                                                                                                                                                                                                                                                           \\
		Wu 2016~\cite{wu2016myocardial}        & SAE             & εντοπισμός και ταξινόμηση MI με SAE και πολυ-κλιμακωτό διακριτό WT (PTBDB)                       & $\sim$99\%\textsuperscript{b}                                                                                                                                                                                                                                                                                                                                                                                                                                                                                                                                                                                                                                                                                                                                                                                                                                               \\
		Reasat 2017~\cite{reasat2017detection} & Inception       & ανίχνευση MI με Inception μπλοκ για κάθε κανάλι ECG (PTBDB)                                    & 84.54\%                                                                                                                                                                                                                                                                                                                                                                                                                                                                                                                                                                                                                                                                                                                                                                                                                                                                          \\
		Zhong 2018~\cite{zhong2018deep}        & CNN             & CNN τριών \textit{επιπέδων} για την ταξινόμηση εμβρυϊκών ECG (PHY13)                         & 77.85\%                                                                                                                                                                                                                                                                                                                                                                                                                                                                                                                                                                                                                                                                                                                                                                                                                                                                          \\
		\midrule
		\multicolumn{4}{l}{\thead{Άλλες εφαρμογές (μη-δημόσιες βάσεις)}}                                                                                                                                                                                                                                                                                                                                                                                                                                                                                                                                                                                                                                                                                                                                                                                                                                                                                                                                                                          \\
		\midrule
		Ripoll 2016~\cite{ripoll2016ecg}       & RBM             & ανίχνευση μη-κανονικών ECG με προεκπαιδευμένα RBMs                                           & 85.52\%                                                                                                                                                                                                                                                                                                                                                                                                                                                                                                                                                                                                                                                                                                                                                                                                                                                                          \\
		Jin 2017~\cite{jin2017classification}  & CNN             & ανίχνευση μη-κανονικών ECG με lead-CNN και κανόνα συμπερασμού                                & 86.22\%                                                                                                                                                                                                                                                                                                                                                                                                                                                                                                                                                                                                                                                                                                                                                                                                                                                                          \\
		Liu 2018~\cite{liu2018detecting}       & Multiple        & συνέκριναν το Inception και ένα 1D CNN για πρόωρη κοιλιακή συστολή με ECG                    & 88.5\%                                                                                                                                                                                                                                                                                                                                                                                                                                                                                                                                                                                                                                                                                                                                                                                                                                                                           \\
		Hwang 2018~\cite{hwang2018deep}        & CNN, RNN        & ανίχνευση στρες με ένα CNN  ενός \textit{επιπέδου} με dropout και δύο RNNs                   & 87.39\%\textsuperscript{b}                                                                                                                                                                                                                                                                                                                                                                                                                                                                                                                                                                                                                                                                                                                                                                                                                                                  \\
		\bottomrule
	\end{tabular}
\end{sidewaystable}

Το ECG έχει επίσης χρησιμοποιηθεί για τον εντοπισμό και την ταξινόμηση του ΜΙ.
Στο άρθρο τους οι Wu et al.~\cite{wu2016myocardial} εντόπισαν και ταξινόμησαν MI στην PTBDB\@.
Χρησιμοποίησαν διακριτό WT πολλαπλών κλιμάκων, για να διευκολύνουν την εξαγωγή χαρακτηριστικών για ΜΙ σε συγκεκριμένες αναλυτικότητες συχνοτήτων και ένα επίπεδο παλινδρόμησης softmax για να δημιουργήσουν έναν ταξινομητή πολλαπλών κατηγοριών.
Τα πειράματα επικύρωσης δείχνουν ότι η μέθοδος τους απέδωσε καλύτερα από τις προηγούμενες μεθόδους, όσον αφορά την ευαισθησία και την εξειδίκευση.
Η PTBDB χρησιμοποιήθηκε επίσης από τους Reasat et al.~\cite{reasat2017detection} για να εκπαιδεύσουν ένα μοντέλο CNN βασισμένο στο inception.
Κάθε ηλεκτρόδιο του ECG τροφοδοτείται σε ένα inception μπλοκ, ακολουθούμενο από επίπεδα συγκόλλησης (concatenation), παγκόσμιας μέσης συγκέντρωσης (global average pooling) και ένα softmax.
Οι συγγραφείς συνέκριναν τη μέθοδο τους με μια προηγούμενη σύγχρονη μέθοδο που χρησιμοποιεί SWT, πετυχαίνοντας καλύτερα αποτελέσματα.

Συμπλέγματα εμβρυϊκού QRS ταυτοποιήθηκαν με ένα CNN τριών \textit{στρωμάτων} με dropout από τους Zhong et al.~\cite{zhong2018deep} με την PHY13.
Αρχικά, τα σήματα κακής ποιότητας απορρίπτονται χρησιμοποιώντας την εντροπία του δείγματος και στη συνέχεια κανονικοποιημένα τμήματα διάρκειας 100ms τροφοδοτούνται στο CNN για εκπαίδευση.
Οι συγγραφείς συνέκριναν τη μέθοδο τους με τα KNN, Naive Bayes και SVM επιτυγχάνοντας καλύτερα αποτελέσματα.

\subsubsection{Άλλες εφαρμογές με μη-δημόσιες βάσεις δεδομένων}
Η ανίχνευση μη-φυσιολογικών ECG μελετήθηκε από μια σειρά δημοσιεύσεων.
Οι Ripoll et al.~\cite{ripoll2016ecg} χρησιμοποίησαν προεκπαιδευμένα μοντέλα με ECG από 1390 ασθενείς, για να αξιολογήσουν εάν ένας ασθενής στο ασθενοφόρο ή στα επείγοντα πρέπει να παραπεμφθεί σε μια καρδιολογική υπηρεσία.
Σύγκριναν το μοντέλο τους με τα KNN, SVM, μηχανές ακραίας μάθησης (Extreme Learning Machines, ELMs) και με ένα σύστημα εμπειρογνωμόνων πετυχαίνοντας καλύτερα αποτελέσματα στην ακρίβεια και την εξειδίκευση.
Στο~\cite{jin2017classification} οι συγγραφείς εκπαίδευσαν ένα μοντέλο το οποίο ταξινομεί φυσιολογικούς και μη-φυσιολογικούς ασθενείς με 193690 αρχεία ECG 10 έως 20 δευτερολέπτων.
Το μοντέλο τους αποτελείται από δύο παράλληλα μέρη; την στατιστική μάθηση και ένα κανόνα συμπεράσματος.
Κατά τη διάρκεια της μάθησης, τα ECG υπόκεινται σε προεπεξεργασία με χρήση χαμηλοπερατών και ζωνοπερατών φίλτρων, στη συνέχεια τροφοδοτούνται σε δύο παράλληλα lead-CNNs και τέλος χρησιμοποιείται Bayesian σύντηξη για τον συνδυασμό των εξόδων πιθανότητας.
Κατά τη διάρκεια του συμπερασμού, ανιχνεύονται οι θέσεις κορυφής R στο αρχείο ECG και χρησιμοποιούνται τέσσερις κανόνες ασθένειας για την ανάλυση.
Τέλος, χρησιμοποιούν το μέσο όρο bias για τον προσδιορισμό του αποτελέσματος.

Άλλες εφαρμογές περιλαμβάνουν την ταξινόμηση της πρόωρης κοιλιακής συστολής και την ανίχνευση του στρες.
Οι Liu et al.~\cite{liu2018detecting} χρησιμοποίησαν ένα σύνολο δεδομένων μονού καναλιού με 2400 φυσιολογικά και πρόωρης κοιλιακής συστολής ECG από το Νοσοκομείο Παίδων της Σαγκάης για εκπαίδευση.
Δύο διαφορετικά μοντέλα εκπαιδεύτηκαν χρησιμοποιώντας τις εικόνες των κυματομορφών.
Το πρώτο ήταν ένα CNN δύο \textit{επιπέδων} με dropout και το δεύτερο ένα Inception-v3 εκπαιδευμένο στην Imagenet.
Άλλα τρία μοντέλα εκπαιδεύτηκαν χρησιμοποιώντας τα 1D σήματα.
Το πρώτο μοντέλο ήταν ένα FNN με dropout, το δεύτερο ένα 1D CNN τριών \textit{επιπέδων} και το τρίτο ένα 2D CNN το ίδιο με το πρώτο αλλά εκπαιδευμένο με μια στοιβαγμένη έκδοση του σήματος (με επαύξηση δεδομένων).
Τα πειράματα τους έδειξαν ότι το 1D CNN με τα τρία \textit{επίπεδα} είχαν καλύτερα και πιο σταθερά αποτελέσματα.
Στο~\cite{hwang2018deep} οι συγγραφείς εκπαίδευσαν ένα δίκτυο με ένα συνελικτικό επίπεδο με dropout, ακολουθούμενο από δύο RNNs για τον εντοπισμό του άγχους χρησιμοποιώντας βραχυπρόθεσμα δεδομένα ECG\@.
Έδειξαν ότι το δίκτυό τους πέτυχε τα καλύτερα αποτελέσματα σε σύγκριση με παραδοσιακές μεθόδους μάθησης μηχανών και DNN βάσης αναφοράς.

\subsubsection{Συνολική άποψη για την χρήση της βαθιάς μάθησης στο ECG}
Πολλές μέθοδοι βαθιάς μάθησης έχουν χρησιμοποιήσει ECG για την εκπαίδευση μοντέλων χρησιμοποιώντας τον μεγάλο αριθμό διαθέσιμων βάσεων δεδομένων.
Είναι προφανές από τη βιβλιογραφία ότι οι περισσότερες μέθοδοι βαθιάς μάθησης (κυρίως CNN και SDAE) σε αυτή την περιοχή αποτελούνται από τρία μέρη: φιλτράρισμα για αποθορυβοποίηση, ανίχνευση κορυφών R για κατάτμηση χτύπων και ένα νευρωνικό δίκτυο για την εξαγωγή χαρακτηριστικών.
Ένα άλλο δημοφιλές σύνολο μεθόδων είναι η μετατροπή των ECG σε εικόνες, για την αξιοποίηση ενός ευρύ φάσματος των αρχιτεκτονικών και των προεκπαιδευμένων μοντέλων που έχουν ήδη κατασκευαστεί για τις μορφές απεικόνισης.
Αυτό έγινε χρησιμοποιώντας τεχνικές ανάλυσης φάσματος~\cite{luo2017patient, xia2018detecting} και μετατροπών σε δυαδική εικόνα~\cite{xiao2018monitoring, liu2018detecting, isin2017cardiac}.

\subsection{Φωνοκαρδιογράφημα με χρήση της Physionet 2016}
Ο διαγωνισμός της Physionet/Computing στην Καρδιολογία (Cinc) 2016 (PHY16) αφορούσε την ταξινόμηση των φυσιολογικών/παθολογικών καρδιακών ηχογραφήσεων.
Το δεδομένα εκπαίδευσης αποτελούνται από πέντε βάσεις δεδομένων (Α έως Ε) που περιέχουν 3126 φωνοκαρδιογραφήματα (PCGs), τα οποία διαρκούν από 5 δευτερόλεπτα έως 120 δευτερόλεπτα.

Οι περισσότερες από τις μεθόδους μετατρέπουν τα PCG σε εικόνες χρησιμοποιώντας τεχνικές φασματογραφίας.
Οι Rubin et al.~\cite{rubin2017recognizing} χρησιμοποίησαν ένα κρυφό ημι-Markov μοντέλο για την κατάτμηση της έναρξης κάθε παλμού της καρδιάς, το οποίο στη συνέχεια μετατράπηκε σε φασματογράφημα με τη χρήση μετρητών συχνότητας Mel-Frequency Cepstral (MFCCs).
Κάθε φασματογράφημα ταξινομήθηκε σε φυσιολογικό ή μη-φυσιολογικό χρησιμοποιώντας ένα CNN δύο \textit{επιπέδων} με τροποποιημένη συνάρτηση απώλειας που μεγιστοποιεί την ευαισθησία και την εξειδίκευση, μαζί με μια παράμετρο συστηματοποίησης.
Η τελική ταξινόμηση του σήματος ήταν η μέση πιθανότητα όλων των πιθανοτήτων των τμημάτων.
Πέτυχαν συνολική ακρίβεια 83.99\% τοποθετώντας την μέθοδο όγδοη κατά τη διάρκεια του διαγωνισμού στο PHY16.
Οι Kucharski et al.~\cite{kucharski2017deep} χρησιμοποίησαν ένα φασματογράφημα οκτώ δευτερολέπτων για τα τμήματα, πριν τροφοδοτηθούν σε ένα CNN πέντε \textit{επιπέδων} με dropout.
Η μέθοδος τους πέτυχε ευαισθησία 99.1\% και ειδικότητα 91.6\% οι οποίες είναι συγκρίσιμες με αποτελέσματα προηγούμενων μεθόδων.
Οι Dominguez et al.~\cite{dominguez2018deep} ταξινόμησαν τα σήματα και τα προεπεξεργάστηκαν χρησιμοποιώντας τον νευρομορφικό ακουστικό αισθητήρα~\cite{jimenez2017binaural} για να αποσυνθέσουν τις πληροφορίες ήχου σε ζώνες συχνοτήτων.
Στη συνέχεια, υπολογίζουν τα φασματογραφήματα που τροφοδοτούνται σε μια τροποποιημένη έκδοση του δικτύου AlexNet.
Το μοντέλο τους πέτυχε ακρίβεια 94.16\%, η οποία είναι μια σημαντική βελτίωση σε σύγκριση με το μοντέλο που κέρδισε το PHY16.
Στο~\cite{potes2016ensemble} οι συγγραφείς χρησιμοποίησαν το Adaboost το οποίο τροφοδοτήθηκε με χαρακτηριστικά φασματογραφίας από PCG και ένα CNN το οποίο εκπαιδεύτηκε χρησιμοποιώντας καρδιακούς κύκλους αποσυνθεμένους σε τέσσερις ζώνες συχνοτήτων.
Τέλος, οι έξοδοι του Adaboost και του CNN συνδυάστηκαν για να παράξουν το τελικό αποτέλεσμα ταξινόμησης χρησιμοποιώντας έναν απλό κανόνα απόφασης.
Η συνολική ακρίβεια ήταν 89\%, τοποθετώντας τη μέθοδο πρώτη στο διαγωνισμό του PHY16.

Τα μοντέλα που δεν μετατρέπουν τα PCG σε φασματογραφήματα φαίνεται να έχουν μικρότερη απόδοση.
Οι Ryu et al.~\cite{ryu2016classification} εφήρμοσαν ένα φίλτρο με παράθυρο Hamming Window-sinc για αποθορυβοποίηση, έπειτα κλιμάκωσαν το σήμα και χρησιμοποίησαν ένα σταθερό παράθυρο για κατάτμηση.
Εκπαίδευσαν ένα 1D CNN τεσσάρων \textit{επιπέδων} χρησιμοποιώντας τα τμήματα ενώ η τελική ταξινόμηση ήταν ο μέσος όρος όλων των πιθανοτήτων των τμημάτων.
Επιτεύχθηκε συνολική ακρίβεια 79.5\% στην επίσημη φάση του PHY16.

Τα φωνοκαρδιογραφήματα έχουν επίσης χρησιμοποιηθεί για προβλήματα όπως η αναγνώριση ήχου καρδιάς S1 και S2 από τον Chen et al.~\cite{chen2017s1}.
Μετασχημάτισαν τα ηχητικά σήματα της καρδιάς σε μια ακολουθία MFCCs και έπειτα εφάρμοσαν Κ-μέσους για να συσσωρεύσουν τα χαρακτηριστικά MFCC σε δύο ομάδες για να βελτιώσουν την εκπροσώπηση και τη διακριτική τους ικανότητα.
Τα χαρακτηριστικά τροφοδοτούνται στη συνέχεια σε ένα DBN για την εκτέλεση ταξινόμησης S1 και S2.
Οι συγγραφείς συνέκριναν τη μέθοδο τους με τα μοντέλα KNN, Gaussian mixture, λογιστική παλινδρόμηση και SVM επιτυγχάνοντας καλύτερα αποτελέσματα.

Σύμφωνα με τη βιβλιογραφία, τα CNN αποτελούν την πλειονότητα των αρχιτεκτονικών νευρωνικών δικτύων που χρησιμοποιήθηκαν για την επίλυση προβλημάτων με PCG\@.
Επιπλέον, όπως και στο ECG, πολλές μέθοδοι βαθιάς μάθησης μετασχηματίζουν τα σήματα σε εικόνες χρησιμοποιώντας τεχνικές φασματογραφίας~\cite{potes2016ensemble, rubin2017recognizing, kucharski2017deep, dominguez2018deep, pan2017variation, shashikumar2017deep}.

\subsection{Άλλα σήματα}
\subsubsection{Παλμομετρικά δεδομένα}
Τα παλμομετρικά δεδομένα χρησιμοποιούνται για την εκτίμηση της SBP και της DBP που είναι οι αιμοδυναμικές πιέσεις που ασκούνται στο αρτηριακό σύστημα κατά τη διάρκεια της συστολής και της διαστολής αντίστοιχα~\cite{everly2012clinical}.

Τα DBN έχουν χρησιμοποιηθεί για την εκτίμηση των SBP και DBP\@.
Στο άρθρο τους οι Lee et al.~\cite{lee2017deepa} χρησιμοποίησαν bootstrap-aggregation για να δημιουργήσουν ensemble παραμέτρους και χρησιμοποίησαν τον Adaboost για την εκτίμηση των SBP και DBP\@.
Στη συνέχεια, χρησιμοποίησαν bootstrap και Monte-Carlo για να προσδιορίσουν τα διαστήματα εμπιστοσύνης βασισμένα στο BP, τα οποία εκτιμήθηκαν χρησιμοποιώντας τον ensemble εκτιμητή παλινδρόμησης του DBN\@.
Αυτή η τροποποίηση βελτίωσε σημαντικά την εκτίμηση του BP σε σχέση με το βασικό μοντέλο DBN\@.
Παρόμοια κατεύθυνση έχει ακολουθηθεί για το ίδιο πρόβλημα από τους ίδιους συγγραφείς στα~\cite{lee2017oscillometric, lee2017deepc, lee2017deepd}.

Παλμομετρικά δεδομένα έχουν επίσης χρησιμοποιηθεί από τους Pan et al.~\cite{pan2017variation} για την εκτίμηση της μεταβολής των ήχων Korotkoff.
Οι χτύποι χρησιμοποιήθηκαν για να δημιουργήσουν παράθυρα κεντραρισμένα στις κορυφές των παλμών, οι οποίες μετά εξήχθησαν.
Ανάλυση φάσματος λήφθηκε από κάθε χτύπο και όλοι οι χτύποι μεταξύ των χειροκίνητα επισημασμένων SBPs και DBPs επισημάνθηκαν ως Korotkoff.
Στη συνέχεια χρησιμοποιήθηκε ένα CNN τριών \textit{επιπέδων}, για την ανάλυση της συνέπειας στα ηχητικά μοτίβα που συσχετίστηκαν με τους ήχους Korotkoff.
Σύμφωνα με τους συγγραφείς, αυτή ήταν η πρώτη μελέτη που διεξήχθη για τέτοιου είδους πρόβλημα, αποδεικνύοντας ότι είναι δύσκολο να προσδιοριστούν οι ήχοι Korotkoff στη συστολή και διαστολή.

\subsubsection{Δεδομένα από φορητές συσκευές}
Οι φορητές συσκευές, οι οποίες επιβάλλουν περιορισμούς στο μέγεθος, την ισχύ και την κατανάλωση μνήμης για τα μοντέλα, έχουν επίσης χρησιμοποιηθεί για τη συλλογή δεδομένων καρδιολογίας για εκπαίδευση μοντέλων βαθιάς μάθησης ανίχνευσης AF\@.

Οι Shashikumar et al.~\cite{shashikumar2017deep} πήραν ECG, φωτοπληθυσμογραφία (PPG) και δεδομένα από επιταχυνσιόμετρο από 98 άτομα χρησιμοποιώντας μια φορητή συσκευή καρπού και παρήξαν την ανάλυση φάσματος χρησιμοποιώντας συνεχή WT\@.
Εκπαίδευσαν ένα CNN πέντε \textit{επιπέδων} σε μια σειρά βραχέων παραθύρων με θόρυβο κίνησης και η έξοδός τους συνδυάστηκε με χαρακτηριστικά που υπολογίστηκαν βάσει της μεταβλητότητας του χτύπου-προς-χτύπο και του δείκτη ποιότητας του σήματος.
Η μέθοδος πέτυχε ακρίβεια 91.8\% στην ανίχνευση AF και σε συνδυασμό με την υπολογιστική αποτελεσματικότητά της είναι πολλά υποσχόμενη για κλινική εφαρμογή σύμφωνα με τους συγγραφείς.
Οι Gotlibovych et al.~\cite{gotlibovych2018end} εκπαίδευσαν ένα δίκτυο CNN ενός \textit{επιπέδου} ακολουθούμενο από ένα LSTM, χρησιμοποιώντας 180 ώρες PPG από δεδομένα φορητών συσκευών για την ανίχνευση AF\@.
Η χρήση του επιπέδου LSTM επιτρέπει στο δίκτυο να μαθαίνει συσχετίσεις μεταβλητού μήκους σε αντίθεση με το σταθερό μήκος του συνελικτικού επιπέδου.
Οι Poh et al.~\cite{poh2018diagnostic} δημιούργησαν μια μεγάλη βάση δεδομένων PPG (πάνω από 180000 σήματα από 3373 ασθενείς), συμπεριλαμβανομένων των δεδομένων από την MIMIC για την ταξινόμηση τεσσάρων ρυθμών.
Ένα πυκνά συνδεδεμένο (densely) CNN με έξι \textit{επίπεδα} χρησιμοποιήθηκε για ταξινόμηση, το οποίο τροφοδοτήθηκε με τμήματα 17 δευτερολέπτων που έχουν αποθορυβοποιηθεί χρησιμοποιώντας ένα ζωνοπερατό φίλτρο.
Τα αποτελέσματα ελήφθησαν χρησιμοποιώντας ένα ανεξάρτητο σύνολο δεδομένων 3039 PPG πετυχαίνοντας καλύτερα αποτελέσματα από τις προηγούμενες μεθόδους που βασίστηκαν σε χειροποίητα χαρακτηριστικά.

Εκτός από την ανίχνευση AF, δεδομένα από φορητές συσκευές χρησιμοποιήθηκαν και για την αναζήτηση καλύτερων προγνωστικών καρδιοαγγειακών παθήσεων.
Στο~\cite{ballinger2018deepheart} οι συγγραφείς εκπαίδευσαν ένα ημι-επιβλεπώμενο, διπλής κατεύθυνσης LSTM σε δεδομένα από 14011 χρήστες της εφαρμογής Cardiogram για την ανίχνευση του διαβήτη, της υψηλής χοληστερόλης, της υψηλής BP και της άπνοιας.
Τα αποτελέσματά τους δείχνουν ότι η ανταπόκριση της καρδιάς στη σωματική δραστηριότητα είναι ένας σημαντικός βιοδείκτης για την πρόβλεψη της εμφάνισης μιας νόσου και μπορεί να εντοπιστεί χρησιμοποιώντας βαθιά μάθηση.

\begin{sidewaystable}
	\centering
	\caption{Εφαρμογές βαθιάς μάθησης με χρήση PCG και άλλων σημάτων}
	\label{table:signals4}
	\begin{tabular}{l c l l}
		\toprule
		\thead{Αναφορά}                             & \thead{Μέθοδος} & \thead{Εφαρμογή/Σημειώσεις\footnote{Σε παρένθεση οι βάσεις δεδομένων που χρησιμοποιήθηκαν. Στον υποπίνακα `Διαγωνισμός PCG/Physionet 2016' όλες οι δημοσιεύσεις χρησιμοποιούν την PHY εκτός του~\cite{chen2017s1}, και στον υποπίνακα `Άλλα Σήματα' όλες οι δημοσιεύσεις χρησιμοποιούν μη-δημόσιες βάσεις δεδομένων εκτός του~\cite{poh2018diagnostic}.}} & \thead{Ακρίβεια\footnote{Υπάρχει μεγάλη μεταβλητότητα στην αναφορά αποτελεσμάτων. Το~\cite{kucharski2017deep} αναφέρουν specificity, το~\cite{pan2017variation} αναφέρει αποτελέσματα για το SBP και το DBP, το~\cite{gotlibovych2018end} αναφέρει ευαισθησία, ειδικότητα, το~\cite{poh2018diagnostic} αναφέρει την τιμή της θετικής προβλεψιμότητας,το~\cite{ballinger2018deepheart} αναφέρει AUC για το diabetest, αποτελέσματα επίσης αναφέρονται για υψηλή χοληστερόλη, άπνοια και υψηλό BP.}} \\
		\midrule
		\multicolumn{4}{l}{\thead{Διαγωνισμός PCG/Physionet 2016}}                                                                                                                                                                                                                                                                                                                                                                                                                                                                                                                                                                                                                                                                                                                                                                                                                                                                                           \\
		\midrule
		Rubin 2017~\cite{rubin2017recognizing}       & CNN             & λογιστική παλινδρόμηση με ένα κρυφό ημι-Markov μοντέλο, MFCCs και ένα CNN δύο \textit{επιπέδων}                                                                                                                                                                                                                                                         & 83.99\%                                                                                                                                                                                                                                                                                                                                                                                                                                                                                                                    \\
		Kucharski 2017~\cite{kucharski2017deep}      & CNN             & ανάλυση φάσματος και ένα CNN πέντε~\textit{επιπέδων} CNN με dropout                                                                                                                                                                                                                                                                                      & 91.6\%\textsuperscript{b}                                                                                                                                                                                                                                                                                                                                                                                                                                                                                             \\
		Dominguez 2018~\cite{dominguez2018deep}      & CNN             & ανάλυση φάσματος και τροποποιημένο AlexNet                                                                                                                                                                                                                                                                                                              & 94.16\%                                                                                                                                                                                                                                                                                                                                                                                                                                                                                                                    \\
		Potes 2016~\cite{potes2016ensemble}          & CNN             & ensemble από Adaboost και CNN, η έξοδος συνδυάζεται με ένα κανόνα απόφασης                                                                                                                                                                                                                                                                               & 86.02\%                                                                                                                                                                                                                                                                                                                                                                                                                                                                                                                    \\
		Ryu 2016~\cite{ryu2016classification}        & CNN             & φίλτρα αποθορυβοποίησης και ένα CNN τεσσάρων \textit{επιπέδων}                                                                                                                                                                                                                                                                                          & 79.5\%                                                                                                                                                                                                                                                                                                                                                                                                                                                                                                                     \\
		Chen 2017~\cite{chen2017s1}                  & DBN             & αναγνώριση S1 και S2 ήχων καρδιάς με χρήση MFCCs, K-means και DBN (μη-δημόσια)                                                                                                                                                                                                                                                                          & 91\%                                                                                                                                                                                                                                                                                                                                                                                                                                                                                                                       \\
		\midrule
		\multicolumn{4}{l}{\thead{Άλλα σήματα}}                                                                                                                                                                                                                                                                                                                                                                                                                                                                                                                                                                                                                                                                                                                                                                                                                                                                                                            \\
		\midrule
		Lee 2017~\cite{lee2017deepa}                 & DBN             & εκτίμηση BP με χρήση bootstrap-aggregation, Monte-Carlo και DBN με παλμομετρικά δεδομένα                                                                                                                                                                                                                                                                & \textit{πολλαπλά}                                                                                                                                                                                                                                                                                                                                                                                                                                                                                                          \\
		Pan 2017~\cite{pan2017variation}             & CNN             & εύρεση ήχων Korotkoff με χρήση ενός CNN τριών \textit{επιπέδων} με παλμομετρικά δεδομένα                                                                                                                                                                                                                                                                & \textit{πολλαπλά}\textsuperscript{b}                                                                                                                                                                                                                                                                                                                                                                                                                                                                                  \\
		Shashikumar 2017~\cite{shashikumar2017deep}  & CNN             & ανίχνευση AF με χρήση ECG, φωτοπληθυσμογραφία, και επιταχυνσιόμετρο με WT και ένα CNN                                                                                                                                                                                                                                                                   & 91.8\%                                                                                                                                                                                                                                                                                                                                                                                                                                                                                                                     \\
		Gotlibovych 2018~\cite{gotlibovych2018end}   & CNN, LSTM       & ανίχνευση AF με χρήση PPG από φορητή συσκευή και ένα LSTM-CNN                                                                                                                                                                                                                                                                                           & \textgreater{99\%}\textsuperscript{b}                                                                                                                                                                                                                                                                                                                                                                                                                                                                                 \\
		Poh 2018~\cite{poh2018diagnostic}            & CNN             & ανίχνευση τεσσάρων ρυθμών με PPG και ένα densely CNN (MIMIC, VORTAL, PPGDB)                                                                                                                                                                                                                                                                             & 87.5\%\textsuperscript{b}                                                                                                                                                                                                                                                                                                                                                                                                                                                                                             \\
		Ballinger 2018~\cite{ballinger2018deepheart} & LSTM            & πρόβλεψη διαβήτη, υψηλής χοληστερόλης, BP και άπνοιας από αισθητήρα                                                                                                                                                                                                                                                     & 0.845\textsuperscript{b}                                                                                                                                                                                                                                                                                                                                                                                                                                                                                                         \\
		\bottomrule
	\end{tabular}
\end{sidewaystable}

\clearpage
\bibliography{chapter3.bib}
\bibliographystyle{unsrt}
