\chapter{Επίλογος}
\label{chapter7}
Με κάθε τεχνολογική πρόοδο η ιατρική, η οποία μέχρι τώρα είναι βαθιά εξαρτώμενη απο τον ανθρώπινο παράγοντα, έρχεται πιο κοντά σε ένα αυτοματοποιημένο τομέα κατευθυνόμενο από την τεχνητή νοημοσύνη.
Το AI όχι μόνο θα φτάσει στο σημείο που θα ανιχνεύει ασθένειες σε πραγματικό χρόνο, αλλά θα ερμηνεύει επίσης αμφιλεγόμενες καταστάσεις, φαινοτυπικές πολύπλοκες ασθένειες και θα παίρνει ιατρικές αποφάσεις.
Ωστόσο, η πλήρης θεωρητική κατανόηση της βαθιάς μάθησης δεν είναι ακόμη διαθέσιμη και η κριτική κατανόηση των πλεονεκτημάτων και των περιορισμών της εσωτερικής λειτουργίας της είναι ζωτικής σημασίας για να κερδίσει τη θέση της στην καθημερινή κλινική χρήση.
Η επιτυχής εφαρμογή του AI στον ιατρικό τομέα βασίζεται στην επίτευξη ερμηνεύσιμων μοντέλων και δημιουργία μεγάλων βάσεων δεδομένων.

Στο πλαίσιο της διδακτορικής διατριβής προτείναμε το μέτρο $\varphi$ για να αξιολογήσουμε πόσο καλά τα μοντέλα μηχανικής μάθησης ανταλλάσσουν απώλεια ανακατασκευής με συμπίεση.
Έπειτα προτείναμε μια νέα αρχιτεκτονική νευρωνικών δικτύων τα \textbf{SANs} τα οποία έχουν ελάχιστη δομή και με την χρήση των συναρτήσεων αραιής ενεργοποίησης μαθαίνουν να συμπιέζουν δεδομένα χωρίς να χάνουν σημαντικές πληροφορίες.
Χρησιμοποιώντας τις βάσεις δεδομένων Physionet και MNIST αποδείξαμε ότι τα SANs είναι σε θέση να δημιουργούν αναπαραστάσεις υψηλής ποιότητας με ερμηνεύσιμους πυρήνες.

Η ελάχιστη δομή των SANs καθιστά εύκολη τη χρήση τους για την εξαγωγή χαρακτηριστικών, την ομαδοποίηση και την πρόβλεψη χρονοσειρών.
Άλλες μελλοντικές εργασίες σχετικά με τα SANs περιλαμβάνουν:
\begin{itemize}
	\item Εφαρμογή αλγορίθμων σταδιακής μείωσης των ελάχιστων αποστάσεων των ακρότατων, για την αύξηση του βαθμού ελευθερίας των πυρήνων.
	\item Επιβολή ελάχιστης απόστασης ακρότατων κατά μήκος όλων των πινάκων ομοιότητας για πολλαπλούς πυρήνες, κάνοντας έτσι τους πυρήνες να ανταγωνίζονται για περιοχές.
	\item Εφαρμογή του dropout στις ενεργοποιήσεις για να διορθωθούν τα βάρη που έχουν υπερβεί, ειδικά όταν αρχικοποιούνται με υψηλές τιμές.
		Ωστόσο, η επίδραση του dropout στα SAN θα ήταν γενικά αρνητική, καθώς τα SANs έχουν πολύ μικρότερο αριθμό βαρών από τα DNNs και επομένως δεν χρειάζονται ισχυρή κανονικοποίηση.
	\item Χρήση των SAN με δυναμικά δημιουργούμενους πυρήνες οι οποίοι θα μπορούσαν να μάθουν πολυτροπικά δεδομένα από μεταβλητές πηγές (e.g.\ από ECG σε αναπνευστικά) χωρίς να καταστρέψουν προηγούμενα βάρη.
\end{itemize}
